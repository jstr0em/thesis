\documentclass{article}

\usepackage{amsmath}
\usepackage[utf8]{inputenc}


\begin{document}

\section{Mass Transport in Partially Saturated Porous Media}

The vadose zone is a three-phase system and thus any chemical specie is distributed between these three phases.
However, the mass transport still only occur through the gas and liquid phases of the system, therefore the transport of the \textit{total} concentration $c_T$ is due to diffusive and advective transport in these phases.
\begin{equation}\label{eq:mass1}
  \frac{\partial c_T}{\partial t} = \nabla [D_w \theta_w \tau_w \nabla c_w + D_g \theta_g \tau_g \nabla c_g]
    - \nabla (v_w \theta_w c_w + v_g \theta_g c_g)
\end{equation}
Here $D_w$ and $D_g$ are the water and gas diffusion constants respectively; $\theta_w$ and $\theta_g$ are the water and gas filled porosities; $\tau_w$ and $\tau_g$ are the tortuosity, correcting for diffusivity in porous media; $v_w$ and $v_g$ are the water and gas velocity; finally $c_w$ and $c_g$ are the water and gas phase concentrations.\par

As stated above, the total concentration is distributed across the three phases.
\begin{equation}
  c_T = \theta_w c_w + c_g \theta_g + c_s \rho_b
\end{equation}
Where the first and second terms correspond to the water and gas concentrations; the third correspond to the sorbed concentration, where $c_s$ is the sorbed concentration by mass and $\rho_b$ is the soil bulk density.
In order to solve \eqref{eq:mass1} we need to state everything in terms of one dependent variable, which we will see is the water concentration $c_w$.

From Henry's Law we know that a gas concentration is proportional to the water concentration via the eponymous constant $H$.
\begin{equation}
  c_g = H c_w
\end{equation}
By assuming linear sorption we can describe the sorbed concentration as
\begin{equation}
  c_s = \begin{cases}
    K_p c_w &\text{Water phase sorption} \\
    K_p c_g = K_p H c_w &\text{Gas phase sorption}
  \end{cases}
\end{equation}
where $K_p$ is the sorption isotherm.
For simplicity we will here assume water phase sorption.

Using this we can restate $c_T$ is terms of the water phase concentration.
\begin{equation}
  c_T = (\theta_w + \theta_g H + K_p \rho_b) c_w = R c_w
\end{equation}
The \textit{retardation factor} $R$ is introduced to simplify writing.

Now we substitute all of this in \eqref{eq:mass1}.
\begin{equation}\label{eq:mass2}
  R \frac{\partial c_w}{\partial t} = \nabla [(D_w \theta_w \tau_w + D_g \theta_g \tau_g H)\nabla c_w]
    - \nabla [(v_w \theta_w + v_g \theta_g H) c_w]
\end{equation}
Here we recognize that $(D_w \theta_w \tau_w + D_g \theta_g \tau_g H)$ is the effective diffusivity $D_\mathrm{eff}$, which gives the final expression
\begin{equation}\label{eq:mass3}
  R \frac{\partial c_w}{\partial t} = \nabla [D_\mathrm{eff} \nabla c_w]
    - \nabla [(v_w \theta_w + v_g \theta_g H) c_w]
\end{equation}

Most soil-physics books are concerned with water moving in porous media, with the gas assumed to be immobile and occupy small pockets in the porous media.
In this case $v_g = 0$, dropping that term which gives
\begin{equation}\label{eq:mass4}
  R \frac{\partial c_w}{\partial t} = \nabla [D_\mathrm{eff} \nabla c_w]
    - \nabla [v_w \theta_w c_w]
\end{equation}
This is the most common form found of the governing equation for mass transport in partially saturated porous media, and the equation that COMSOL solves.

Obviously this does not quite describe the vapor intrusion scenario, where we are concerned with a mobile gas phase and a stationary water phase.
Although, this does not necessarily have to be the case, and we could in theory keep both velocity fields if we were interested in such an problem.
Regardless, for most of our applications we assume that the soil water is stationary $v_w = 0$ leading to
\begin{equation}\label{eq:mass5}
  R \frac{\partial c_w}{\partial t} = \nabla [D_\mathrm{eff} \nabla c_w]
    - \nabla [v_g \theta_g H c_w]
\end{equation}
The implications of this is that we must multiply the gas velocity field with the Henry's Law constant to correctly reflect the transport problem.

Another implication is that \textbf{we must set all our boundary conditions in terms of the water phase concentration} $c_w$.
So for one, the concentration boundary condition at the groundwater source must be the groundwater \textit{water} concentration, and not the typical one where we multiply it by $H$; as we've seen, this previous correction is built into the governing equation.

The crack entry flux $j_{ck}$ must also be adjusted.
This one is a bit trickier though, since we're only concerned the gas phase concentration entering through the crack.
Thus it must be stated as a function of the gas phase concentration, i.e. $j_{ck} = f(c_g)$ and this is the contaminant flux that enters the overlying building.
But since we must state every boundary condition in terms of $c_w$, we must scale the boundary condition in the model using Henry's Law as well, and thus it should be $j_{ck}/H$.

\end{document}
