\documentclass[../thesis.tex]{subfiles}

\chapter*{Abstract}

The significant temporal variability of vapor intrusion (VI) complicate site investigations and increase the chances of over- or underestimating human exposure to indoor air contaminants.
Conducting longer and more temporally high-resolution studies of suspected VI sites is one way to address this issue.
However, this approach is undesirable as these types of studies can be more technically difficult and exceedingly costly.
Improving our understanding of the factors that influence this temporal variability is therefore essential for accurate assessment of human exposure to these contaminants.\par

This is challenging due to the complex nature of VI and the heterogeneity of VI sites, leading to highly non-linear and often unique systems.
Adding to this is the fact that only a few long-term and high-resolution VI site studies exist, further rendering a robust understanding of the drivers the temporal variability in VI elusive.
In response to this, techniques such as a the controlled pressure method or the use of seasonal correction factors based on statistical analysis of VI sites, have been developed to help heuristically determine the relevant human exposure.
But none of these circumvent the need for a careful understanding of the drivers of VI, without which new mischaracterization may be introduced.\par

By implementing the physics that govern VI from first-principles in three-dimensional finite element models, we gain complete control of the VI phenomena.
This allows us to investigate the role of individual factors and site characteristics that influence VI.
Using these models in combination with statistical analysis of the aforementioned VI high-resolution datasets, we gain an opportunity to investigate the highly dynamic effects of building pressurization and sorption on transient variability in VI.
As well as how these dynamic processes interplay with external weather factors in the short-term and in the long-term.\par

% TODO: Add paragraph 4
% -- What are the consequences of your choice/thesis?
% - What is the result/impact?
% - What does this enable?
