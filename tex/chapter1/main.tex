\documentclass[../thesis.tex]{subfiles}

\begin{document}
\chapter{Introduction}


\section{Indoor Air Quality Perspective}
% 1. Historic/broad background of VI and indoor air quality in general
% - Spend lots of time indoors
% - Summary of various VI pollutants (graphic/table?)

Concerns about air quality are as old as civilization itself, ranging from beliefs that disease is caused by bad air - a \textit{miasma}, to more recent concerns about exposure to combustion particulates.
Since industrialization the number of potential hazardous pollutants has increased significantly, followed by increased concerns about air quality.
At the same time, people now spend more time indoors now than ever before, with Americans spending up to 90\% of their waking time indoors\cite{klepeis_national_2001}.
This change in human habitation has put a special emphasis on indoor air quality.

Some early scientific inquiries into indoor air quality focused on pollutant sources that were generated in the home, such as heating and cooking systems.
Increased levels of carbon monoxide, sulfur dioxide, and various nitrous oxides could be detected due to these systems\cite{craig_d._hollowell_combustion-generated_1976}.
These types of pollutants are still relevant today, but of particular concern in developing countries\cite{world_health_organisation_who_2014}.

Some indoor air pollutants may be emitted from building materials, such as asbestos from older types of insulation, or formaldehyde from pressed-wood products, and various petroleum or chlorinated compounds may be emitted from common household product. % TODO: Cite these examples
Molds are another common indoor quality concern\cite{world_health_organisation_who_2009}.

Indoor air quality may be significantly affected by external contaminant sources as well, with Radon being perhaps one of the most well-known cases of this.
Radon is a radioactive gas produced by the natural decay of Uranium 238, which is commonly found in many rocks and minerals.
These in turn are often found in the soil or bedrock and the Radon gas enter the overlying buildings, exposing the inhabitants.

\section{Vapor Intrusion}
% 2. Background on...
% - How VI works (graphics/flowsheet)
% - How VI site investigations are conducted
% - Some of the current issues

A recent addition to the myriad of indoor air quality concerns is vapor intrusion (VI) - a process that is similar to Radon intrusion.
As with Radon intrusion, VI contaminants are typically originate from underneath a building, migrate through the soil, entering it, and exposing the inhabitants.
VI is different from Radon intrusion because it is more broad and general; the contaminant sources are usually of anthropogenic origin, and not limited to the soil.
As such, Radon intrusion can be thought of as a subset of VI, but they are typically considered separately.
In VI, the primary contaminants of concern are chlorinated solvents, e.g. trichloroethylene (TCE), tetrachlorethylene (PCE), vinyl chloride, chloroform, and organic compounds, e.g. benzene.

The prototypical VI scenario is one where groundwater has been contaminated with one or many of these contaminants.
Contaminant vapors evaporate from the groundwater, and are transported through the soil into the overlying buildings through small cracks and breaches in the foundation.
While this simple example is helpful for gaining a rudimentary understanding of VI, it fails to capture much of the complexities of an actual VI site.

% Some issues with VI investigations
Determining if vapor intrusion occurs at a house or structure is often difficult.
One might be tempted to believe that taking an air sample inside the house would be sufficient, i.e. that vapor contaminant concentrations is over some threshold in the house is proof of VI; absence of contaminant vapors is proof of no VI.
The reality is that indoor air samples can be problematic for a few reasons.
Due to their distributive nature, the residents or owners of the structure may be unwilling to let indoor samples to be taken.
Indoor air samples are also susceptible to false positives and negatives.

% False positives
Many common consumer products contain the same contaminants that is often of concern in VI.
The presence of these contaminants in a house is thus not necessarily proof of VI but rather a line-of-evidence.
Great care should be and is taken to remove any potential indoor contaminant sources before any VI investigation can begin (contributing to the distributive nature of these investigations).

% False negatives
There can be significant temporal variability of indoor contaminant concentrations and some sites may have "active" and "inactive" periods, thus the absence of indoor contaminant is not proof that VI is not occurring, but yet another a line-of-evidence.
This temporal variability occurs on different time-scales as mean indoor contaminant concentrations often fluctuate across seasons, and may even significantly vary across weeks, days, or even within a day.

% Some other investigative techniques used
Another approach might be to collect groundwater and/or soil-gas samples, but this also has it's inherent issues as well.
The presence of contaminant in the groundwater or surrounding soil-gas (even if found right underneath the foundation) is evidence that VI occurs.
Likewise, the absence or low concentration of contaminants may only indicate that there is significant spatial variability in contaminant concentration or that the source has not been found (hidden preferential pathways may especially be issues in the latter case).
The result of these samples is the same as indoor samples, they may only be used as a line-of-evidence for VI.

% Introducing MLE concept
The combination of these line-of-evidence are usually required to prove that VI occurs; the presence of contaminant in the groundwater, in the soil-gas underneath the structure, and finally inside the structure would be good evidence that VI occurs.
This multiple line-of-evidence (MLE) approach is necessary when conducting VI investigations and is recommended by the US EPA.

\section{Research Motivation}
% 3. Motivate research




\section{Numerical Modeling of Vapor Intrusion}
% 4. Motivate modeling
% - Deterministic vs. statistical modeling
% - Benefits of modeling
% - History of modeling (briefly)

\section{Research Objective}
% 5. Define research objective

This research's broad aim is to improve our understanding of the complex fate and transport of VI contaminants, with a particular focus on the dynamic processes that drive the temporal variability in VI.
The ultimate goal is to reduce the uncertainty in VI investigations, and to make these easier and cheaper to conduct.
To achieve this we mathematically describe the processes that govern the fate and transport of VI from a first principles approach.
These are implemented and numerically solved using a finite-element method (FEM) solver package - COMSOL Multiphysics, which generate deterministic models of VI scenarios.
By combining these models with statistical analysis of high-resolution datasets from the two well-studied VI sites near Hill Air Force Base in Utah (called the ASU house from here on), and the EPA house in Indianapolis, Indiana, we gain an opportunity to explore the dynamic VI processes.\par

These models and analysis are applied to understand the role that pressure fluctuations have on determining temporal variability, and how the preferential pathway discovered at the ASU house enhanced this influence.
The influence the preferential pathway had at the EPA house is likewise investigated.
Seasonal driving forces, and in particular how temperature and wind affect building pressurization and air exchange rate, and how these can be used to explain much of the long-term seasonal changes in VI potential can be explained by these.
We also explore the role that contaminant sorption, both onto soil and various indoor materials, have on the transient response to pressure changes.
Lastly, these findings are explored in relation to the application of the controlled pressure method (CPM), and the significance of their impact on effective use of CPM.\par

\section{Outline}
% 6. Thesis outline
% - Chapter details
% - Last chapter should be future work


\end{document}
