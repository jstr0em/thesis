\documentclass[../thesis.tex]{subfiles}

\begin{document}
\chapter{Introduction}


\section{Indoor Air Quality Perspective}
% 1. Historic/broad background of VI and indoor air quality in general
% - Spend lots of time indoors
% - Summary of various VI pollutants (graphic/table?)

Concerns about air quality are as old as civilization itself, ranging from beliefs that disease is caused by bad air - a \textit{miasma}, to more recent concerns about exposure to combustion particulates.
Since industrialization the number of potential hazardous pollutants has increased significantly, followed by increased concerns about air quality.
Another effect is that people spend more time indoors now than ever before, with Americans spending up to 90\% of their waking time indoors\cite{klepeis_national_2001}.
This change in human habitation has put a special emphasis on indoor air quality.\par

Some early scientific inquiries into indoor air quality focused on pollutant sources that were generated in the home, such as heating and cooking systems.
Increased levels of carbon monoxide, sulfur dioxide, and various nitrous oxides could be detected due to these systems\cite{craig_d._hollowell_combustion-generated_1976}.
These types of pollutants are still relevant today, but of particular concern in developing countries\cite{world_health_organisation_who_2014}.\par

Some indoor air pollutants may be emitted from building materials, such as asbestos from older types of insulation, or formaldehyde from pressed-wood products, and various petroleum or chlorinated compounds may be emitted from common household product. % TODO: Cite these examples
Molds is another common indoor quality concern\cite{world_health_organisation_who_2009}.

\section{Vapor Intrusion}
% 2. Background on...
% - How VI works (graphics/flowsheet)
% - How VI site investigations are conducted
% - Some of the current issues

A recent addition to the family of indoor air quality concerns is vapor intrusion.

\section{Research Motivation}
% 3. Motivate research

\section{Numerical Modeling of Vapor Intrusion}
% 4. Motivate modeling
% - Deterministic vs. statistical modeling
% - Benefits of modeling
% - History of modeling (briefly)

\section{Research Objective}
% 5. Define research objective

\section{Outline}
% 6. Thesis outline
% - Chapter details
% - Last chapter should be future work


% Something general intro to VI

% Some issues with VI investigations
Determining if vapor intrusion occurs at a house or structure is often difficult.
One might be tempted to believe that taking an air sample inside the house would be sufficient, i.e. that vapor contaminant concentrations is over some threshold in the house is proof of VI; absence of contaminant vapors is proof of no VI.
The reality is that indoor air samples can be problematic for a few reasons.
Due to their distributive nature, the residents or owners of the structure may be unwilling to let indoor samples to be taken.
Indoor air samples are also susceptible to false positives and negatives. \par

% False positives
Many common consumer products contain the same contaminants that is often of concern in VI.
The presence of these contaminants in a house is thus not necessarily proof of VI but rather a line-of-evidence.
Great care should be and is taken to remove any potential indoor contaminant sources before any VI investigation can begin (contributing to the distributive nature of these investigations). \par

% False negatives
There can be significant temporal variability of indoor contaminant concentrations and some sites may have "active" and "inactive" periods, thus the absence of indoor contaminant is not proof that VI is not occurring, but also a line-of-evidence.
This temporal variability occurs on different time-scales as mean indoor contaminant concentrations often fluctuate across seasons, and may even significantly vary across weeks, days, or even within a day. \par

% Some other investigative techniques used
Another approach might be to collect groundwater and/or soil-gas samples, but this also has it's inherent issues as well.
The presence of contaminant in the groundwater or surrounding soil-gas (even if found right underneath the foundation) is evidence that VI occurs.
Likewise, the absence or low concentration of contaminants may only indicate that there is significant spatial variability in contaminant concentration or that the source has not been found (hidden preferential pathways may especially be issues in the latter case).
The result of these samples is the same as indoor samples, they may only be used as a line-of-evidence for VI.\par

% Introducing MLE concept
The combination of these line-of-evidence are usually required to prove that VI occurs; the presence of contaminant in the groundwater, in the soil-gas underneath the structure, and finally inside the structure would be good evidence that VI occurs.
This multiple line-of-evidence (MLE) approach is necessary when conducting VI investigations and is recommended by the US EPA.\par

\end{document}
