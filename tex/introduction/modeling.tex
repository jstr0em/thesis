\begin{comment}
Main point: Modeling is a useful tool for understanding the physical processes in VI.

- Mention some past use, advantages, and disadvantages. (Keep it brief and refer more to modeling chapter).
- Some examples again.

Mention in the end that we will use the model to investgiate some topics:
- Variability issue/ITS etc
- Mention sorption
\end{comment}


\section{Mathematical Modeling of Vapor Intrusion}\label{sec:intro_modeling}
% 3. Motivate modeling
% - Deterministic vs. statistical modeling
% - Benefits of modeling
% - History of modeling (briefly)

Early in the history of VI, mathematical models of VI were formulated to aid investigators and regulators predict human exposure risk.
Most of these adapted work done by \citeauthor{nazaroff_predicting_1988}\cite{nazaroff_predicting_1988} who had developed mathematical models of radon intrusion.
Perhaps one of the more well-used VI models was the one developed by Johnson and Ettinger in 1991\cite{johnson_heuristic_1991}.
This analytical model could be used to describe the transport of various VI contaminants from a groundwater source into the overlying building, and it was quickly adopted by the EPA as a spreadsheet tool that has seen widespread use since\cite{u.s._environmental_protection_agency_oswer_2015}.
Over time more advanced numerical models were developed, which allowed for more physics and VI scenarios to be modeled in greater detail.
Some notable examples are the work by \citeauthor{abreu_effect_2005}\cite{abreu_effect_2005} and \citeauthor{pennell_development_2009}\cite{pennell_development_2009}.\par

A benefit of using models in VI, besides as a predictive tool, is that their use allows the user to inspect in detail the role that various physics plays in determining VI.
In a field that is dominated by empirical field studies of VI sites (which are environments that are difficult to control), models are an invaluable tool that help deepening our understanding of the VI phenomena.
Already these have been used to investigate topics such as the role of soil moisture in VI transport\cite{shen_influence_2013}, how different foundation features affect contaminant entry\cite{yao_simulating_2013}, or how wind affects building air exchange with the outdoor and pressurization\cite{shirazi_three-dimensional_2017}.\par

In this work, we will likewise use numerical models, in combination with analysis of field-data from well-studied VI sites to explore the nature of contaminant transport.
Some specific topics that will be covered include analysis of how preferential pathways can fundamentally change contaminant transport, and how the lessons learnt can be used to aid in resolving some of the issues in VI investigations, and broaden our understanding of the temporal and spatial variability at VI sites.
The role that sorption of contaminant vapors on soils and in the indoor environment, a largely unstudied phenomena, will also be explored, and some of the implications discussed.\par
