\begin{comment}
Purpose is to demonstrate the issues that temporal and spatial variability causes in VI site investigations, and impress upon the reader the necessity to solve these.

- State the general issues, i.e. harder to assess real human exposure, more expensive, etc.
- Give examples in the literature and describe the situation. (1-2 examples for spatial/temporal respectively should suffice.)

Topics to cover:

Issues that confound VI investigations:
* Indoor sources/false positives (problem with only indoor samples)
* Sub-surface contaminant formation by indoor (sub-surface samples)
* Groundwater samples (spatial variability & presence of contaminants here does not mean entry occurs)
* Issues with preferential pathways
  * Indie style (long distance transport of contaminants)
  * Kelly style (plumbing fixtures)
  * ASU style (+ Danish study?) (nearby sub surface features that are not readily apparent)
* Seasonal aspect
* Necessity of MLE
  * Expensive and hard
  * Need for more robust methods

\end{comment}

\section{Issues In Vapor Intrusion Investigations}

% Some issues with VI investigations
Determining if vapor intrusion occurs at a building is often difficult.
One might be tempted to believe that collecting an indoor air sample inside the building would be sufficient, i.e. that if the vapor contaminant concentrations is over some threshold in the house, proves that IV occurs, and the absence of contaminant vapors is proof that no VI occurs.
However, this approach is too simplistic and may yield false positives.\par % TODO: Source

% False positives
Many common consumer products contain the same contaminants that is often of concern in VI, and the presence of e.g. a petroleum containing gas tank may be the culprit.
Not all indoor contaminant sources are so obvious though, and many contaminants may be inadvertently introduced, such as by bringing home newly dry-cleaned clothing.
Great care is taken to eliminate such indoor sources, but can be challenging.\par % TODO: Source

A compounding issue related to this (and other topics) is that many of the contaminants can sorb onto/into various materials and subsequently desorb for significant periods of time, potentially extending the influence of indoor source beyond their removal.
This a phenomena, among other related issues with sorption, we will discuss in Chapter (TBD).\par % TODO: Reference chapter later

Since indoor air samples may not sufficiently prove that VI occurs, one usually have to take further steps.
One such is to collect air samples right below the foundation of the building, and if contaminant vapors are found there, as well as in the indoor air, that is more compelling evidence that VI occurs.
However, as work by \citeauthor{holton_creation_2018}\cite{holton_creation_2018} has shown, contaminant vapors in the indoor environment may migrate from inside the building to the subslab, creating a contaminant cloud that may persist for significant periods of time, rendering some extra uncertainty.\par

Collecting samples from the contaminant source, such as the groundwater underneath a building, to determine the presence of contaminants can be used as another potential evidence of VI.
However, even identifying the source alone is not enough evidence as the presence of a contaminant source does not mean that the contaminant vapor enter the overlying building.
\citeauthor{folkes_observed_2009}\cite{folkes_observed_2009} who conducted a decade long study of VI sites in Redfield, Colorado, and a 19 month long study in New York showed that many of the sites, even though they were above a contaminated groundwater source, were not impacted by VI.\par

Often one would take samples from different locations, the types already discussed as well as soil-gas samples, and compare the relative decrease in contaminant vapor concentration from the source to the indoor to determine if entry ultimately occurs from the source.
This subsequent decrease is called \textit{attenuation} and is quantitively determined as an \textit{attenuation factor}, where the indoor contaminant concentration is divided by the contaminant concentration in the soil-gas or groundwater.
$\alpha$ is often used to represent the attenuation factor, and in this work we will often use a subscript to denote attenuation from where.
In the case of groundwater attenuation for instance, one would divide the indoor contaminant concentration by the groundwater contaminant vapor concentration (that is when the contaminant concentration is in equilibrium with air, e.g. Henry's Law).
\begin{equation}
  \alpha_\mathrm{gw} = \frac{c_\mathrm{in}}{c_\mathrm{gw} K_H}
\end{equation}
Where $\alpha_\mathrm{gw}$ is the attenuation from groundwater;
$c_\mathrm{in}$ i sthe indoor contaminant concentration;
$c_\mathrm{gw}$ is the groundwater \textit{liquid} contaminant concentration;
and $K_H$ is the dimensionless Henry's Law constant.\par

Over time, the U.S. Environmental Protection Agency (EPA) has compiled data of attenuation factors relative to different depths and source types.
Using these data, standards as to which attenuation factors are to be expected relative to these metrics have been established to help guide investigators and regulator determine if VI occurs.
This is helpful, but often we are faced with VI sites that render these sort of standards difficult to use.\par

Heterogeneity in soil, distribution of contaminant source concentration, as well as source type (e.g. a contaminant spill in the soil itself or some leaky chemical tank underground), can all contribute to significant spatial variability in contaminant concentration at a site.
A good example of this is a study by \citeauthor{luo_spatial_2009}\cite{luo_spatial_2009} where contaminant concentration beneath a build foundation varied from 200 to less than 0.01 mg/L.
\citeauthor{bekele_influence_2014}\cite{bekele_influence_2014} likewise found that TCE soil-gas concentration could vary an order of magnitude underneath the foundation of their site.\par

Likewise, not all indoor environments are perfectly mixed and indoor contaminant concentration can vary significantly between different rooms or compartments in a building.
This was observed at a site in Boston, Massachusetts by \citeauthor{pennell_sewer_2013}\cite{pennell_sewer_2013} who found that the indoor contaminant concentration was significantly higher in the upstairs bathroom than in the basement, where one typically would expect higher concentrations.\par

The work by \citeauthor{pennell_sewer_2013}\cite{pennell_sewer_2013} further revealed that VI could occur through sewers and enter the building through broken plumbing fixtures, which adds another level of complexity in VI - preferential pathways.
A preferential pathways is a term used describe something allows contaminant vapors to enter near or into a building, in contrast with the more "traditional" view that contaminant transport occurs through soil.
\citeauthor{mchugh_evidence_2017}\cite{mchugh_evidence_2017} studied a VI impacted building in Indianapolis, Indiana, and revealed that the sewage system there acted as a preferential pathway (in addition to the contaminated groundwater).
There contaminated groundwater infiltrated into the sewage system a few blocks away from the site, where originally an old dry cleaner had been located, and was transported to the site.\par

Similarly, \citeauthor{guo_identification_2015}\cite{guo_identification_2015} studied a site in Layton, Utah, that was found to be impacted by a sewer connected land drain that allowed contaminant vapors to be transported to the near-slab region (and close to a visible breach in the foundation).
This study and its preferential pathway is a significant focus of this work and will be discussed in more detail in Chapter (TBD).\par

\citeauthor{holton_temporal_2013}\cite{holton_temporal_2013} (same group as \citeauthor{guo_identification_2015}) demonstrated the very significant temporal variability that exist at some sites, where they found close to four orders of magnitude in variability during the multi-year study period.
\citeauthor{hosangadi_high-frequency_2017}\cite{hosangadi_high-frequency_2017} studied another site in San Diego, California, that likewise showed orders of magnitude temporal variability, albeit on a shorter time scale.
Likewise, the aforementioned site in Indianapolis had significant temporal variability\cite{schumacher_fluctuation_2012}.\par

This temporal variability often has a seasonal component, and the highest indoor contaminant concentration are usually found during the colder months of the year\cite{schumacher_fluctuation_2012,holton_temporal_2013}, although there are cases when the opposite is true. % TODO: See if you can find good source that shows VI is highest during summer
\cite{steck_indoor_2004}\cite{steck_indoor_2004} shows this at a radon impacted building in Minnesota, where the highest radon levels where observed during summer.
Some authors such as \citeauthor{bekele_influence_2014}\cite{bekele_influence_2014} suggesting that it may be necessary to collect samples at intervals across a full year, to account for all seasonal effects, to avoid mischaracterization of VI.\par

The complexity and nuances of VI has made it necessary for a multiple line-of-evidence (MLE) approach to be taken when determining if a building is impacted by VI\cite{u.s._environmental_protection_agency_oswer_2015,pennell_field_2016}.
However, the complexities associated with this necessary approach has prompted the development of new methodologies and techniques that can reduce the uncertainty and complexity associated with conducting VI site investigations.
