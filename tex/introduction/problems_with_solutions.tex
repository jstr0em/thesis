\begin{comment}
Main point is to point the issues with manipulating or utilizing an external variable.
I.e. sites give different responses to the same stimuli.
Again present a few examples of this (maybe just one?), but perhaps quite short as this will be covered in more detail in subsequent chapters.

Present thesis at the end of this section? (Probably)

Thesis:

Manipulation or use of external variables in VI will not yield the desired outcome, unless we improve our physical or mechanistic understanding of how these variables affect contaminant transport (or other phenomena).

Specifically, considering or determining the nature of the contaminant transport, whether advective or diffusive transport dominates, as it is key for understanding why different sites respond so differently to a change in pressurization.

For example, if contaminant transport is dominated diffusion, almost no relationship between entry rate and pressurization will exist, and as such pressurization might be a poor ITS, or CPM might not be as effective at such a site.

On the other hand, if advective transport dominates, pressurization can be a great ITS as advective transport is driven to a large degree by a pressure gradient.
\end{comment}

\section{Issues With Applying CPM \& Using ITS} % TODO: Improve title

On the surface, the CPM and ITS methods are two different approaches that try to address the same problem.
However, they both attempt to utilize some external variable, such as building pressurization, and either manipulate it directly, or by inference, to determine or predict the indoor contaminant concentration.

The issue with these approaches is that they assume that different VI sites will respond comparably to these external variables, which in reality is not the case.
\citeauthor{guo_identification_2015}\cite{guo_identification_2015} used CPM to (inadvertently) identify a preferential pathway at their site, closed it, and noticed a markedly different relationship between indoor contaminant concentration and building pressurization for the period before and after the closing of the preferential pathway.\par % TODO: Add reference to preferential pathway chapter

The reason for this change in behavior will be elaborated on in Chapter (TBD), but consider that contaminant transport from soil into a building across a foundation breach occurs through two means - diffusion and advection. % TODO: Chapter reference
If advective transport dominates, then one would expect a strong correlation between building pressurization and contaminant entry rate, which determines the indoor contaminant concentration; if diffusive transport dominates, this relationship would be absent or weak.\par

Thus, to reliably apply any technique such as CPM or to use ITS, one needs a good mechanistic understanding of how, for instance, contaminant transport occurs at a VI site, and how the various site specific characteristics give rise to different transport phenomena.
Assessing this in the field can be challenging, and therefore we turn to the use of numerical models of VI scenarios, guided by a first-principles perspective, which gives the ability to study physical phenomena in great detail.\par
