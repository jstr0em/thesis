\section{Issues With Applying CPM and Using ITS}

On the surface, the CPM and ITS methods are two different approaches that try to address the same problem.
However, they both attempt to utilize some external variable, such as building pressurization, and either manipulate it directly, or by inference, to determine or predict the indoor contaminant concentration.\par

The issue with these approaches is that they assume that different VI sites will respond comparably to these external variables, which in reality is not the case.
\citeauthor{guo_identification_2015}\cite{guo_identification_2015} used CPM to (inadvertently) identify a preferential pathway at their site, closed it, and noticed a markedly different relationship between indoor contaminant concentration and building pressurization for the period before and after the closing of the preferential pathway.\par

The reason for this change in behavior will be elaborated on in Chapter \ref{chp:preferential_pathways}, but consider that contaminant transport from soil into a building across a foundation breach occurs by two means - diffusion and advection.
If advective transport dominates, then one would expect a strong correlation between building pressurization and contaminant entry rate, which determines the indoor contaminant concentration; if diffusive transport dominates, this relationship would be absent or weak.\par

Thus, to reliably apply any technique such as CPM or to use ITS, one needs a good mechanistic understanding of how, for instance, contaminant transport occurs at a VI site, and how the various site specific characteristics give rise to different transport phenomena.
Assessing this in the field can be challenging, and therefore we turn to the use of numerical models of VI scenarios, guided by a first-principles perspective, which gives the ability to study physical phenomena in great detail.\par
