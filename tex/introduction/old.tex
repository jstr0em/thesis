
\section{Research Motivation}
% 4. Motivate research

Recently significant effort has been spent on studying VI.
Some very notable examples of this are two VI impacted residential homes that were purchased for the sole purpose of conducting highly detailed and long-term studies of VI at these sites.
A key motivator for these studies was to in particular study and understand the temporal variability of VI that has been found at many other sites.\par

The first was a house located near Hill AFB in Utah, which was a co-financed project between Arizona State University (ASU) and the EPA.
The principal investigators at the site was a research team lead by Dr. Paul Johnson from ASU and therefore this site is hereafter referred to as simply the ASU house.
Detailed descriptions of the experimental setup at this house may be found in .% TODO: Add x et al. references here
Suffice to say that this was a highly detailed study where high-frequency indoor air samples of various contaminants were collected in different parts of the house as well as soil-gas and groundwater samples in different locations and depths.
Simultaneously tracer gas studies were continuously conducted to measure the building air exchange rate.
Pressure differences between the indoor and outdoor as well as several meteorological metrics were also collected.\par

The second house was located in Indianapolis, Indiana, and was purchased and invested solely by the EPA.
Similar to the ASU house, this house was outfitted with multiple probes for sampling indoor, groundwater, and soil-gas contaminant concentration at various locations and at high frequency.
Likewise a pressure differences and meteorological metrics were measured, amongst other things.
These sites present a rich and invaluable dataset for understanding VI.\par

% Discussing sewers and preferential pathways
From the onset of the studies at the EPA and ASU houses, it was believed that the VI source was the contaminated groundwater underneath these structures.
Later it was discovered that these sites both were impacted by what is now known as preferential pathways.
A preferential pathway is a term that refers to a process or feature that facilitates the transport of vapor contaminants from a source into the building of concern.
An example of this is subsurface pipe networks such as sewers, land drain, or other piping.\par

At the ASU house, the preferential pathway was a sewer connected land drain that exited underneath the house (with the purpose of draining excess water underneath)\cite{guo_identification_2015}.
For the EPA house, a leaky, house connected sewer pipe was the preferential pathway\cite{mchugh_evidence_2017}.
In both of these cases it seems like the main sewer line had been infiltrated by contaminated groundwater and through the piping, a preferential pathway for the vapor contaminant was established.
Both of these preferential pathways mainly introduced vapor contaminants in the near subsurface underneath, and were subsequently transported into the houses, but other studies show that contaminant vapors may be introduced from sewers through broken plumbing fixtures\cite{pennell_sewer_2013,nielsen_remediation_2017}.

% Need to investigate impact and role of PPs better
The ASU house study demonstrated the significant role preferential pathways can play in VI, and their discovery is an important part of a VI investigation.
But as of yet it is fairly poorly understood how and when a preferential pathway may play a significant role or not.
By contrast to the ASU house, the significance of the preferential pathway at the EPA house is not as clear, and seem to have had quite different effects of the two respective sites.
A lesson learnt from these two sites is that finding these preferential pathways is not always a trivial matter, and research into if and when preferential pathways matter can help by narrowing down the search for them.\par

% Discussing the role of pressure
One significant effect of the preferential pathway at the ASU house was that it greatly enhanced the advective potential at the site, making the house much more sensitive to changes in pressurization\cite{strom_factors_2019,guo_identification_2015,holton_temporal_2013}.
This same strong association between the VI potential and building pressurization was not observed at the EPA house however, even with a preferential pathway present (again showing that their effect can vary significantly).
Another site in San Diego, California, however, again showed very significant correlation between VI potential and building pressurization\cite{hosangadi_high-frequency_2017}.\par

These varied associations between VI potential and building pressurization are not very well understood.
Typically the building pressurization fluctuate across the span of a day and often exhibit seasonal trends - making pressurization a prime candidate in driving much of the temporal variability of VI.
Therefore, an examination of which site characteristic less or increase a building's sensitivity to pressurization is necessary to bridge this gap.\par

% Seasonal trends
Building pressurization is not the only factor that exhibit seasonal trends; temperature, air exchange rate, rainfall, snow coverage, and groundwater depth all vary across time and often (depending on a site's climate) exhibit seasonal trends.
Understanding these trends, and how they impact VI are crucial for collecting representative and dependable samples at a site.
E.g. should one immediately collect indoor air samples after rain? Or is it more prudent to wait a few days? Or weeks? \par

% Adsorption
Many VI contaminants have the potential to sorb onto various common indoor materials and soils.
However, so far it is relatively unknown how sorption processes affects VI as a whole.

% Use of CPM to deal with variability. How do PPs and adsorption in particular play into its applications?
% - How much do we need to pressurize? How long? What affects these choices?
% - Role of sorption in indoor and soil

\section{Research Objective}
% 5. Define research objective

This research's broad aim is to improve our understanding of the complex fate and transport of VI contaminants, with a particular focus on the dynamic processes that drive the temporal variability in VI.
The ultimate goal is to reduce the uncertainty in VI investigations, and to make these easier and cheaper to conduct.
To achieve this we mathematically describe the processes that govern the fate and transport of VI from a first principles approach.
These are implemented and numerically solved using a finite-element method (FEM) solver package - COMSOL Multiphysics, which generate deterministic models of VI scenarios.
By combining these models with statistical analysis of high-resolution datasets from the two well-studied VI sites, the ASU house and the EPA house, we gain an opportunity to explore the dynamic VI processes.\par

These models and analysis are applied to understand the role that pressure fluctuations have on determining temporal variability, and how the preferential pathway discovered at the ASU house enhanced this influence.
The influence the preferential pathway had at the EPA house is likewise investigated.
Seasonal driving forces, and in particular how temperature and wind affect building pressurization and air exchange rate, and how these can be used to explain much of the long-term seasonal changes in VI potential can be explained by these.
We also explore the role that contaminant sorption, both onto soil and various indoor materials, have on the transient response to pressure changes.
Lastly, these findings are explored in relation to the application of the controlled pressure method (CPM), and the significance of their impact on effective use of CPM.\par
