\section{Outline}

Numerical models of VI scenarios will be used throughout this work, and understanding the underlying mathematics that governs VI, as well as how these models are implemented, are crucial for understanding this work.
Chapter \ref{chp:modeling} covers these aspects of VI modeling, where we develop a model of a hypothetical VI scenario.
Here, the governing equations will be introduced, as well as the finite element method (FEM), which will be used to solve our model.
In this process, we will cover the work of constructing a model geometry mesh, configuring solvers, post-processing results, and a variety of practical considerations when modeling VI.
Lastly, a brief summary of VI models and recent developments will be addressed.\par

With this knowledge of VI modeling, we will first use them to tackle the issues of preferential pathways in Chapter \ref{chp:preferential_pathways}.
A significant focus is  placed on modeling and analyzing the preferential pathway found at the "ASU house" VI site in Layton, Utah.
This work will demonstrate how and why preferential pathways can contribute so greatly to temporal and spatial variability in VI.
The preferential pathway at "EPA duplex" VI site in Indianapolis, Indiana, will also be explored here.\par

From the work in Chapter \ref{chp:preferential_pathways}, we find that it is important to consider if vapor contaminant transport from the subsurface into a building is dominated by advective or diffusive transport, a topic that will be further explored in Chapter \ref{chp:transport_implications}.
Here we discuss some of the site specific conditions that give rise to either of these transport mechanisms to dominate.
These conclusions have wider implications for CPM or using ITS, and the efficacy of these investigative methods can hinge on characterizing the dominant transport mechanism at a site.
We also explore how weather conditions and outdoor temperature can be used to predict building pressurization, which becomes an important potential ITS for advection dominated sites.\par

In Chapter \ref{chp:sorption} the role that sorption of vapor contaminant in the indoor environment and onto soil particles is explored.
The capacity of a variety of common materials to sorb TCE are measured at relevant conditions, where we find that these capacities can vary by orders of magnitude.
These sorption data are then applied to a VI model, where the potential influence of these sorption effects on contaminant transport, VI investigations, and application of mitigation systems.\par

Lastly, Chapter \ref{chp:future_work} provides a summary of the conclusions and findings in this thesis, and suggestions future work.\par
