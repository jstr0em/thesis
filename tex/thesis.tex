\documentclass[12pt,twoside]{book}
% structural packages
\usepackage[utf8]{inputenc}
%\usepackage[a4paper,width=150mm,top=25mm,bottom=25mm]{geometry}
\usepackage[letterpaper,margin=1in]{geometry}

\usepackage[hidelinks]{hyperref}
\usepackage[autostyle, english = american]{csquotes}
\MakeOuterQuote{"}

% packages for nicer/more complex floats
\usepackage{caption}
\usepackage{subcaption}
\usepackage{multirow}
\usepackage{booktabs}
\usepackage{enumitem}
\usepackage{tabularx}
\usepackage{import}
\usepackage{comment}
\usepackage{nth}
\usepackage[toc,page]{appendix}
\usepackage{setspace}
\doublespacing


% graphics/figures
\usepackage{graphicx}
\usepackage{rotating}
\usepackage{textcomp}

% reference package
\usepackage[style=numeric,sorting=none]{biblatex}
\addbibresource{references.bib}
%\usepackage[sectionbib]{natbib}
%\usepackage{chapterbib}


% subfiles package
\usepackage{subfiles}

% math packages
\usepackage{amsmath}
\usepackage{siunitx}

\begin{document}

\title{
{Understanding The Dynamics Of Vapor Intrusion Processes}\\ % Ideally max 7-8 words (look to abstract for key words and concepts). Think what someone might google if they're interested in my work.
{\large Brown University}\\
}
\author{Jonathan G. V. Ström}
\date{Spring 2020}

\maketitle

\begin{comment}

\chapter*{Dedication}
To mum and dad

\chapter*{Declaration}
I declare that..

\chapter*{Acknowledgements}
I want to thank...
\end{comment}
% frontmatter


%\import{./}{abstract}



\tableofcontents{}

% main body
\subfile{introduction/main}
\subfile{methods/main}
\subfile{preferential_pathways/main}
%\subfile{transport_classification/main}
%\subfile{sorption/main}
%\subfile{conclusions/main}

\printbibliography
% endmatter
\appendix

\begin{comment}
Outline of the paper:

Introduction:
- Background/perspective on indoor air quality and VI

- Temporal & spatial variability in VI

- Overcoming variability
-- Techniques/methods; CPM, ITS, etc.
-- Problems with this approach (buildings respond differently to same stimuli)
-- Need better mechanistic understanding o VI to effectively use these (common thread is they seek to use or manipulate external variable for certain outcome.)

- Modeling for understanding VI
-- Background/past applications
-- Brief description of what they can model?
-- Allow us to understand physical mechanism

- Sorption & VI
-- Presented as another issue with unknown consequences
-- Combined model & experimental effort

- Outline
-- Basically this but in words!

Methods:
- Introduction
-- More detailed history of modeling
--- Prominent models/developments
-- Limitations of many of these models (including ours)

- Model methodology
-- CSM description & how to model this
-- Soil moisture
-- Darcy's law
-- Contaminant transport
-- Indoor air modeling

- FEM
-- Brief description
-- Why we pick it/benefits/advantages
-- Limitations
-- COMSOL
-- Considerations
--- Geometry generation
--- Meshing
--- Solver settings (?)


Preferential pathways:

Classifying Transport & ITS:
- Predicting ITS

Sorption:

Conclusions/Future work:

\end{comment}

\end{document}
