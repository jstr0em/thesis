\section{Introduction}\label{sec:chp2_intro}

% some more general history here (move this to intro chapter?)
Vapor intrusion (VI) models were first applied at the well-known Love Canal Superfund site during the 1970s, as part of the risk assessment of the site\cite{noauthor_history_2019}.
While VI had become of concern at Love Canal, indoor air contaminant concentrations were considered too dilute, and with ambiguity regarding exposure standards (residential vs. occupational), VI was not considered a major concern in general at the time\cite{little_measuring_2017}, so little further development occurred at this time.\par

% short broad background on VI modeling (from pesticide and Radon modeling)
The history of VI modeling can be traced back to the development of pesticide transport models.
While pesticide and VI exposure are quite different, these early models lay the foundation for modeling contaminant transport in soils and groundwater, which later would become an integral part of VI modeling; entry into buildings was not part of these models.
With the advent of concern for Radon intrusion in the 1970s and -80s, models were developed to predicting the entry of Radon into buildings, which lay the foundations for future VI models.\par

% Radon modeling work and in particular Nazaroff's contributions
One of the most significant contributors to the development of Radon intrusion models was the work by William W. Nazaroff during the later half of the 1980s.
