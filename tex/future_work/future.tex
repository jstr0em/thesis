\section{Suggestions for Future Research}

\subsection{Advective Transport and Specific Site Characteristics}

It has been shown that most soils are not permeable enough to provide enough airflow for advective transport of contaminant vapor from the subsurface into building to dominate, and such conditions are most likely to arise as a result of some site specific characteristic, such as existence of a preferential pathway.
However, more of these site specific characteristics needs to be explored to gain a more holistic view of which one's are likely to be important during a VI site investigation.
Some examples of cases in which significant additional air flow may be possible include those in which there are:
\begin{itemize}
  \item Gravel backfills around a building
  \item French drains (or similar)
  \item Disturbed or unpacked soil around the building
  \item Air pulled through a permeable layer that connects two adjacent buildings, i.e. can one building use another as a preferential air source?
\end{itemize}\par

\subsection{Preferential Pathways}

More types of sewer connected preferential pathways should be considered.
For instance, at the EPA duplex, it is likely that the sewer line there leaked a few meters away from the edge of the duplex, and existence of a leaky preferential pathway should be considered.\par

\subsection{Sorption and Vapor Intrusion}

Sorption is a relatively unexplored phenomena in VI, but has been shown to potentially have significant consequences, in particular with regards to mitigation of VI at a building.
More work is needed to collect sorptive capacities of more materials, considering a greater variety of VI contaminant (at relevant vapor concentrations).\par

\subsection{Modeling and Design of Mitigation Systems}

Mitigating VI at a site is obviously an important task, but it is not always clear what type of mitigation system design is most appropriate for a given site.
Modeling may here offer insights on optimizing a mitigation system design.
A mitigation system was installed at the EPA duplex during the latter of that study, and its rich dataset offers an excellent opportunity to examine the efficacy of various designs using modeling.\par

\subsection{Model Effects of Weather and Seasons on Vapor Intrusion}

In this work, we considered how temperature and wind pressurizes a building relative to ambient, which help explain some of the seasonal trends observed at some VI sites.
This work should be expanded to consider other weather phenomena, such as rainfall, snow coverage, or other, to gain a more holistic view of how VI and weather are related.\par
