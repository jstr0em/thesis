\section{Summary of Results}

Throughout this thesis, actual field investigation results have been presented from vapor intrusion impacted buildings.
The results shown here were among the most carefully collected and analyzed results in the vapor intrusion field.
And yet, those results still left those who collected them puzzling about what they meant, and how they could be generalized to other VI sites where the expenditure of such detailed characterization efforts is out of the question.
This is what led to the conclusion that application of advanced numerical modeling is necessary, in order to begin to make sense out of what otherwise looks difficult to explain or even chaotic.
This was the starting point for this effort.\par

As summarized in this thesis, this is by no means the first effort at numerical modeling of vapor intrusion.
But the reality in the field has been that most numerical models have been designed to study the steady state, which as shown with the field data, is almost invariably not the case.
So we have built in the capacity to model transient VI phenomena, something that has not been emphasized before.
But beyond that, we have been able to use the numerical modeling to begin to develop a better sense of what parameters are key to determining the results seen at different sites.
It was only through such modeling that the concepts of "advective" vs. "diffusive" entry were identified, and begin to suggest how to categorize sites which will respond very differently to the different kinds of field tests that are being performed.\par

In this work a framework for creating numerical models of vapor intrusion (VI) scenarios has been presented.
These models are able to simulate contaminant transport from a source into a building through soils, while taking site specific characteristics, soil moisture, and heterogenous soils into account.
Sorption effects, biodegradation, and other phenomena, at both steady state and in time dependent simulations can also be taken into consideration.\par

These models have been used to explore the temporal and spatial variability that has been observed at VI sites, and in particular the variability associated with preferential pathways like that at the ASU house site.
It was demonstrated that preferential pathways may contribute significantly to temporal variation of indoor contaminant concentration, and VI in general, by greatly enhancing the role of advective contaminant transport into a building.
This in turn required that three conditions to be satisfied:
\begin{enumerate}
  \item A ready source of air must be supplied.
  \item Likewise, a preferential source of contaminant vapors must be supplied.
  \item A permeable zone, e.g. a gravel layer, between the preferential pathway and indoors must exist to facilitate transport.
\end{enumerate}\par

With the revelation that significant temporal variability of indoor contaminant concentration being associated with advective transport of vapors into a building, we explored the potential for various soils to support sufficient airflow rates such that advection dominate entry.
Twelve different soil types were considered, and compared for a house featuring a basement, and on that had a slab-on-grade type foundation.
Regardless of foundation type, only sandy type soils can be expected to be permeable enough to support airflow for advective transport to dominate.
Thus, various site specific characteristics, such as preferential pathways (but not necessarily limited to these) are needed for the elevated airflow rates required for advection to dominate.\par

It is important to consider if advective or diffusive transport dominates at a site, as the association between building pressurization and contaminant entry are very different for each.
For sites characterized by advection, this association is likely to be strong, and it is weak at diffusion dominated sites.
Consequently, the application of VI investigation techniques like the controlled pressure method are likely to only be effective at advection dominated sites.\par

Another consequence is that for advection dominated sites, building pressurization can be used as an effective metric for determining when indoor contaminant concentrations are likely to be the highest.
We also showed that building pressurization in turn can be predicted relatively easily based on indoor/outdoor temperature differences and wind effects.
As the indoor/outdoor temperature differences increases, i.e. it is warmer inside than outside, a building is increasingly depressurized; which is likely why for many sites indoor contaminant concentration are higher during winter.\par

The trichloroethylene (TCE) sorption capacity of a variety of common materials was measured at relevant contaminant concentration, showing that some of these materials can contain significant amounts of TCE; cinderblock was able hold up to almost 41,000 times more contaminant than a comparable TCE contaminated air volume.
These sorptive data were then used to explore the role of sorption in some modeled VI scenarios.
The modeling showed the significant retarding effect on contaminant transport, due to the increased residence time in the soil pores, that soil sorption can have.\par

It was also shown that significant amounts of contaminants can be sorbed in the indoor environment, and in some cases maintain a pseudo steady-state, where contaminant vapors are sorbed or desorbed with changing indoor contaminant concentrations.
In a situation where a VI site has been effectively mitigated, i.e. contaminant entry into the building completely stopped, the contaminant desorption from indoor materials can maintain significant indoor contaminant concentrations even weeks after the mitigation system has been implemented.\par

The work herein presented shows the value of this type of advanced numerical modeling in a field that is otherwise characterized by field studies of VI sites.
The nature and heterogeneity of these sites renders it difficult to control for any given condition or variable, which is a hindrance for developing generalizable conclusions.
The type of modeling used here shows that even with relatively simple models based on VI sites, it is possible to capture much of the observed physical behavior.
This allows us to use these same models as an effective complimentary tool for investigating various VI related phenomena, and determining how sensitive VI is to site specific conditions and other variables.\par

Recently, the use of models have somewhat fallen out of favor in the VI community.
One reason for this is that they have simply been inadequate in addressing and explaining the observed VI phenomena at VI sites.
This has been particularly true for characterizing temporal and spatial variability.
However, it is important to remember that VI models used by investigators and regulators are of the analytical variety.
While many of these offer good insights, they are limited as they simply cannot be modified beyond their underlying conceptual site model (CSM) and assumptions.
For instance, the popular Johnson and Ettinger model cannot capture the behavior of a land drain, such as we could in this work, simply because one is not assumed to exist, and the user can do little to address this.\par

The advanced numerical model used in this thesis, while relatively simple and often borrow many aspects of the CSMs that underpin analytical models, can be extended to include site features or conditions that are impossible for analytical models.
There is nothing technically preventing numerical models from simulating a VI site in great details, and characterizing not only the specific building, but can also include a full accounting of soil heterogeneity, e.g. by including clay layers, larger rock formations, or other subsurface features, which have a significant impact on VI.
Numerical models are necessary if one wishes to model a site in any specific way.\par

Numerical modeling should play a much larger role in VI investigations as they can be used to develop more advanced and realistic CSMs.
For instance, an investigators can go to a site, develop a CSM based on initial observations, construct a model and gain some predictions of how the site may be expected to behave.
If there is some disparity between the model and reality, this may indicate that there is something missing from the CSM, e.g. does the indoor contaminant concentration temporal variability suggest that there is a preferential pathway?
Different cases that aim to explain the disparity can then be constructed as a means to guide the site investigation.
Ultimately, the use of numerical models of VI sites are a great tool that can greatly shorten the time required to determine human VI risk and at the same time reduce the uncertainty of determining this.\par
