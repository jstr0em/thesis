% TODO: Maybe change c_g -> c_{g,ck} ?

\subsection{The Indoor Environment}\label{sec:indoor}

The impacts on the indoor air space is perhaps the most important part of modeling VI, as the goal of these models ultimately is to predict indoor exposure given some external factors.
The indoor environment is, however, only modeled implicitly as a continuously stirred tank reactor (CSTR).\par
We assume that all contaminant entry into the house occurs via the foundation crack.
Once the contaminant enters the interior, it is instantly perfectly mixed, which is a key assumption of a CSTR.
Contaminant expulsion occurs via air exchange with the outdoor environment, and is regulated by \textit{air exchange rate} $A_e$, which dictates what fraction of the indoor air is exchanged with outdoor air over a given period of time.
For instance, a common air exchange rate for a house is \SI{0.5}{\per\hour}, i.e. half of the indoor air is exchanged every hour.
It should be noted that in this simple VI model implementation, we assume that there are no indoor sources nor that any sorption of contaminant to/from any indoor materials occurs.
Thus, the reaction term that would ordinarily be part of a CSTR is dropped (but is reintroduced in Chapter (TBD)) and the change in indoor contaminant concentration is thus given by \eqref{eq:cstr}. % TODO: Add chapter reference
\begin{equation}\label{eq:cstr}
  V_\mathrm{bldg}\frac{\partial c_\mathrm{in}}{\partial t} = n_\mathrm{ck} - V_\mathrm{bldg} A_e c_\mathrm{in}
\end{equation}
Here $c_\mathrm{in}$ [\si{\mol\per\metre\cubed}] is the indoor air contaminant concentration;
$n_\mathrm{ck}$ [\si{\mol\per\second}] is the contaminant entry (or exit) rate into the building via the foundation crack;
$A_e = \SI{0.5}{\per\hour}$ is the air exchange rate;
Finally, $V_\mathrm{bldg} = \SI{300}{\metre\cubed}$ is the volume of the house interior (or basement in this case).\par

A limitation of this approach is that we only consider one control volume or compartment, while in reality indoor contaminant concentrations can vary significantly between compartments, in particular between different floors.
There are VI models that use multiple compartments, which in essence are just coupled CSTRs\cite{murphy_multi-compartment_2011}.
Basements typically have higher indoor contaminant concentrations than other floors, so in this implementation we assume that our sole compartment is the house basement, which $V_\mathrm{bldg} = \SI{300}{\metre\cubed}$ reflects.\par

Solving \eqref{eq:cstr} requires us to determine the contaminant entry and air exchange rates.
Air exchange rates can vary quite significantly, and are a significant source of temporal variability in VI, a topic that will be further explored in Chapter (TBD). % TODO: Reference chapter
However, they typically vary around relatively well-known values as air exchange rates are regulated in building codes.
For residential buildings, it is typical that air exchange rate is around $A_e = \SI{0.5}{\per\hour}$ and thus for simplicity we will choose this value.\par

\paragraph{Contaminant entry into the building}

Contaminant entry rates are significantly more difficult to determine, as they depend on air velocity through the foundation breach and the concentration gradient across it.
The determination of these is the main point, and challenge in VI modeling.\par

% TODO Add figure showing schematic of the entry into the building

The contaminant entry $n_\mathrm{ck}$ is given by integrating the contaminant entry flux $j_\mathrm{ck}$ across the foundation crack boundary $A_\mathrm{ck}$.
\begin{equation}
  n_\mathrm{ck} = \int_{A_\mathrm{ck}} j_\mathrm{ck} dA
\end{equation}
The contaminant flux through the foundation crack is modeled as transport between two parallel plates and has an advective and a diffusive component.
\begin{equation}
  j_\mathrm{ck} = j_\mathrm{advection} + j_\mathrm{diffusion}
\end{equation}
Since contaminant concentration indoors is lower than it is in the soil or near foundation crack region a concentration gradient from the soil-gas to the indoor will exist.
The interior of the crack is not explicitly modeled, but assumed to only contain air and thus we assume the diffusion coefficient is the same as in air.
\begin{equation}
  j_\mathrm{diffusion} = - \frac{D_\mathrm{air}}{L_\mathrm{slab}} (c_{in} - c_g)
\end{equation}
here $D_\mathrm{air} = \SI{7.2e-6}{\metre\squared\per\second}$ is the diffusion coefficient of TCE in air as a sample contaminant of interest; other contaminant of common concern have comparable diffusivities. % TODO: Make sure this is right
$L_\mathrm{slab} = \SI{15}{\centi\metre}$ is the thickness of the foundation slab;
$c_{in}$ [\si{\mol\per\metre\cubed}] is the indoor contaminant concentration;
$c_g$ [\si{\mol\per\metre\cubed}] is the contaminant gas-phase concentration at the foundation crack boundary.\par

Advective transport through the slab can occur in both directions, i.e. contaminants can be carried from the soil into the house and from the house into the soil\cite{holton_creation_2018}.
The direction of this transport depend on the direction of the flow, with a positive sign indicating that airflow goes into the house.
\begin{equation}
  j_{advection} = \begin{cases}
    u_{ck} c_g & u_{ck} \geq 0 \\
    u_{ck} c_{in} & u_{ck} < 0
\end{cases}
\end{equation}
here $u_{ck}$ [\si{\metre\per\second}] is the airflow velocity through the foundation crack.

Thus the total contaminant transport through the foundation crack is given by \eqref{eq:contaminant_entry}.
\begin{equation}\label{eq:contaminant_entry}
  j_{ck} = \begin{cases}
    u_{ck} c_g - \frac{D_\mathrm{air}}{L_\mathrm{slab}} (c_{in} - c_g) & u_{ck} \geq 0 \\
    u_{ck} c_{in} - \frac{D_\mathrm{air}}{L_\mathrm{slab}} (c_{in} - c_g) & u_{ck} < 0
\end{cases}
\end{equation}
Not only will \eqref{eq:contaminant_entry} be used to calculate the contaminant entry rate into house, but it is a necessary boundary condition for calculating the contaminant concentration in the soil.
However, as we see, \eqref{eq:contaminant_entry} is a function of both the soil-gas concentration at the foundation crack boundary $c_g$ and the indoor contaminant concentration $c{in}$, thus these two are coupled and need to be solved simultaneously.\par
