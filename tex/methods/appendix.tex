\clearpage
\begin{appendices}

  \section{Geometry Generation}

  To create our quarter geometry, only a few simple geometric objects and Boolean operations are required: two cuboids, two rectangles, one Boolean difference operation, and one Boolean join operation.
  Figure \ref{fig:geometry} shows the resulting geometry.
  Note that $z = \SI{0}{\metre}$ is the groundwater/soil interface and the plane of symmetry is around the $(x, y) = (\SI{0}{\metre},\SI{0}{\metre})$ axis\par

  To create the soil surrounding the building using the COMSOL geometry generator:
  \begin{enumerate}
    \item Create a \SI{15}{\metre} by \SI{15}{\metre} by \SI{4}{\metre} block with its base at $(x, y, z) = (\SI{0}{\metre},\SI{0}{\metre},\SI{0}{\metre})$. This is the entire soil domain.
    \item Create a \SI{5}{\metre} by \SI{5}{\metre} by \SI{1}{\metre} block with its base at $(x, y, z) = (\SI{0}{\metre},\SI{0}{\metre},\SI{3}{\metre})$. This will represent the volume that the house take up in the soil, i.e. the underground portion of the basement.
    \item Perform a difference operation, removing the "basement" block from the "soil" block.
  \end{enumerate}
  At this point you will see that a quarter soil domain has been created, with an empty space that represents a house with a foundation slab located \SI{1}{\metre} bgs.\par

  The foundation crack will be modeled by joining two  \SI{1}{\centi\metre} wide strip that spans the perimeter of the surface that represents the house foundation.
  This strip is created by joining two rectangles on foundation surface:
  \begin{enumerate}
    \item Define a work plane \SI{3}{\metre} above zero. This allows us to place two-dimensional objects on the surface of or inside a three-dimensional object.
    \item On the work plane create a \SI{5}{\metre} by \SI{1}{\centi\metre} rectangle with its base at $(x, y) = (\SI{0}{\metre},\SI{5}{\metre} - \SI{1}{\centi\metre})$. This represents one side of the perimeter crack.
    \item Copy the rectangle and rotate it \SI{90}{\degree} around the corner of the foundation, i.e. $(x, y) = (\SI{5}{\metre} - \SI{0.5}{\centi\metre},\SI{5}{\metre} - \SI{0.5}{\centi\metre})$.
    \item Join the two rectangles to create a unified perimeter foundation crack.
  \end{enumerate}
  Now the geometry of this VI scenario is complete.\par

  \section{Properties}
  % TODO: Make sure gravel density data is correct
  % TODO: Flip the table?
  \begin{table}
    \centering
    \caption{Properties and van Genuchten parameters of select soil types\cite{abreu_conceptual_2012}.}
    \label{tbl:soils}
  \begin{tabular}{c c c c c c c}
    \toprule
    \multirow{2}{*}{Soil type} & Permeability & Density & Porosity & Residual moisture & \multicolumn{2}{c}{van Genuchten parameters} \\
    & $\kappa \; \mathrm{(m^2)}$ & $\rho \; \mathrm{(kg/m^3)}$ & $\theta_t$ & $\theta_r$ & $\alpha$ & $m$ \\
    \hline
    Sand & \num{9.9e-12} & 1430 & 0.38 & \num{5.3e-2} & 3.5 & 3.2 \\
    Loamy sand  & \num{1.6e-12} & 1430 & 0.39 & \num{4.9e-2} & 3.5 & 1.7 \\
    Sandy loam  & \num{5.9e-13}  & 1460 & 0.39 & \num{3.9e-2} & 2.7 & 1.4 \\
    Sandy clay loam  & \num{2.0e-13} & 1430 & 0.38 & \num{6.3e-2} & 2.1 & 1.3 \\
    Loam  & \num{1.9e-13}& 1380 & 0.40 & \num{6.1e-2} & 1.5 & 1.5 \\
    Silt loam  & \num{2.8e-13} & 1380 & 0.44 & \num{6.5e-2} & 0.51 & 1.7 \\
    Clay loam  & \num{1.3e-13}  & 1500 & 0.44 & \num{7.9e-2} & 1.6 & 1.4 \\
    Silty clay loam & \num{1.7e-13} & 1390 & 0.48 & \num{9.0e-2} & 0.84 & 1.5 \\
    Silty clay  & \num{1.5e-13} & 1300 & 0.48 & \num{1.1e-1} & 1.6 & 1.3 \\
    Silt  & \num{6.7e-13} & 1260 & 0.49 & \num{5.0e-2} & 0.66 & 1.7 \\
    Sandy clay  & \num{1.7e-13} & 1470 & 0.39 & \num{1.2e-1} & 3.3 & 1.2 \\
    Clay  & \num{2.3e-13} & 1330 & 0.46 & \num{9.8e-2} & 1.3 & 1.3 \\
    Gravel\cite{dan_capillary_2012} & \num{1.3e-9} & 1430 & 0.42 & \num{5.0e-3} & 100 & 2.19 \\
    \bottomrule
  \end{tabular}
  \end{table}
\end{appendices}
