\subsection{Mass Transport In The Vadose Zone}\label{sec:transport}

To begin deriving a governing equation for contaminant transport in our VI scenario, we consider the continuity equation which states that the change of concentration in some volume of space depends on the advective and diffusive fluxes in and out of the system, as well as any generation or consumption inside the system.
\begin{equation}
  \frac{\partial c}{\partial t} + \nabla \cdot(j_\mathrm{adv} + j_\mathrm{diff}) - G = 0
\end{equation}
here $c$ [\si{\mol\per\metre\cubed}] is the concentration of the chemical species;
$t$ [\si{\second}] is time;
$j_\mathrm{adv}$ and $j_\mathrm{diff}$ [\si{\mol\per\second\per\metre\squared}] are the advective and diffusive fluxes respectively;
and $G$ [\si{\mol\per\second}] is the generation or consumption of the chemical species.\par

In our model we will assume that $G = 0$ as the groundwater is the sole contaminant source and TCE does not readily degrade in soils. % TODO: Source for TCE degradation
However, this term should remain and an appropriate expression developed if one wants to model:
\begin{itemize}
  \item Biodegradation of some compound in the soil.
  \item Radon intrusion (remember radon gas is generated in soils and rocks).
  \item A soil or subsurface source, e.g. a leaky tank or evaporation from a (pure) contaminant spill.
\end{itemize}\par

The advective flux is given by
\begin{equation}
  j_\mathrm{adv} = \vec{u} c
\end{equation}
where $\vec{u}$ [\si{\metre\per\second}] is a velocity vector.
The diffusive flux is given by Fick's Law
\begin{equation}
  j_\mathrm{diff} = -D \nabla c
\end{equation}
where $D$ [\si{\metre\squared\per\second}] is the diffusion coefficient of the contaminant in the solute;
and $\nabla c$ [\si{\mol\per\metre\cubed\per\metre}] is a concentration gradient.
Thus we get the advection-diffusion equation which generally governs transport of a chemical species
\begin{equation}
  \frac{\partial c}{\partial t} + \nabla \cdot(\vec{u} c + -D \nabla c) = 0
\end{equation}
However, this will not accurately represent contaminant transport in the vadose zone due to
\begin{itemize}
  \item Contaminant transport occurs inside a variably saturated porous matrix, significantly affecting transport properties.
  \item The contaminant concentration in the vadose zone will be distributed between three phases - gas, water, and solid (via sorption).
\end{itemize}\par

The total contaminant concentration in the soil will be used in lieu of just concentration, i.e. $c \rightarrow c_T$ and thus the total contaminant concentration is the sum of the gas, water, and solid phase concentrations.
\begin{equation}
  c_T = \theta_w c_w + \theta_g c_g + c_s \rho_b
\end{equation}
Here $\theta_g$ and $\theta_w$ are the gas-filled and water-filled porosities respectively;
$c_w$ and $c_g$ [\si{\mol\per\metre\cubed}] are the contaminant concentrations in water and gas respectively;
$c_s$ [\si{\mol\per\kilogram}] is the solid phase or sorbed concentration per mass of soil;
and $\rho_b = (1-\theta_t) \rho$ [\si{\kilogram\per\metre\cubed}] is the bulk density of the soil, which can be calculated from the soil porosity $\theta_t$ and solid phase density of the soil $\rho$ [\si{\kilogram\per\metre\cubed}].\par

The attentive reader will now notice that our governing equation depend on three variables instead of one.
However, remember that we're concerned with low contaminant concentrations, we can relate the gas and liquid phase concentrations via Henry's Law \eqref{eq:henrys_law}
\begin{equation}\label{eq:henrys_law}
  c_g = K_H c_w
\end{equation}
where $K_H = 0.402$ is the dimensionless Henry's Law constant for TCE at \SI{20}{\degreeCelsius}. % TODO: TCE K_H source
We also assume that there are no temperature gradients throughout the vadose zone.\par

The solid phase concentration can be related to the others via a linear sorption isotherm.
Here either the gas-solid or water-solid sorption interaction can be chosen; the former is used in Chapter (TBD) to we will explore effect of gas-solid sorption. % TODO: Chapter reference
\begin{equation}
  c_s = \begin{cases}
    K_p c_w & \text{Water-solid sorption} \\
    K_p c_g = K_p K_H c_w & \text{Gas-solid sorption}
\end{cases}
\end{equation}
here $K_p$ [\si{\metre\cubed\per\kilogram}] is a sorption partitioning coefficient.\par

Another approach is to simply ignore the role of sorption completely, i.e. $K_p = 0$, which has historically been done in VI modeling and is done in this example too.
The reason for this is two-fold.
\begin{enumerate}
  \item Relevant sorption data has not been available.
  \item With an infinite source assumption, and at steady-state, sorption doesn't affect the solution; these have been common assumptions in most VI models so far.
\end{enumerate}
Regardless, we will continue with the sorption $K_p$ term, because this will become relevant in Chapter (TBD) where experimentally derived relevant sorption data is available.\par % TODO: Chapter reference

With Henry's Law and the linear sorption assumption we can relate the total contaminant concentration in the soil matrix to the water-phase contaminant concentration.
\begin{equation}
  c_T = (\theta_w + \theta_g K_H + K_H K_p \rho_b) c_w
\end{equation}
The terms in front of $c_w$ are collected as $R = (\theta_w + \theta_g K_H + K_H K_p \rho_b)$.
This term is called the \textit{retardation factor} and reflects the increased contaminant residence time in the matrix due to shifting between the phases, and becomes an important factor in transient transport simulations.\par

In the vadose zone, advective transport can occur in both the water and gas phases inside the soil pores.
\begin{equation}
  j_\mathrm{adv} = \vec{u}_{w,pore} c_w \theta_w + \vec{u}_{g,pore} c_g \theta_g
\end{equation}
here $\vec{u}_{w,pore}$ and $\vec{u}_{g,pore}$ [\si{\metre\per\second}] are the water and gas phase \textit{pore} velocities vectors respectively.
However, from mass conservation, we know that the product of the pore velocity and porosity gives the superficial velocity of a fluid in porous media, i.e. the Darcy's Law velocities.
This together with Henry's Law gives
\begin{equation}
  j_\mathrm{adv} = (\vec{u}_w + \vec{u}_g K_H ) c_w
\end{equation}
and here $\vec{u}_w$ + $\vec{u}_g$ [\si{\metre\per\second}] are the Darcy or superficial velocity vectors.\par

In section \ref{sec:van_genuchten} we assumed that there is no gravitational water potential in the soil matrix, and it follows that the water in the pores is stationary, i.e. $\vec{u}_w = 0$ giving
\begin{equation}
  j_\mathrm{adv} = \vec{u}_g K_H c_w
\end{equation}
where $\vec{u}_g$ is the solution from solving Darcy's Law in section \ref{sec:darcys_law}.\par

To model a scenario where there is a gravitational water potential, one would have to solve two-phase Darcy's Law to get both $\vec{u}_g$ and $\vec{u}_w$.
This significantly complicates the mass transport aspect as well, and as such is beyond the scope of this work.\par

The diffusive transport expression likewise needs to be adjusted, and the total diffusive flux through the pore matrix is given by
\begin{equation}
  j_\mathrm{diff} = -(D_w \theta_w \tau_w \nabla c_w + D_g \theta_g \tau_g \nabla c_g)
\end{equation}
here $D_w = \SI{1.02e-9}{\metre\squared\per\second}$ and $D_g = \SI{6.87e-6}{\metre\squared\per\second}$ are the contaminant diffusion coefficients of TCE in pure water and gas respectively; % TODO: Add diffusion values here?
and $\tau_w$ and $\tau_g$ are the water and gas tortuosity terms.\par

Due to the irregular shapes of pores, diffusion of a chemical species will inevitably often occur along a tortuous path, which the tortuosity terms attempt to capture.
As tortuosity depend on the structure of the porous matrix, it is difficult to accurately portray, but a popular approach is to use Millington \& Quirks model\cite{millington_permeability_1961}.
\begin{equation}\label{eq:millington-quirk}
  \tau_w = \frac{\theta_w^{\frac{7}{3}}}{\theta_t^2}, \; \tau_g = \frac{\theta_g^{\frac{7}{3}}}{\theta_t^2}
\end{equation}
here $\theta_t$ is the total or saturated porosity of the soil matrix.
Another popular approach is Bruggeman's model, or if possible, a custom version can be used.\par % TODO: Add Bruggeman's reference here

Combining \eqref{eq:millington-quirk} and \eqref{eq:henrys_law} in our diffusion flux expression gives
\begin{equation}
  j_\mathrm{diff} = -\Big(D_w \frac{\theta_w^{\frac{10}{3}}}{\theta_t^2} + D_g \frac{\theta_g^{\frac{10}{3}}}{\theta_t^2} K_H\Big) \nabla c_w
\end{equation}
the terms in front of $\nabla c_w$ can be collected as an effective diffusion coefficient $D_\mathrm{eff}$ [\si{\metre\squared\per\second}], which with our isothermal vadose zone assumption only depends on the soil moisture content.
Thus we get the final diffusive flux expression
\begin{equation}
  j_\mathrm{diff} = - D_\mathrm{eff}\nabla c_w
\end{equation}
Figure \ref{fig:D_eff} shows how the effective diffusivity varies from being close to that of the pure water diffusivity near the capillary zone, and increases to something closer to gas-phase diffusivity as the soil moisture decreases.\par

\begin{figure}
  \includegraphics[width=\textwidth]{effective_diffusivity.pdf}
  \caption{Effective diffusivity of TCE in the vadose zone using Millington-Quirks model. Soil water and gas filled porosites are calculated using van Genuchten's equations.}
  \label{fig:D_eff}
\end{figure}

Putting all this together finally gives us the governing equation for contaminant transport in the vadose zone for our modeled VI scenario.
\begin{equation}\label{eq:mass_transport}
  R \frac{\partial c_w}{\partial t} = \nabla \cdot (D_\mathrm{eff} \nabla c_w) - K_H \vec{u}_g \cdot \nabla c_w
\end{equation}
To solve this we need to define some boundary and initial conditions.\par

% TODO: Rewrite these in the form c(t, vec{x}) = ..., i.e. c(t, atm) = 0, c(t, source) = c_gw etc

\paragraph{Boundary Conditions}

In this VI scenario, the sole contaminant source is assumed to be the homogenously contaminated groundwater, which we assume to have a fixed concentration.
The atmosphere acts as a contaminant sink and thus this is simply a zero concentration boundary condition.
Contaminants leave the soil domain and enter the building through a combination of advective and diffusive gas phase transport.
The boundary condition applied to all other boundaries is a no-flow boundary.
\begin{align}
  &\text{Atmosphere} & c_w = \SI{0}{\mol\per\metre\cubed} \\
  &\text{Groundwater} & c_w = c_{gw} = \SI{0.1}{\mol\per\metre\cubed} \\
  &\text{Foundation crack} & -\vec{n} \cdot \vec{N} = \frac{-j_{ck}}{K_H} \; \si{\mol\per\metre\squared\per\second}\\
  &\text{All other} & -\vec{n} \cdot \vec{N} = \SI{0}{\mol\per\metre\squared\per\second}
\end{align}
$\vec{n} \cdot \vec{N}$ is the dot product between the boundary normal vector and the contaminant flux;
$j_ck$ is the contaminant vapor flux into the building.
We assume that only contaminants in the gas phase enter the building, and dividing $j_{ck}$ by $K_H$ we get proper accounting in terms of the water phase concentration accounting in the main transport equation \ref{eq:mass_transport}.\par

\paragraph{Initial Conditions}

For steady-state problems, the initial conditions do not influence the solution.
Transient simulations however, require initial conditions and these are assumed to be given by the steady-state solution.\par

\paragraph{Basis function}

A second order polynomial (quadratic) function is used as the basis function for solving the transport equation.\par
