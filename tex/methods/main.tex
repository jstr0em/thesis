\documentclass[../thesis.tex]{subfiles}

% graphics path
\graphicspath{
  {../../figures/chapter3/},
  {./../figures/chapter3/}
}

\begin{document}
\chapter{Developing Vapor Intrusion Models}

\begin{abstract}
\chaptermark{abstract}
\newpage
\textit{Abstract of Understanding the Dynamics of Vapor Intrusion Processes, by Jonathan G. V. Ström, Ph.D., Brown University, May 2020.}

Vapor intrusion (VI) investigations, the effort to determine the exposure and associated human health-risk at a VI impacted building, are often complicated by significant spatial and temporal variability in concentrations of contaminants of concern.
Over the years there have been efforts to develop new techniques and methodologies that aim to reduce the uncertainties associated with these variabilities.
The goal is to simplify and improve the robustness of VI site investigations.
The development of the controlled pressure method (CPM), where the pressurization of a building is controlled in an effort to increase or decrease contaminant entry into the building, is one such example.
Another approach is to use indicators, tracers, and surrogates (ITS) to help guide when to conduct site investigations, ideally increasing the likelihood of determining the maximum indoor contaminant concentrations.\par

Both of these approaches rely on a quasi-deterministic relationship between some external variable, such as building pressurization, and indoor contaminant concentration.
However, site-specific conditions can give rise to very different responses to such an external variable.
To effectively use CPM or ITS, a better mechanistic understanding of contaminant transport and exposure is needed.\par

In this thesis, we develop three-dimensional finite element models of VI impacted buildings from a first principles perspective.
These models combined with analysis of field data from VI sites, allows us to explore the physical mechanisms that drive VI.
By considering the dominant contaminant transport mechanism at a site, e.g. if advective or diffusive transport dominates, we can explain why a change in building pressurization can lead to differences in contaminant concentration variability at different sites.
We can also better understand how the various factors governing VI contribute to the overall variability.\par

By classifying the dominant contaminant transport mechanism at a site, we can more effectively anticipate how a particular site will respond to some external stimuli.
This will in turn reduce the effort required to, and increase the robustness of the techniques used determine the relevant human exposure at a VI site.\par

We also applied the model to investigate the role of contaminant sorption to and from soils and common materials.
Sorptive capacities of these materials were determined experimentally at relevant conditions, and found that some materials, such as cinderblock, can hold up to 41,000 times more contaminant than a comparable contaminated air volume.
Sorption and desorption of contaminant can significantly delay changes in contaminant concentration with respect to time, both in the soil-gas and in the indoor environment.
This phenomena is particularly relevant after successful implementation of VI mitigation scheme, where contaminant desorption from certain materials may maintain indoor contaminant concentrations for months longer than if there were no sorbed contaminants.\par

\vfill

\end{abstract}

% TODO: Restructure to reflect the COMSOL workflow
% - Keep main body simple
% - Use appendices as building/lego blocks to add complexity, e.g. one appendix about how to model a PP, and just that.

No models are true representations of reality, but some of them may be useful.
Ever since Newton first wrote his laws of motion, mankind has tried to describe reality with an ever increasing number of mathematical statements.
With the advent of computation and advancements in numerical methods our capabilities to mathematically describe physical systems has dramatically increased.
Even so, real-world systems are too complex to be fully modeled, but mathematical representations may be used to approximate and reveal useful insights of how they function.\par

This is especially true for vapor intrusion (VI) models.
Often it is impossible or difficult to conduct controlled studies of VI sites making models an important tool for understanding these sites and the VI phenomena.
The previous chapter is proof of this as it is readily apparent that a multitude of VI models of varying complexity have been developed over the years, and has become an integral part of the scientific VI community.
From the simple Johnson \& Ettinger one-dimensional model to full three-dimensional finite element models we see that the increased complexity of the model allowed for a greater number of VI topics and phenomena to be explored.\par

The development of the three-dimensional finite element models begin with a conceptual site model (CSM) of a VI site.
In general when one develops models, it is best in the beginning to keep the model as simple as possible, and not to add overly complex features or excessive physics.


% TODO: Integrate or remove paragraphs below
To model vapor intrusion (VI), a number of partial differential equations (PDEs) which describe the relevant physics must be solved.
Many of these PDEs are also coupled in implicit and explicit ways, e.g. the PDE describing vapor flow in the soil affect the contaminant concentration, which in turn affect the indoor air concentration, which in turn also affects the contaminant concentration in the soil.

The indoor air space is perhaps the most important part of modeling VI, as the goal of these models ultimately is to predict indoor exposure given external factors.
One could therefore assume that most of the effort in modeling VI should be spent to accurately represent the interior.
This would be very impractical however, as building interiors are so diverse.
Even if one would spend the time to model an interior, this would dramatically increase the number of mesh elements required to solve the model.
Additionally, the air flow inside the building must be calculated, and even using a simplified version of Navier-Stokes, like Reynolds Averaged Navier-Stokes, the computational cost would be significant.
For these reasons, the indoor air space is simply modeled as a continuously stirred tank (CST), and paradoxically becomes the simplest component of the VI model.\par

Most of the effort will be spent to accurately model the physics in the soil underlying the building of interest.

\chapter{Geometry Generation}

To create our quarter geometry, only a few simple geometric objects and Boolean operations are required: two cuboids, two rectangles, one Boolean difference operation, and one Boolean join operation.
Figure \ref{fig:geometry} shows the resulting geometry.
Note that $z = \SI{0}{\metre}$ is the groundwater/soil interface and the plane of symmetry is around the $(x, y) = (\SI{0}{\metre},\SI{0}{\metre})$ axis\par

To create the soil surrounding the building using the COMSOL geometry generator:
\begin{enumerate}
  \item Create a \SI{15}{\metre} by \SI{15}{\metre} by \SI{4}{\metre} block with its base at $(x, y, z) = (\SI{0}{\metre},\SI{0}{\metre},\SI{0}{\metre})$. This is the entire soil domain.
  \item Create a \SI{5}{\metre} by \SI{5}{\metre} by \SI{1}{\metre} block with its base at $(x, y, z) = (\SI{0}{\metre},\SI{0}{\metre},\SI{3}{\metre})$. This will represent the volume that the house take up in the soil, i.e. the underground portion of the basement.
  \item Perform a difference operation, removing the "basement" block from the "soil" block.
\end{enumerate}
At this point you will see that a quarter soil domain has been created, with an empty space that represents a house with a foundation slab located \SI{1}{\metre} bgs.\par

The foundation crack will be modeled by joining two  \SI{1}{\centi\metre} wide strip that spans the perimeter of the surface that represents the house foundation.
This strip is created by joining two rectangles on foundation surface:
\begin{enumerate}
  \item Define a work plane \SI{3}{\metre} above zero. This allows us to place two-dimensional objects on the surface of or inside a three-dimensional object.
  \item On the work plane create a \SI{5}{\metre} by \SI{1}{\centi\metre} rectangle with its base at $(x, y) = (\SI{0}{\metre},\SI{5}{\metre} - \SI{1}{\centi\metre})$. This represents one side of the perimeter crack.
  \item Copy the rectangle and rotate it \SI{90}{\degree} around the corner of the foundation, i.e. $(x, y) = (\SI{5}{\metre} - \SI{0.5}{\centi\metre},\SI{5}{\metre} - \SI{0.5}{\centi\metre})$.
  \item Join the two rectangles to create a unified perimeter foundation crack.
\end{enumerate}
Now the geometry of this VI scenario is complete.\par

% TODO: Maybe change c_g -> c_{g,ck} ?

\subsection{The Indoor Environment}\label{sec:indoor}

The impacts on the indoor air space is perhaps the most important part of modeling VI, as the goal of these models ultimately is to predict indoor exposure given some external factors.
The indoor environment is, however, only modeled implicitly as a continuously stirred tank reactor (CSTR).\par
We assume that all contaminant entry into the house occurs via the foundation crack.
Once the contaminant enters the interior, it is instantly perfectly mixed, which is a key assumption of a CSTR.
Contaminant expulsion occurs via air exchange with the outdoor environment, and is regulated by \textit{air exchange rate} $A_e$, which dictates what fraction of the indoor air is exchanged with outdoor air over a given period of time.
For instance, a common air exchange rate for a house is \SI{0.5}{\per\hour}, i.e. half of the indoor air is exchanged every hour.
It should be noted that in this simple VI model implementation, we assume that there are no indoor sources nor that any sorption of contaminant to/from any indoor materials occurs.
Thus, the reaction term that would ordinarily be part of a CSTR is dropped (but is reintroduced in Chapter (TBD)) and the change in indoor contaminant concentration is thus given by \eqref{eq:cstr}. % TODO: Add chapter reference
\begin{equation}\label{eq:cstr}
  V_\mathrm{bldg}\frac{\partial c_\mathrm{in}}{\partial t} = n_\mathrm{ck} - V_\mathrm{bldg} A_e c_\mathrm{in}
\end{equation}
Here $c_\mathrm{in}$ [\si{\mol\per\metre\cubed}] is the indoor air contaminant concentration;
$n_\mathrm{ck}$ [\si{\mol\per\second}] is the contaminant entry (or exit) rate into the building via the foundation crack;
$A_e = \SI{0.5}{\per\hour}$ is the air exchange rate;
Finally, $V_\mathrm{bldg} = \SI{300}{\metre\cubed}$ is the volume of the house interior (or basement in this case).\par

A limitation of this approach is that we only consider one control volume or compartment, while in reality indoor contaminant concentrations can vary significantly between compartments, in particular between different floors.
There are VI models that use multiple compartments, which in essence are just coupled CSTRs\cite{murphy_multi-compartment_2011}.
Basements typically have higher indoor contaminant concentrations than other floors, so in this implementation we assume that our sole compartment is the house basement, which $V_\mathrm{bldg} = \SI{300}{\metre\cubed}$ reflects.\par

Solving \eqref{eq:cstr} requires us to determine the contaminant entry and air exchange rates.
Air exchange rates can vary quite significantly, and are a significant source of temporal variability in VI, a topic that will be further explored in Chapter (TBD). % TODO: Reference chapter
However, they typically vary around relatively well-known values as air exchange rates are regulated in building codes.
For residential buildings, it is typical that air exchange rate is around $A_e = \SI{0.5}{\per\hour}$ and thus for simplicity we will choose this value.\par

\paragraph{Contaminant entry into the building}

Contaminant entry rates are significantly more difficult to determine, as they depend on air velocity through the foundation breach and the concentration gradient across it.
The determination of these is the main point, and challenge in VI modeling.\par

% TODO Add figure showing schematic of the entry into the building

The contaminant entry $n_\mathrm{ck}$ is given by integrating the contaminant entry flux $j_\mathrm{ck}$ across the foundation crack boundary $A_\mathrm{ck}$.
\begin{equation}
  n_\mathrm{ck} = \int_{A_\mathrm{ck}} j_\mathrm{ck} dA
\end{equation}
The contaminant flux through the foundation crack is modeled as transport between two parallel plates and has an advective and a diffusive component.
\begin{equation}
  j_\mathrm{ck} = j_\mathrm{advection} + j_\mathrm{diffusion}
\end{equation}
Since contaminant concentration indoors is lower than it is in the soil or near foundation crack region a concentration gradient from the soil-gas to the indoor will exist.
The interior of the crack is not explicitly modeled, but assumed to only contain air and thus we assume the diffusion coefficient is the same as in air.
\begin{equation}
  j_\mathrm{diffusion} = - \frac{D_\mathrm{air}}{L_\mathrm{slab}} (c_{in} - c_g)
\end{equation}
here $D_\mathrm{air} = \SI{7.2e-6}{\metre\squared\per\second}$ is the diffusion coefficient of TCE in air as a sample contaminant of interest; other contaminant of common concern have comparable diffusivities. % TODO: Make sure this is right
$L_\mathrm{slab} = \SI{15}{\centi\metre}$ is the thickness of the foundation slab;
$c_{in}$ [\si{\mol\per\metre\cubed}] is the indoor contaminant concentration;
$c_g$ [\si{\mol\per\metre\cubed}] is the contaminant gas-phase concentration at the foundation crack boundary.\par

Advective transport through the slab can occur in both directions, i.e. contaminants can be carried from the soil into the house and from the house into the soil\cite{holton_creation_2018}.
The direction of this transport depend on the direction of the flow, with a positive sign indicating that airflow goes into the house.
\begin{equation}
  j_{advection} = \begin{cases}
    u_{ck} c_g & u_{ck} \geq 0 \\
    u_{ck} c_{in} & u_{ck} < 0
\end{cases}
\end{equation}
here $u_{ck}$ [\si{\metre\per\second}] is the airflow velocity through the foundation crack.

Thus the total contaminant transport through the foundation crack is given by \eqref{eq:contaminant_entry}.
\begin{equation}\label{eq:contaminant_entry}
  j_{ck} = \begin{cases}
    u_{ck} c_g - \frac{D_\mathrm{air}}{L_\mathrm{slab}} (c_{in} - c_g) & u_{ck} \geq 0 \\
    u_{ck} c_{in} - \frac{D_\mathrm{air}}{L_\mathrm{slab}} (c_{in} - c_g) & u_{ck} < 0
\end{cases}
\end{equation}
Not only will \eqref{eq:contaminant_entry} be used to calculate the contaminant entry rate into house, but it is a necessary boundary condition for calculating the contaminant concentration in the soil.
However, as we see, \eqref{eq:contaminant_entry} is a function of both the soil-gas concentration at the foundation crack boundary $c_g$ and the indoor contaminant concentration $c{in}$, thus these two are coupled and need to be solved simultaneously.\par

\section{Soil Physics Governing Vapor Intrusion}\label{sec:soil_domain}

The soil surrounding the structure in our VI model is, unlike the indoor environment, modeled explicitly.
In this section we will walk through each physics, the associated governing equation, and boundary conditions required to model the contaminant transport in soil.
The following physics and governing equations will be covered:
\begin{enumerate}
  \item Water flow in unsaturated porous media
  \item Vapor transport in unsaturated porous media
  \item Mass transport in partially saturated medium
\end{enumerate}
In addition to these, the modeling of the temperature distribution in the soil is covered in Appendix .\par % TODO: Reference the appendix once it exists.

The vapor contaminant transport through the soil in VI occurs through the vadose zone - soil that is partially filled with water, giving a three-phase transport system.
This partial water content has profound effects on both advective and diffusive transport and modeling the water content is achieved via \textit{Richard's equation}.
Since the vapor and mass transport in the soil are so dependent on the soil water content, it is covered first in the following section \ref{sec:richards}.\par

The mass transport in the soil has both an advective and diffusive component.
The advective transport in the soil is dictated by the vapor flow in the soil which is described by \textit{Darcy's Law} making it the next logical step to cover in section \ref{sec:darcys}.
The diffusive transport depends on the contaminant vapor concentration itself and accurately modeling this requires coupling with all of the physics discussed so far (including the indoor environment).
Therefore the mass transport physics, governed by the \textit{advection-diffusion equation} will be covered last in section \ref{sec:mass_transport}.\par
% TODO: Fix this so that you don't need to switch paths here. Ideally just reformatting the whole thing.
% TODO: You really need to sit down and wrap your head around all of this and how it's derived.

\subsection{Water Flow in Unsaturated Porous Media}\label{sec:richards}

The vadose zone or unsaturated zone is a region of soil between the top of the ground surface and the water table.
In the vadose zone there are two fluid phases, one gas and the other liquid (usually air and water) inside the porous soil matrix giving a three phase system; only one fluid phase (gas or liquid) exist in the saturated zone.
As a result, the transport properties in the vadose zone differ from that in a zone saturated where there are only two phases present - water and soil.\par



\subsubsection{Soil-Water Potential}

The driving force, or soil-water potential, for the filling and draining of pore water in soils are due to a pressure and a gravitational potential and given by $\phi$.


The driving force for the pore water in the vadose zone is a negative pressure caused by the surface tension of the water.
This phenomena is called \textit{capillary potential} or \textit{matrix potential}, $\psi$, which depends on the volumetric water content $\theta$ in the soil.

\subsubsection{Soil-Water Retention Curve}

The distribution of soil moisture in the soil matrix has profound implications for the advective and diffusive transport of contaminants.
Soil has a limited amount of pore volume available for contaminant transport, and the presence of water restricts this further; decreasing permeability of the soil and subsequently reduces air flow.
Diffusivity of the contaminant will also be retarded by the water.
The contaminant will dissolve into and evaporate from water and the transport will partially occur through water.
Liquid diffusion coefficients are usually around four orders of magnitude smaller than in air.

The soil moisture content of soils can be estimated in many ways, but two common approaches is to use the analytical formulas of \textit{van Genuchten} or \textit{Brooks and Corey}.
Both of these formulas give the soil moisture content as a function of the fluid pressure head, $H_p$.
By definition, when the pressure head is equal to or greater than zero, $H_p \geq 0$, the soil is assumed to be 100\% saturated with the fluid.
In this work, \textit{van Genuchten's} formula is used.

The soil moisture content, $\theta$ is given by.
\begin{equation}
  \theta = \begin{cases}
    \theta_r + \mathrm{Se}(\theta_s - \theta_r) & H_p < 0 \\
    \theta_s & H_p \geq 0
\end{cases}
\end{equation}

The saturation is given by.
\begin{equation}
  \mathrm{Se} = \begin{cases}
    \frac{1}{(1 + |\alpha H_p|^m)^m} & H_P < 0 \\
    1 & H_p \geq 0
  \end{cases}
\end{equation}

\begin{equation}
  C_m = \begin{cases}
    \frac{\alpha m}{1-m}(\theta_s - \theta_r)\mathrm{Se}^{\frac{1}{m}}\big( 1 - \mathrm{Se}^{\frac{1}{m}} \big)^m & H_p < 0 \\
    0 & H_p \geq 0
  \end{cases}
\end{equation}

\begin{equation}
  k_r = \begin{cases}
    \mathrm{Se}^l \big[ 1 - \big( 1 - \mathrm{Se}^\frac{1}{m} \big) \big]^2 & H_p < 0 \\
    0 & H_p \geq 0
  \end{cases}
\end{equation}



\subsubsection{Richard's Equation}

\begin{equation}\label{eq:richards}
  \rho \Big( \frac{C_m}{\rho g} + \mathrm{Se}S \Big) \frac{\partial p}{\partial t} +
  \nabla \cdot \rho \Big( -\frac{\kappa_s}{\mu} k_r (\nabla p + \rho g \nabla D)\Big) =
  Q_m
\end{equation}
Where $p$ is the capillary potential; $C_m$ is the specific moisture capacity; $\mathrm{Se}$ is the effective saturation; $S$ is the storage coefficient; $\kappa_s$ is the saturated permeability of the porous media; $\mu$ is the fluid viscosity; $k_r$ is the effective permeability; $\rho$ is the fluid density; $g$ is the acceleration of gravity; $D$ is the elevation or head; and $Q_m$ is a source term, a positive or negative value represent a source or sink respectively.\par

\subsection{Vapor Transport in Unsaturated Porous Media}\label{sec:darcys}

Fluid transport in porous media is governed by \textit{Darcy's Law} and was originally formulated by Henry Darcy based on his work on describing water flow through soil under the influence of gravity.
Since then it been found to derivable in several ways from the Navier-Stokes equations\cite{bear_dynamics_1972} and may be stated as a pressure gradient driven velocity.
\begin{equation}\label{eq:darcys_law_saturated}
  \vec{u} = -\frac{\kappa}{\mu}\nabla p
\end{equation}
Here $\vec{u}$ is the fluid velocity; $\kappa$ the soil permeability; $\mu$ is the fluid viscosity; and $\nabla p$ is the pressure gradient.
In VI modeling we're interested in the flow of contaminant vapors but since the contaminant concentrations are typically very low, the transport properties may be taken from those of pure air.\par

% TODO: Elaborate on the assumptions, i.e. that in DL the viscosity shear effects may be neglected. But this breaks down when Re is too high.

For Darcy's Law to be valid, two assumptions must be fulfilled:
\begin{enumerate}
  \item The fluid must be in the laminar regime, typically $\mathrm{Re} < 1$.
  \item The soil matrix must be saturated with the fluid.
\end{enumerate}
Typically the vapor flows in most VI scenarios are sufficiently slow for the first condition to be fulfilled.
And if they are not, there are modifications to Darcy's Law that
Most of the contaminant vapor transport takes place in the partially saturated vadose zone and thus, \eqref{eq:darcys_law_saturated} needs modification.\par

In partially saturated soils, a varying portion of the soil pores are available for vapor transport, with the rest being occupied by water, affecting the effective permeability of the soil.
To model this, we use the relative permeability property, $k_r$, from section \ref{sec:richards} is used.
\begin{equation}
  \kappa_\mathrm{eff} = (1-k_r) \kappa_s
\end{equation}
Note that in this Darcy's formulation $(1 - k_r)$ is used to refer to the relative permeability of vapor, e.g. that 0 indicates the soil is completely impermeable for vapor flow (and vice versa).\par
This gives the modified Darcy's Law used in VI-modeling:
\begin{equation}\label{eq:darcys_law}
  \vec{u} = -\frac{(1-k_r) \kappa_s}{\mu}\nabla p
\end{equation}

However, \eqref{eq:darcys_law} only gives the vapor velocity as a function of the pressure gradient, to properly model the vapor flow in the soil matrix we need to incorporate Darcy's Law into a continuity equation giving
\begin{equation}\label{eq:vapor_transport}
  \frac{\partial}{\partial t} (\rho \epsilon) + \nabla \cdot \rho \Big( -\frac{(1-k_r) \kappa_s}{\mu} \nabla p \Big) = Q_m
\end{equation}
which is governing equation for vapor flow in porous media.\par

\paragraph{Boundary conditions}

In order to solve \eqref{eq:vapor_transport} we need to define some boundary conditions.
In our CSM, air is pulled from the atmosphere through the ground surface and into the building via the foundation crack.
To model this only three boundary conditions are required.\par

The first is to define a pressure gauge, i.e. a reference point for where the pressure is zero, which is where air will be pulled from.
This is the applied to the ground surface boundary.
The second is that we apply the indoor/outdoor pressure difference (~5 Pa) to the foundation crack boundary.
The third type is applied to all remaining boundaries and is a no flow boundary condition, indicating that no flow passes through these boundaries.
We also make sure that we specify the symmetry planes present.
\begin{align}
  &\text{Ground surface} &p = 0 \; \mathrm{(Pa)} \\
  &\text{Foundation crack} &p = p_\mathrm{in/out} = -5 \; \mathrm{(Pa)} \\
  &\text{Remaining} &-\vec{n}\cdot\rho\vec{u} = 0
\end{align}
where $\vec{n}$ is the boundary normal vector.\par

\subsection{Mass Transport in Unsaturated Porous Media}\label{sec:mass_transport}

Mass transport of a chemical species occurs primarily through diffusive and advective transport and is typically governed by the advection-diffusion equation (sometimes -reaction is added)
\begin{equation}
  \frac{\partial c_i}{\partial t} + \nabla \cdot (-D \nabla c_i) + \vec{u} \cdot \nabla c_i = R_i
\end{equation}
where $c_i$ is the concentration of the chemical species; $t$ is time; $D$ is the diffusion coefficient; $\vec{u}$ is the bulk fluid velocity vector; and $R_i$ is a reaction term.
As such the first term is the change of concentration in some control volume, the second and third terms are the diffusive and advective fluxes leaving or entering the control volume, and the fourth is whatever change in concentration due to chemical reactions.\par

However, this governing equation is too simplistic to accurately model vapor contaminant transport in the vadose zone.
As has been discussed before, the transport properties vary significantly in unsaturated porous depending on the soil matrix water saturation.
The vadose zone is also a three-phase system, where at any given time some contaminant will be partitioned between the soil, vapor, and liquid phases.
The contaminant will also move between these three phases through volatilization/solvation and sorption.
To accurately model the mass transport of contaminants through the vadose zone, all of these phenomena must be accounted for.\par

% TODO: Rewrite so c_i is the gas-phase instead
\begin{equation}
  \frac{\partial}{\partial t} (\theta c_i) +
  \frac{\partial}{\partial t} (\rho_b c_{P,i}) +
  \frac{\partial}{\partial t} (a_v c_{G,i}) +
  \vec{u} \cdot \nabla c_i =
  \nabla \cdot [(D_{D,i} + D_{eff,i}) \nabla c_i] +
  R_i + S_i
\end{equation}


\section{Meshing}\label{sec:meshing}

A mesh is a collection of small discrete elements that in combination form a larger geometry or domain.
Meshing is the process of generating a mesh.
Meshing is perhaps one of the most important and challenging aspects of solving a FEM model and a well-constructed mesh is necessary for accurate and reliable results.\par

In theory, an infinitely fine mesh will give the analytical solution to a PDE but obviously the computational costs would be infinite then as well; one must always balance the accuracy of the solution and computational resources.
This balancing act is somewhat of an art and there are no easily defined rights or wrongs.
However, there are general guidelines that are useful to keep in mind while meshing.
But before we get into those it is worth to spend some more time on what a mesh is.\par

The most fundamental unit of the mesh is the element(s) that comprise the mesh.
There are many different types of elements that can be used for meshing and choosing which ones to use depend primarily on the spatial dimensionality of the model, the particularities of the geometry, and the physics that we wish to model.
Obviously different element types are by necessity needed to model a 2D vs. 3D geometry; you cannot mesh a 3D geometry with 2D squares.
This distinction is not very interesting and any lesson learnt about meshing in one of these dimensions is easily generalizable to the other.
Thus, we will exclusively discuss the meshing of 3D elements.\par

There are primarily four types of 3D mesh elements available - the tetrahedral, cuboid, prism, and pyramid - see Figure \ref{fig:3d_elements}.
These can be combined in various ways to represent any 3D geometry.
The most general out of these is the tetrahedral and will approximate any geometry well.
It is not always the most effective choice for meshing a geometry and another element type may be better suited.
This is easiest illustrated with an example.\par

Imagine that you are trying to simulate the laminar flow of some fluid through a pipe and been clever enough to realize that by virtue of symmetry only a wedge of the pipe is necessary to be explicitly modeled.
We also realize that the flow through the pipe is going to primarily have a gradient in the direction of the flow.
In this scenario, it might be beneficial to use prism elements rather than tetrahedrals.
Furthermore we could also primarily make the mesh fine in the flow direction while keeping it relatively coarse in other directions.
This would allow us to achieve a solution of high accuracy while still keeping the number of elements relatively small.\par


% TODO: Add figure
\begin{figure}
  %\includegraphics{}
  \caption{Four common mesh elements used to mesh three-dimensional geometries.}
  \label{fig:3d_elements}
\end{figure}




\subsection{Mesh Study}


\section{Solver Configuration}


\section{References}
\begin{comment}
Things to bring up:

- Why is it beneficial to model VI?


- Background on modeling in VI research
-- History & modeling applications
-- Limitations of other people's models
* Should this come before or after my mathematical description of VI?

- Mathematical description of VI modeling
* Start with a non-technical description of a CSM and the processes/physics underlying it. A nice graph showing all of it (including labeled physics and fluxes) would be nice.

- Short description of FEM
-- COMSOL
-- Basic mathematical concept
-- Advantages/disadvantages of this approach compared to others, i.e. why pick FEM over other numerical schemes?
-- Meshing
\end{comment}


\end{document}
