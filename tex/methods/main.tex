\documentclass[../thesis.tex]{subfiles}

% graphics path
\graphicspath{
  {../../figures/methods/},
  {./../figures/methods/}
}

\begin{document}
\chapter{Numerical Modeling of Vapor Intrusion}

\import{./}{abstract.tex}

\import{./}{intro.tex}
\import{./}{geometry.tex}
\import{./}{physics.tex}
\import{./}{indoor.tex}
\import{./}{soil_moisture.tex}
\import{./}{darcys_law.tex}
\import{./}{soil_transport.tex}
\import{./}{meshing.tex}
\import{./}{solvers.tex}
\import{./}{post_processing.tex}
\import{./}{appendix.tex}

\begin{comment}
Things to bring up:

- Why is it beneficial to model VI?


- Background on modeling in VI research
-- History & modeling applications
-- Limitations of other people's models
* Should this come before or after my mathematical description of VI?

- Mathematical description of VI modeling
* Start with a non-technical description of a CSM and the processes/physics underlying it. A nice graph showing all of it (including labeled physics and fluxes) would be nice.

- Short description of FEM
-- COMSOL
-- Basic mathematical concept
-- Advantages/disadvantages of this approach compared to others, i.e. why pick FEM over other numerical schemes?
-- Meshing
\end{comment}


\end{document}
