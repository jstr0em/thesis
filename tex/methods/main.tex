\documentclass[../thesis.tex]{subfiles}

% graphics path
\graphicspath{
  {../../figures/methods/},
  {./../figures/methods/}
}

\begin{document}
\chapter{Numerical Modeling of Vapor Intrusion}

\import{./}{abstract.tex}

% TODO: Restructure to reflect the COMSOL workflow
% - Keep main body simple
% - Use appendices as building/lego blocks to add complexity, e.g. one appendix about how to model a PP, and just that.

% TODO: Move model section as its own chapter.


\import{./}{intro.tex}
\import{./}{geometry.tex}
\import{./}{physics.tex}
\import{./}{indoor.tex}
\import{./}{soil_moisture.tex}
\import{./}{darcys_law.tex}
\import{./}{soil_transport.tex}
\import{./}{meshing.tex}
\import{./}{solvers.tex}
\import{./}{post_processing.tex}
\import{./}{appendix.tex}

\begin{comment}
No models are true representations of reality, but some of them may be useful.
Ever since Newton first wrote his laws of motion, mankind has tried to describe reality with an ever increasing number of mathematical statements.
With the advent of computation and advancements in numerical methods our capabilities to mathematically describe physical systems has dramatically increased.
Even so, real-world systems are too complex to be fully modeled, but mathematical representations may be used to approximate and reveal useful insights of how they function.\par

This is especially true for vapor intrusion (VI) models.
Often it is impossible or difficult to conduct controlled studies of VI sites making models an important tool for understanding these sites and the VI phenomena.
The previous chapter is proof of this as it is readily apparent that a multitude of VI models of varying complexity have been developed over the years, and has become an integral part of the scientific VI community.
From the simple Johnson \& Ettinger one-dimensional model to full three-dimensional finite element models we see that the increased complexity of the model allowed for a greater number of VI topics and phenomena to be explored.\par

The development of the three-dimensional finite element models begin with a conceptual site model (CSM) of a VI site.
In general when one develops models, it is best in the beginning to keep the model as simple as possible, and not to add overly complex features or excessive physics.


% TODO: Integrate or remove paragraphs below
To model vapor intrusion (VI), a number of partial differential equations (PDEs) which describe the relevant physics must be solved.
Many of these PDEs are also coupled in implicit and explicit ways, e.g. the PDE describing vapor flow in the soil affect the contaminant concentration, which in turn affect the indoor air concentration, which in turn also affects the contaminant concentration in the soil.

The indoor air space is perhaps the most important part of modeling VI, as the goal of these models ultimately is to predict indoor exposure given external factors.
One could therefore assume that most of the effort in modeling VI should be spent to accurately represent the interior.
This would be very impractical however, as building interiors are so diverse.
Even if one would spend the time to model an interior, this would dramatically increase the number of mesh elements required to solve the model.
Additionally, the air flow inside the building must be calculated, and even using a simplified version of Navier-Stokes, like Reynolds Averaged Navier-Stokes, the computational cost would be significant.
For these reasons, the indoor air space is simply modeled as a continuously stirred tank (CST), and paradoxically becomes the simplest component of the VI model.\par

Most of the effort will be spent to accurately model the physics in the soil underlying the building of interest.

\chapter{Geometry Generation}

To create our quarter geometry, only a few simple geometric objects and Boolean operations are required: two cuboids, two rectangles, one Boolean difference operation, and one Boolean join operation.
Figure \ref{fig:geometry} shows the resulting geometry.
Note that $z = \SI{0}{\metre}$ is the groundwater/soil interface and the plane of symmetry is around the $(x, y) = (\SI{0}{\metre},\SI{0}{\metre})$ axis\par

To create the soil surrounding the building using the COMSOL geometry generator:
\begin{enumerate}
  \item Create a \SI{15}{\metre} by \SI{15}{\metre} by \SI{4}{\metre} block with its base at $(x, y, z) = (\SI{0}{\metre},\SI{0}{\metre},\SI{0}{\metre})$. This is the entire soil domain.
  \item Create a \SI{5}{\metre} by \SI{5}{\metre} by \SI{1}{\metre} block with its base at $(x, y, z) = (\SI{0}{\metre},\SI{0}{\metre},\SI{3}{\metre})$. This will represent the volume that the house take up in the soil, i.e. the underground portion of the basement.
  \item Perform a difference operation, removing the "basement" block from the "soil" block.
\end{enumerate}
At this point you will see that a quarter soil domain has been created, with an empty space that represents a house with a foundation slab located \SI{1}{\metre} bgs.\par

The foundation crack will be modeled by joining two  \SI{1}{\centi\metre} wide strip that spans the perimeter of the surface that represents the house foundation.
This strip is created by joining two rectangles on foundation surface:
\begin{enumerate}
  \item Define a work plane \SI{3}{\metre} above zero. This allows us to place two-dimensional objects on the surface of or inside a three-dimensional object.
  \item On the work plane create a \SI{5}{\metre} by \SI{1}{\centi\metre} rectangle with its base at $(x, y) = (\SI{0}{\metre},\SI{5}{\metre} - \SI{1}{\centi\metre})$. This represents one side of the perimeter crack.
  \item Copy the rectangle and rotate it \SI{90}{\degree} around the corner of the foundation, i.e. $(x, y) = (\SI{5}{\metre} - \SI{0.5}{\centi\metre},\SI{5}{\metre} - \SI{0.5}{\centi\metre})$.
  \item Join the two rectangles to create a unified perimeter foundation crack.
\end{enumerate}
Now the geometry of this VI scenario is complete.\par

\section{Vapor Intrusion Physics and Governing Equations}

Throughout this section the books \textit{Contaminant hydrogeology}\cite{fetter_contaminant_1993} by Fetter, and \textit{Dynamics of fluids in porous media}\cite{bear_dynamics_1972} by Bear are used as sources, especially on topics regarding soil physics and contaminant transport in soils.
The COMSOL reference manuals\cite{comsol_comsol_nodate,comsol_subsurface_nodate} are also used as sources.

% TODO: Maybe change c_g -> c_{g,ck} ?

\subsection{The Indoor Environment}\label{sec:indoor}

The impacts on the indoor air space is perhaps the most important part of modeling VI, as the goal of these models ultimately is to predict indoor exposure given some external factors.
The indoor environment is, however, only modeled implicitly as a continuously stirred tank reactor (CSTR).\par
We assume that all contaminant entry into the house occurs via the foundation crack.
Once the contaminant enters the interior, it is instantly perfectly mixed, which is a key assumption of a CSTR.
Contaminant expulsion occurs via air exchange with the outdoor environment, and is regulated by \textit{air exchange rate} $A_e$, which dictates what fraction of the indoor air is exchanged with outdoor air over a given period of time.
For instance, a common air exchange rate for a house is \SI{0.5}{\per\hour}, i.e. half of the indoor air is exchanged every hour.
It should be noted that in this simple VI model implementation, we assume that there are no indoor sources nor that any sorption of contaminant to/from any indoor materials occurs.
Thus, the reaction term that would ordinarily be part of a CSTR is dropped (but is reintroduced in Chapter (TBD)) and the change in indoor contaminant concentration is thus given by \eqref{eq:cstr}. % TODO: Add chapter reference
\begin{equation}\label{eq:cstr}
  V_\mathrm{bldg}\frac{\partial c_\mathrm{in}}{\partial t} = n_\mathrm{ck} - V_\mathrm{bldg} A_e c_\mathrm{in}
\end{equation}
Here $c_\mathrm{in}$ [\si{\mol\per\metre\cubed}] is the indoor air contaminant concentration;
$n_\mathrm{ck}$ [\si{\mol\per\second}] is the contaminant entry (or exit) rate into the building via the foundation crack;
$A_e = \SI{0.5}{\per\hour}$ is the air exchange rate;
Finally, $V_\mathrm{bldg} = \SI{300}{\metre\cubed}$ is the volume of the house interior (or basement in this case).\par

A limitation of this approach is that we only consider one control volume or compartment, while in reality indoor contaminant concentrations can vary significantly between compartments, in particular between different floors.
There are VI models that use multiple compartments, which in essence are just coupled CSTRs\cite{murphy_multi-compartment_2011}.
Basements typically have higher indoor contaminant concentrations than other floors, so in this implementation we assume that our sole compartment is the house basement, which $V_\mathrm{bldg} = \SI{300}{\metre\cubed}$ reflects.\par

Solving \eqref{eq:cstr} requires us to determine the contaminant entry and air exchange rates.
Air exchange rates can vary quite significantly, and are a significant source of temporal variability in VI, a topic that will be further explored in Chapter (TBD). % TODO: Reference chapter
However, they typically vary around relatively well-known values as air exchange rates are regulated in building codes.
For residential buildings, it is typical that air exchange rate is around $A_e = \SI{0.5}{\per\hour}$ and thus for simplicity we will choose this value.\par

\paragraph{Contaminant entry into the building}

Contaminant entry rates are significantly more difficult to determine, as they depend on air velocity through the foundation breach and the concentration gradient across it.
The determination of these is the main point, and challenge in VI modeling.\par

% TODO Add figure showing schematic of the entry into the building

The contaminant entry $n_\mathrm{ck}$ is given by integrating the contaminant entry flux $j_\mathrm{ck}$ across the foundation crack boundary $A_\mathrm{ck}$.
\begin{equation}
  n_\mathrm{ck} = \int_{A_\mathrm{ck}} j_\mathrm{ck} dA
\end{equation}
The contaminant flux through the foundation crack is modeled as transport between two parallel plates and has an advective and a diffusive component.
\begin{equation}
  j_\mathrm{ck} = j_\mathrm{advection} + j_\mathrm{diffusion}
\end{equation}
Since contaminant concentration indoors is lower than it is in the soil or near foundation crack region a concentration gradient from the soil-gas to the indoor will exist.
The interior of the crack is not explicitly modeled, but assumed to only contain air and thus we assume the diffusion coefficient is the same as in air.
\begin{equation}
  j_\mathrm{diffusion} = - \frac{D_\mathrm{air}}{L_\mathrm{slab}} (c_{in} - c_g)
\end{equation}
here $D_\mathrm{air} = \SI{7.2e-6}{\metre\squared\per\second}$ is the diffusion coefficient of TCE in air as a sample contaminant of interest; other contaminant of common concern have comparable diffusivities. % TODO: Make sure this is right
$L_\mathrm{slab} = \SI{15}{\centi\metre}$ is the thickness of the foundation slab;
$c_{in}$ [\si{\mol\per\metre\cubed}] is the indoor contaminant concentration;
$c_g$ [\si{\mol\per\metre\cubed}] is the contaminant gas-phase concentration at the foundation crack boundary.\par

Advective transport through the slab can occur in both directions, i.e. contaminants can be carried from the soil into the house and from the house into the soil\cite{holton_creation_2018}.
The direction of this transport depend on the direction of the flow, with a positive sign indicating that airflow goes into the house.
\begin{equation}
  j_{advection} = \begin{cases}
    u_{ck} c_g & u_{ck} \geq 0 \\
    u_{ck} c_{in} & u_{ck} < 0
\end{cases}
\end{equation}
here $u_{ck}$ [\si{\metre\per\second}] is the airflow velocity through the foundation crack.

Thus the total contaminant transport through the foundation crack is given by \eqref{eq:contaminant_entry}.
\begin{equation}\label{eq:contaminant_entry}
  j_{ck} = \begin{cases}
    u_{ck} c_g - \frac{D_\mathrm{air}}{L_\mathrm{slab}} (c_{in} - c_g) & u_{ck} \geq 0 \\
    u_{ck} c_{in} - \frac{D_\mathrm{air}}{L_\mathrm{slab}} (c_{in} - c_g) & u_{ck} < 0
\end{cases}
\end{equation}
Not only will \eqref{eq:contaminant_entry} be used to calculate the contaminant entry rate into house, but it is a necessary boundary condition for calculating the contaminant concentration in the soil.
However, as we see, \eqref{eq:contaminant_entry} is a function of both the soil-gas concentration at the foundation crack boundary $c_g$ and the indoor contaminant concentration $c{in}$, thus these two are coupled and need to be solved simultaneously.\par

\subsection{The Importance Of Soil Moisture Content}\label{sec:van_genuchten}

The portion of soil between groundwater and ground surface is variably saturated with water and is called the vadose zone.
Under environmental conditions, TCE and many other contaminants are miscible in water, and will be partitioned between water and vapor phases, which has profound effects on the total contaminant transport in the vadose zone.
The rates of diffusion in liquids and gases usually differ by orders of magnitudes.
Likewise, advective transport in the two phases occurs at vastly different rates, and as such it is important to account for the effects of soil moisture content on transport of contaminant.\par

Water filling of soil pores from the groundwater surface is driven by a negative pressure gradient induced by surface tension, called capillary potential and is here represented by $\psi$.
This capillary potential is a function of the soil moisture content, and becomes increasingly negative as the water content decreases, and is zero when the soil matrix is saturated with water.
The capillary potential varies with the hydraulic properties of specific soil types.\par

In addition to the capillary potential, soil moisture content is driven by a gravitational potential, e.g. induced groundwater flow due to some elevation gradient.
The total soil moisture potential $\phi$, is the sum of the capillary and gravitational potentials, here expressed as a pressure head.
\begin{equation}
  \phi = \frac{\psi(\theta_w)}{\rho g} + h_\mathrm{g} = h + h_\mathrm{g}
\end{equation}
where $\phi$ [\si{\metre}] is the total soil moisture potential;
$\psi$ [\si{\pascal}] is the capillary potential;
$h$ [\si{\metre}] is the capillary potential expressed as a pressure head above the groundwater/soil interface;
$\theta_w$ is the volumetric moisture content by volume soil, i.e. dimensionless;
$\rho$ [\si{\kilo\gram\per\metre\cubed}] is the density of water;
$g$ [\si{\metre\per\second\squared}] is the acceleration due to gravity;
and $h_\mathrm{g}$ [\si{\metre}] is the gravitational potential above a reference plane.\par

In the VI scenario considered here, we assume that the groundwater is at steady-state and its soil interface is flat\footnote{
If these assumptions cannot be made and one wishes to model the groundwater and vadose zones as a continuum, the soil moisture content is determined by Richard's equation.
}.
Thus, the soil moisture content is entirely determined by the capillary potential $h$.\par

There are two common methods for modeling the capillary potential, one developed by \citeauthor{brooks_properties_1966}\cite{brooks_properties_1966} in 1966, and another by \citeauthor{van_genuchten_closed-form_1980}\cite{van_genuchten_closed-form_1980} in 1980.
Both of these are semi-empirical approaches and relies on experimentally determined parameters for a specific soil type to be used.
In this work, we only use van Genuchten's method, simply because parameters for a wide variety of soils have been made available by the EPA (Table \ref{tbl:soils}).\par

\subsubsection{van Genuchten's Soil-Water Retention Model}

The relationship between capillary pressure and moisture content is called \textit{soil moisture retention}, and is what van Genuchten's method describes.
Specifically, his method models the water saturation of the soil and is given by \eqref{eq:van_genuchten_saturation}.
\begin{equation}\label{eq:van_genuchten_saturation}
  \mathrm{Se} =
    \begin{cases}
      \frac{1}{(1 + (\alpha |h|)^n)^m} & h < 0 \\
    1 & h \geq 0
    \end{cases}
\end{equation}
here $\mathrm{Se}$ is the saturation, which ranges from 0 to 1, where 1 is fully saturated with water;
$\alpha$, $m$, and $n=\frac{1}{1-m}$ are the empirically determined van Genuchten parameters given in Table \ref{tbl:soils};
and $h$ [\si{\metre}] is the capillary pressure head as elevation above the groundwater/soil interface.\par

It is important to note that $\mathrm{Se} = 0$ does not mean that there is no moisture in the soil; soils retain a small amount of water in the matrix - residual moisture content (which is soil specific).
Thus, the soil moisture content is given by \eqref{eq:van_genuchten_soil_moisture}
\begin{equation}\label{eq:van_genuchten_soil_moisture}
  \theta_w =
    \begin{cases}
      \theta_r + \mathrm{Se}(\theta_t - \theta_r) & h < 0 \\
      \theta_t & h \geq 0
    \end{cases}
\end{equation}
where $\theta_w$ is the volumetric soil moisture content;
$\theta_t$ is the soil porosity;
and $\theta_r$ is the residual moisture content.\par

By extension, the soil gas or air content is given by
\begin{equation}\label{eq:gas_porosity}
  \theta_g = \theta_t - \theta_w
\end{equation}
An example of the soil saturation and moisture content as a function of height above groundwater is shown in Figure \ref{fig:retention_curve}.
Note the steep decline in moisture content near the groundwater interface - that is the capillary zone and we will see that this zone presents a significant barrier to contaminant transport from the groundwater in section \ref{sec:transport}.\par

\begin{figure}[htb!]
  \centering
  \begin{subfigure}{\textwidth}
    \centering
    \includegraphics[width=0.75\textwidth]{van_genuchten.pdf}
    \caption{Example of a soil moisture retention curve as a function of pressure head above the groundwater/soil interface.}
    \label{fig:retention_curve}
  \end{subfigure}
  \begin{subfigure}{\textwidth}
    \centering
    \includegraphics[width=0.75\textwidth]{relative_permeability.pdf}
    \caption{Relative air and water permeability in the vadose zone or above the groundwater interface.}
    \label{fig:relative_permeability}
  \end{subfigure}
  \caption[van Genuchten soil moisture retention curve and relative permeability]{The van Genuchten soil moisture retention curve and relative permeability of soil to air and water transport.}
\end{figure}

The presence of water in the soil matrix has profound implications for transport as the pore space available for transport may be restricted.
For instance, in the capillary zone, contaminant transport is mostly limited by the water phase, while gas phase transport is extremely limited because air-filled porosity is limited and largely isolated.
The opposite is true near the ground surface, where most of the pore space is filled with air.\par

Soil already limits transport, i.e. it is harder to pump water through a soil column than a pipe of the same diameter.
This extra phase-specific transport inhibition is modeled by a \textit{relative permability} and is given by \eqref{eq:van_genuchten_relative_permeability}.
\begin{equation}\label{eq:van_genuchten_relative_permeability}
  k_r =
    \begin{cases}
      (1-\mathrm{Se})^l \big(1 - \mathrm{Se}^\frac{1}{m} \big)^{2m} & h < 0 \\
      0 & h \geq 0
    \end{cases}
\end{equation}
here $k_r$ is the relative permeability for air;
and $l = 0.5$ is another van Genuchten parameter.\par % TODO: How do I motivate this?

The relative permeability varies from 0 to 1, where 0 indicates that the soil matrix is completely impermeable to the fluid, while a value of 1 means that there is no additional permeability cost.
Figure \ref{fig:relative_permeability} shows how the relative gas permeability varies in the vadose zone.\par

\subsection{Airflow In The Vadose Zone}\label{sec:darcys_law}

In our VI scenario, the depressurized house induces an advective airflow from the ground surface, through the vadose zone, and into the house via the foundation crack, carrying contaminant vapors with it.
This airflow is modeled using a modified version of Darcy's Law.
The modification is made to account for the variable moisture content in the vadose zone, which is established in section \ref{sec:van_genuchten}.\par

Darcy's Law describes the flow of a fluid through a porous medium.
This flow is driven by a pressure gradient, and its magnitude depends on the permeability of the porous medium and the fluids viscosity.
\begin{equation}\label{eq:darcys_law_saturated}
  \vec{u} = -\frac{\kappa}{\mu} \nabla p
\end{equation}
here $\vec{u}$ [\si{\m\per\second}] is the airflow velocity vector;
$\kappa$ [\si{\metre\squared}] is the permeability of the porous media;
$\mu$ [\si{\pascal\second}] is the dynamic viscosity of the fluid;
and $\nabla$ [\si{\pascal\per\metre}] is the pressure gradient.\par

This \eqref{eq:darcys_law_saturated} formulation of Darcy's Law assumes that the porous media is saturated with the fluid,\footnote{Darcy's Law also assumes that the flow is in the laminar regime, i.e. the Reynolds number $\mathrm{Re} < 1$.
Due to the small pressure gradients in most VI scenarios, this assumption is rarely unfulfilled, but if it is, then Brinkman's equation should be used instead.} % TODO: Source for Brinkmans equation
hence the need to modify this expression.
While porosity is not directly part of \eqref{eq:darcys_law_saturated}, it is an intrinsic property of the permeability $\kappa$ of the porous media; the variably saturated pores variably changes the permeability.
This variation in permeability is modeled using the relative permeability expression from van Genuchten's equation \eqref{eq:van_genuchten_relative_permeability}.
The effective soil permeability is the product of the soil permeability and its relative permeability giving our modified Darcy's Law \eqref{eq:darcys_law_unsaturated}.
\begin{equation}\label{eq:darcys_law_unsaturated}
  \vec{u} = -\frac{(1-k_r)\kappa}{\mu} \nabla p
\end{equation}
Recall that by definition $k_r$ is the relative permeability of the soil to \textit{water} and thus $1-k_r$ is that to \textit{gas}.\par

To calculate the soil-gas velocity field in the vadose zone, we need a continuity equation, which for fluid flow is \eqref{eq:fluid_continuity}.
\begin{equation}\label{eq:fluid_continuity}
  \frac{\partial \rho}{\partial t} + \nabla \cdot (\rho \vec{u}) = 0
\end{equation}
Inserting our modified Darcy's Law for the velocity gives \eqref{eq:vapor_transport}.
\begin{equation}\label{eq:vapor_transport}
  \frac{\partial}{\partial t} (\rho \theta_g) + \nabla \cdot \rho \Big( -\frac{(1-k_r) \kappa}{\mu} \nabla p \Big) = 0
\end{equation}
where $\theta_g$ is the gas-filled porosity of the soil from \eqref{eq:gas_porosity};
$\rho = \SI{1.225}{\kilogram\per\metre\cubed}$ is the density of air;
and $\mu = \SI{18.5e-6}{\pascal\second}$ is the dynamic viscosity of air.
Contaminant vapor concentrations are typically highly dilute in VI scenarios, and therefore we assume that the contaminant does not affect the transport properties of air.\par

In order to solve \eqref{eq:vapor_transport} we need to define some boundary conditions.
In our CSM, air is pulled from the atmosphere through the ground surface and into the building via the foundation crack.
To model this only three boundary conditions are required.\par

\paragraph{Boundary Conditions}

The first is to define a pressure gauge, i.e. a reference point for where the pressure is zero, which is where air will be pulled from.
This is the applied to the ground surface boundary.
The second is that we apply the indoor/outdoor pressure difference (-5 Pa) to the foundation crack boundary.
The third type is applied to all remaining boundaries and is a no flow boundary condition, indicating that no flow passes through these boundaries.
We also make sure that we specify the symmetry planes present.
\begin{align}
  &\text{Ground surface} &p = 0 \; \mathrm{(Pa)} \\
  &\text{Foundation crack} &p = p_\mathrm{in/out} = -5 \; \mathrm{(Pa)} \\
  &\text{Remaining} &-\vec{n}\cdot\rho\vec{u} = 0
\end{align}
where $\vec{n}$ is the boundary normal vector.\par

\paragraph{Initial Conditions}

For steady-state problems, the initial conditions do not influence the solution.
Transient simulations however, require initial conditions and these are assumed to be given by the steady-state solution.\par

\paragraph{Basis function}

A hat function is used as the basis function for solving Darcy's Law in FEM.\par

\subsection{Mass Transport In The Vadose Zone}\label{sec:transport}

To begin deriving a governing equation for contaminant transport in our VI scenario, we consider the continuity equation which states that the change of concentration in some volume of space depends on the advective and diffusive fluxes in and out of the system, as well as any generation or consumption inside the system.
\begin{equation}
  \frac{\partial c}{\partial t} + \nabla \cdot(j_\mathrm{adv} + j_\mathrm{diff}) - G = 0
\end{equation}
here $c$ [\si{\mol\per\metre\cubed}] is the concentration of the chemical species;
$t$ [\si{\second}] is time;
$j_\mathrm{adv}$ and $j_\mathrm{diff}$ [\si{\mol\per\second\per\metre\squared}] are the advective and diffusive fluxes respectively;
and $G$ [\si{\mol\per\second}] is the generation or consumption of the chemical species.\par

In our model we will assume that $G = 0$ as the groundwater is the sole contaminant source and TCE does not readily degrade in soils. % TODO: Source for TCE degradation
However, this term should remain and an appropriate expression developed if one wants to model:
\begin{itemize}
  \item Biodegradation of some compound in the soil.
  \item Radon intrusion (remember radon gas is generated in soils and rocks).
  \item A soil or subsurface source, e.g. a leaky tank or evaporation from a (pure) contaminant spill.
\end{itemize}\par

The advective flux is given by
\begin{equation}
  j_\mathrm{adv} = \vec{u} c
\end{equation}
where $\vec{u}$ [\si{\metre\per\second}] is a velocity vector.
The diffusive flux is given by Fick's Law
\begin{equation}
  j_\mathrm{diff} = -D \nabla c
\end{equation}
where $D$ [\si{\metre\squared\per\second}] is the diffusion coefficient of the contaminant in the solute;
and $\nabla c$ [\si{\mol\per\metre\cubed\per\metre}] is a concentration gradient.
Thus we get the advection-diffusion equation which generally governs transport of a chemical species
\begin{equation}
  \frac{\partial c}{\partial t} + \nabla \cdot(\vec{u} c + -D \nabla c) = 0
\end{equation}
However, this will not accurately represent contaminant transport in the vadose zone due to
\begin{itemize}
  \item Contaminant transport occurs inside a variably saturated porous matrix, significantly affecting transport properties.
  \item The contaminant concentration in the vadose zone will be distributed between three phases - gas, water, and solid (via sorption).
\end{itemize}\par

The total contaminant concentration in the soil will be used in lieu of just concentration, i.e. $c \rightarrow c_T$ and thus the total contaminant concentration is the sum of the gas, water, and solid phase concentrations.
\begin{equation}
  c_T = \theta_w c_w + \theta_g c_g + c_s \rho_b
\end{equation}
Here $\theta_g$ and $\theta_w$ are the gas-filled and water-filled porosities respectively;
$c_w$ and $c_g$ [\si{\mol\per\metre\cubed}] are the contaminant concentrations in water and gas respectively;
$c_s$ [\si{\mol\per\kilogram}] is the solid phase or sorbed concentration per mass of soil;
and $\rho_b = (1-\theta_t) \rho$ [\si{\kilogram\per\metre\cubed}] is the bulk density of the soil, which can be calculated from the soil porosity $\theta_t$ and solid phase density of the soil $\rho$ [\si{\kilogram\per\metre\cubed}].\par

The attentive reader will now notice that our governing equation depend on three variables instead of one.
However, remember that we're concerned with low contaminant concentrations, we can relate the gas and liquid phase concentrations via Henry's Law \eqref{eq:henrys_law}
\begin{equation}\label{eq:henrys_law}
  c_g = K_H c_w
\end{equation}
where $K_H = 0.402$ is the dimensionless Henry's Law constant for TCE at \SI{20}{\degreeCelsius}. % TODO: TCE K_H source
We also assume that there are no temperature gradients throughout the vadose zone.\par

The solid phase concentration can be related to the others via a linear sorption isotherm.
Here either the gas-solid or water-solid sorption interaction can be chosen; the former is used in Chapter (TBD) to we will explore effect of gas-solid sorption. % TODO: Chapter reference
\begin{equation}
  c_s = \begin{cases}
    K_p c_w & \text{Water-solid sorption} \\
    K_p c_g = K_p K_H c_w & \text{Gas-solid sorption}
\end{cases}
\end{equation}
here $K_p$ [\si{\metre\cubed\per\kilogram}] is a sorption partitioning coefficient.\par

Another approach is to simply ignore the role of sorption completely, i.e. $K_p = 0$, which has historically been done in VI modeling and is done in this example too.
The reason for this is two-fold.
\begin{enumerate}
  \item Relevant sorption data has not been available.
  \item With an infinite source assumption, and at steady-state, sorption doesn't affect the solution; these have been common assumptions in most VI models so far.
\end{enumerate}
Regardless, we will continue with the sorption $K_p$ term, because this will become relevant in Chapter (TBD) where experimentally derived relevant sorption data is available.\par % TODO: Chapter reference

With Henry's Law and the linear sorption assumption we can relate the total contaminant concentration in the soil matrix to the water-phase contaminant concentration.
\begin{equation}
  c_T = (\theta_w + \theta_g K_H + K_H K_p \rho_b) c_w
\end{equation}
The terms in front of $c_w$ are collected as $R = (\theta_w + \theta_g K_H + K_H K_p \rho_b)$.
This term is called the \textit{retardation factor} and reflects the increased contaminant residence time in the matrix due to shifting between the phases, and becomes an important factor in transient transport simulations.\par

In the vadose zone, advective transport can occur in both the water and gas phases inside the soil pores.
\begin{equation}
  j_\mathrm{adv} = \vec{u}_{w,pore} c_w \theta_w + \vec{u}_{g,pore} c_g \theta_g
\end{equation}
here $\vec{u}_{w,pore}$ and $\vec{u}_{g,pore}$ [\si{\metre\per\second}] are the water and gas phase \textit{pore} velocities vectors respectively.
However, from mass conservation, we know that the product of the pore velocity and porosity gives the superficial velocity of a fluid in porous media, i.e. the Darcy's Law velocities.
This together with Henry's Law gives
\begin{equation}
  j_\mathrm{adv} = (\vec{u}_w + \vec{u}_g K_H ) c_w
\end{equation}
and here $\vec{u}_w$ + $\vec{u}_g$ [\si{\metre\per\second}] are the Darcy or superficial velocity vectors.\par

In section \ref{sec:van_genuchten} we assumed that there is no gravitational water potential in the soil matrix, and it follows that the water in the pores is stationary, i.e. $\vec{u}_w = 0$ giving
\begin{equation}
  j_\mathrm{adv} = \vec{u}_g K_H c_w
\end{equation}
where $\vec{u}_g$ is the solution from solving Darcy's Law in section \ref{sec:darcys_law}.\par

To model a scenario where there is a gravitational water potential, one would have to solve two-phase Darcy's Law to get both $\vec{u}_g$ and $\vec{u}_w$.
This significantly complicates the mass transport aspect as well, and as such is beyond the scope of this work.\par

The diffusive transport expression likewise needs to be adjusted, and the total diffusive flux through the pore matrix is given by
\begin{equation}
  j_\mathrm{diff} = -(D_w \theta_w \tau_w \nabla c_w + D_g \theta_g \tau_g \nabla c_g)
\end{equation}
here $D_w = \SI{1.02e-9}{\metre\squared\per\second}$ and $D_g = \SI{6.87e-6}{\metre\squared\per\second}$ are the contaminant diffusion coefficients of TCE in pure water and gas respectively; % TODO: Add diffusion values here?
and $\tau_w$ and $\tau_g$ are the water and gas tortuosity terms.\par

Due to the irregular shapes of pores, diffusion of a chemical species will inevitably often occur along a tortuous path, which the tortuosity terms attempt to capture.
As tortuosity depend on the structure of the porous matrix, it is difficult to accurately portray, but a popular approach is to use Millington \& Quirks model\cite{millington_permeability_1961}.
\begin{equation}\label{eq:millington-quirk}
  \tau_w = \frac{\theta_w^{\frac{7}{3}}}{\theta_t^2}, \; \tau_g = \frac{\theta_g^{\frac{7}{3}}}{\theta_t^2}
\end{equation}
here $\theta_t$ is the total or saturated porosity of the soil matrix.
Another popular approach is Bruggeman's model, or if possible, a custom version can be used.\par % TODO: Add Bruggeman's reference here

Combining \eqref{eq:millington-quirk} and \eqref{eq:henrys_law} in our diffusion flux expression gives
\begin{equation}
  j_\mathrm{diff} = -\Big(D_w \frac{\theta_w^{\frac{10}{3}}}{\theta_t^2} + D_g \frac{\theta_g^{\frac{10}{3}}}{\theta_t^2} K_H\Big) \nabla c_w
\end{equation}
the terms in front of $\nabla c_w$ can be collected as an effective diffusion coefficient $D_\mathrm{eff}$ [\si{\metre\squared\per\second}], which with our isothermal vadose zone assumption only depends on the soil moisture content.
Thus we get the final diffusive flux expression
\begin{equation}
  j_\mathrm{diff} = - D_\mathrm{eff}\nabla c_w
\end{equation}
Figure \ref{fig:D_eff} shows how the effective diffusivity varies from being close to that of the pure water diffusivity near the capillary zone, and increases to something closer to gas-phase diffusivity as the soil moisture decreases.\par

\begin{figure}
  \includegraphics[width=\textwidth]{effective_diffusivity.pdf}
  \caption{Effective diffusivity of TCE in the vadose zone using Millington-Quirks model. Soil water and gas filled porosites are calculated using van Genuchten's equations.}
  \label{fig:D_eff}
\end{figure}

Putting all this together finally gives us the governing equation for contaminant transport in the vadose zone for our modeled VI scenario.
\begin{equation}\label{eq:mass_transport}
  R \frac{\partial c_w}{\partial t} = \nabla \cdot (D_\mathrm{eff} \nabla c_w) - K_H \vec{u}_g \cdot \nabla c_w
\end{equation}
To solve this we need to define some boundary and initial conditions.\par

% TODO: Rewrite these in the form c(t, vec{x}) = ..., i.e. c(t, atm) = 0, c(t, source) = c_gw etc

\paragraph{Boundary Conditions}

In this VI scenario, the sole contaminant source is assumed to be the homogenously contaminated groundwater, which we assume to have a fixed concentration.
The atmosphere acts as a contaminant sink and thus this is simply a zero concentration boundary condition.
Contaminants leave the soil domain and enter the building through a combination of advective and diffusive gas phase transport.
The boundary condition applied to all other boundaries is a no-flow boundary.
\begin{align}
  &\text{Atmosphere} & c_w = \SI{0}{\mol\per\metre\cubed} \\
  &\text{Groundwater} & c_w = c_{gw} = \SI{0.1}{\mol\per\metre\cubed} \\
  &\text{Foundation crack} & -\vec{n} \cdot \vec{N} = \frac{-j_{ck}}{K_H} \; \si{\mol\per\metre\squared\per\second}\\
  &\text{All other} & -\vec{n} \cdot \vec{N} = \SI{0}{\mol\per\metre\squared\per\second}
\end{align}
$\vec{n} \cdot \vec{N}$ is the dot product between the boundary normal vector and the contaminant flux;
$j_ck$ is the contaminant vapor flux into the building.
We assume that only contaminants in the gas phase enter the building, and dividing $j_{ck}$ by $K_H$ we get proper accounting in terms of the water phase concentration accounting in the main transport equation \ref{eq:mass_transport}.\par

\paragraph{Initial Conditions}

For steady-state problems, the initial conditions do not influence the solution.
Transient simulations however, require initial conditions and these are assumed to be given by the steady-state solution.\par

\paragraph{Basis function}

A second order polynomial (quadratic) function is used as the basis function for solving the transport equation.\par

\section{Meshing}\label{sec:meshing}

A mesh is a collection of small discrete elements that in combination form a larger geometry or domain.
Meshing is the process of generating a mesh.
Meshing is perhaps one of the most important and challenging aspects of solving a FEM model and a well-constructed mesh is necessary for accurate and reliable results.\par

In theory, an infinitely fine mesh will give the analytical solution to a PDE but obviously the computational costs would be infinite then as well; one must always balance the accuracy of the solution and computational resources.
This balancing act is somewhat of an art and there are no easily defined rights or wrongs.
However, there are general guidelines that are useful to keep in mind while meshing.
But before we get into those it is worth to spend some more time on what a mesh is.\par

The most fundamental unit of the mesh is the element(s) that comprise the mesh.
There are many different types of elements that can be used for meshing and choosing which ones to use depend primarily on the spatial dimensionality of the model, the particularities of the geometry, and the physics that we wish to model.
Obviously different element types are by necessity needed to model a 2D vs. 3D geometry; you cannot mesh a 3D geometry with 2D squares.
This distinction is not very interesting and any lesson learnt about meshing in one of these dimensions is easily generalizable to the other.
Thus, we will exclusively discuss the meshing of 3D elements.\par

There are primarily four types of 3D mesh elements available - the tetrahedral, cuboid, prism, and pyramid - see Figure \ref{fig:3d_elements}.
These can be combined in various ways to represent any 3D geometry.
The most general out of these is the tetrahedral and will approximate any geometry well.
It is not always the most effective choice for meshing a geometry and another element type may be better suited.
This is easiest illustrated with an example.\par

Imagine that you are trying to simulate the laminar flow of some fluid through a pipe and been clever enough to realize that by virtue of symmetry only a wedge of the pipe is necessary to be explicitly modeled.
We also realize that the flow through the pipe is going to primarily have a gradient in the direction of the flow.
In this scenario, it might be beneficial to use prism elements rather than tetrahedrals.
Furthermore we could also primarily make the mesh fine in the flow direction while keeping it relatively coarse in other directions.
This would allow us to achieve a solution of high accuracy while still keeping the number of elements relatively small.\par


% TODO: Add figure
\begin{figure}
  %\includegraphics{}
  \caption{Four common mesh elements used to mesh three-dimensional geometries.}
  \label{fig:3d_elements}
\end{figure}




\subsection{Mesh Study}

\section{Solver Configuration}

A solver(s) is required to solve the VI model, and a few considerations need to be taken when choosing one.
For simplicity we will now first consider a stationary or steady-state problem.
Since our model is a multiphysics problem, i.e. many of the physics depend on each other, we first need consider how to couple our physics.
The physics can be coupled by either using a \textit{segregated} or \textit{fully coupled} approach.\par

\paragraph{Segregated vs. fully coupled physics}

In a segregated solver, each governing equation or physics is solved separately in a specific order.
For instance, in our VI example we could solve Darcy's Law first, get some solution, then use that in the transport equation, solve that, and then solve the indoor concentration equation, i.e. we solve one system of equation per physics.
These steps are simply iterated until convergence occurs in all of the separated steps.
The fully coupled approach assembles a single large system of equation from all of the physics.
Both of these approaches will reach the same solution, but the fully coupled approach will do so faster, but at the expense of using more memory.\par

\paragraph{Direct vs. iterative solver}

Within each of these coupling approaches, we need to specify a solver to solve the system of equations.
Here we are again faced with a choice, and we could either use a \textit{direct} or \textit{iterative} solver.
Direct solvers, as the name implies, arrive at a solution directly and are based on LU-decomposition.
Iterative solvers on the other hand, iteratively approach the solution, and are based on conjugate gradient method.
The advantage of direct solvers is that they are faster, but use more memory, while iterative solvers are slower but use less memory.
In terms of choosing a solver algorithm, there are many options, but MUMPS and GMRES will be used as the respective algorithm for direct and iterative solvers.\par

\paragraph{Time-dependent solvers}

To solve a transient or time-dependent problem (which will be done in subsequent chapters) a solver to step forward in time is required.
A too large time step will cause stability issues and ultimately convergence will be impossible, but obviously some discrete time step is required for a solution to be achievable.
A time-dependent solver picks an appropriate time-step and there are some popular approaches, such as using some high-order Runge-Kutta (RK) or backwards differentiation formula (BDF).
Regardless of the type of solver, for each time step the system of equations will be solved using one of the aforementioned solvers.
The difference between RK and BDF is that RK explicitly discretizes time while BDF does so implicitly.
In this work we will only use BDF as it is more stable than RK.\par

\paragraph{Choosing solvers}

The choice of solver will not affect (or should not at least) the solution to the problem.
However, it can have a large impact on computational time and resources, and these considerations dictate solver choice (this is also partially dependent on the mesh used, as this will affect memory usage too).
In this example, and throughout the models used in this work, we will favor speed over memory and therefore fully couple all our equations and use direct solvers.\par

\subsection{Adaptive Mesh Refinement}

The accuracy of the solution obtained by FEM is dependent on the quality of the mesh, something that was discussed in section \ref{sec:meshing}.
While the mesh designer can do much to create a mesh that performs well for the particular problem posed, refinement of the mesh is often needed and should be performed for every new model.\par

There are two types of mesh refinements in FEM.
The first type reduces the size of the elements and thereby the accuracy of the solution, this is called \textit{h-type} refinement ($h$ is often used to denote the mesh size).
The second increases the order of the polynomial of the basis function, called \textit{p-type} refinement which will likewise increases the solution accuracy.\par

h-type refinement is generally more attractive because it is simpler and the computational cost of p-type refinement increase faster than h-type.
However, p-types are useful if the user imports an already existing mesh, and is unable to change it, rendering h-type refinement impossible.
These two method can be combined to perform a \textit{hp-type} refinement.\par

Refinement is usually done by an algorithm, which is possible because FEM has the built-in ability to estimate the local error of the solution anywhere in the domain.
The downside with using an algorithm is that the user has little control over how the mesh is refined.
The user can also manually refine the mesh by solving the model and plot how some relevant metric converges as the mesh is refined.
This can be a very time consuming, and therefore algorithms are usually preferable; a hybrid solution is to manually alter the mesh after the algorithmic mesh refinement.\par

Refinement can either be done locally or globally.
Global refinement involves defining some singular metric that will be used to evaluate the quality of the mesh, e.g. one might use the total stress in a metal bar as a metric here.
In local refinement, one still has to define some metric for evaluating the quality of the refinement, but evaluation only occurs on a subset of the domain, e.g. the stress on just one boundary of the same metal bar.
In both approaches the elements that have the largest estimated local error are refined; this error estimation is an inherent feature of FEM.
The optimal type of refinement varies by problem, but a global refinement will generally be more computationally expensive.\par

In this work we will use a global h-type refinement and use the indoor contaminant concentration $c_{in}$ as our refinement metric.
COMSOLs refinement algorithm has the nice ability to reinitialize the mesh, and can thereby coarsen elements, i.e. increase $h$ where the local error is very small.
This is handy as a fine mesh is not needed far away from the foundation crack - saving computational resources.
In this example we will tell the algorithm to refine the mesh three times, and stop if the total number of elements exceed 1 million, with a maximum coarsening factor of 3, and element growth rate of 1.7, i.e. the number of elements increase by roughly 70\% each iteration.\par

The result of this refinement can be seen in Figure \ref{fig:mesh_refinement} where the original and refined mesh are juxtaposed.
Notice how the mesh is now denser near the foundation, the boundary layers tighter (in particular near the groundwater boundary), and how the elements are larger in the periphery.
The original and refined meshes has 362,657 and 1,065,743 elements respectively.\par

\begin{figure}[htb!]
  \centering
  \begin{subfigure}[b]{\textwidth}
    \includegraphics[width=\textwidth]{meshed_model.png}
    \caption{Original mesh. 362,657 elements.}
    \label{fig:mesh_before_refinement}
  \end{subfigure}
  \begin{subfigure}[b]{\textwidth}
    \includegraphics[width=\textwidth]{mesh_refined.png}
    \caption{Refined mesh after two steps of global refinement w.r.t. the indoor contaminant concentration. 1,065,743 elements.}
    \label{fig:mesh_after_refinement}
  \end{subfigure}
    \caption{Original and refined mesh.}
    \label{fig:mesh_refinement}
\end{figure}

\section{Post-processing \& Results}

One of the benefits of using a FEM software like COMSOL is its advanced post-processing capabilities.
This allows the user to examine the physics driving VI in great detail.
Figure \ref{fig:model_pressure} shows the resulting pressure field from solving Darcy's Law, as well as the associated airflow streamlines in the soil.
Here we see the pressure in the near foundation crack region is roughly the same as the house pressurization of \SI{-5}{\pascal}, which quickly decreases towards the ground surface.
It is also apparent how this pressure field induces a airflow from the ground surface, with air near the house heading relatively straight to the foundation crack, whereas the air further away from the house penetrates deeper into the soil and almost "whirlwinds" underneath the house.\par

\begin{figure}[htb!]
  \centering
  \includegraphics[width=0.75\textwidth]{model_pressure.png}
  \caption[Modeled Darcy's pressure field in soil]{Pressure field from Darcy's Law with associated airflow streamlines.}
  \label{fig:model_pressure}
\end{figure}

\begin{figure}[htb!]
  \centering
  \includegraphics[width=0.75\textwidth]{model_concentration.png}
  \caption[Modeled contaminant concentration in soil]{Contaminant concentration in the soil, normalized to groundwater concentration and log-transformed, with transport streamlines.}
  \label{fig:model_concentration}
\end{figure}

The contaminant concentration in the soil, normalized to the groundwater source concentration and log-transformed, with the contaminant flux streamlines, is examined in Figure \ref{fig:model_concentration}.
Here see that far away from the house, the contaminant vapor simply diffuse straight from the groundwater source to the atmosphere, while beneath the house foundation, contaminant vapors accumulate because the foundation acts as a diffusion blocker.
Based on those streamlines we can conclude that the advective component of the flux is very here small.
Perhaps surprisingly, we do not see a significant advective transport downwards along the wall of the house.
However, considering that the soil type is sandy loam, airflow velocities are expected to be small.\par

One might think that advective transport is large in the horizontal direction along the foundation slab, as the transport and airflow streamlines are so similar.
However, by inspecting Figure \ref{fig:model_velocity_crack} we see that airflow velocities are not greater here than elsewhere, and therefore the advective transport is not either.
To make sense of this, we can inspect the horizontal diffusive flux, divided by the magnitude of the total flux
\begin{equation}
  \frac{j_\mathrm{diff,y-direction}}{|j_\mathrm{total}|}
\end{equation}
to see what portion of the total contaminant flux transport the diffusive horizontal represents here.
Figure \ref{fig:model_horizontal_diff} shows that the horizontal contaminant transport underneath the foundation is in fact driven by the large contaminant concentration gradient between the region underneath and outside the house foundation.
This shows the power of modeling and how it can reveal things that at first seem intuitively correct, but in fact are not.\par

\begin{figure}[htb!]
  \centering
  \includegraphics[width=0.75\textwidth]{model_velocity_crack.png}
  \caption{Airflow velocity $\vec{u}_g$ [\si{\milli\metre\per\hour}] near the foundation crack with associated its streamlines.}
  \label{fig:model_velocity_crack}
\end{figure}

\begin{figure}[htb!]
  \centering
  \includegraphics[width=0.75\textwidth]{model_transport_flux_y.png}
  \caption[Analysis of horizontal diffusion flux.]{Horizontal (y-axis) diffusive flux component normalized to the magnitude of the total flux. A value of 1 here indicates that the total contaminant transport flux is due to diffusion, while 0 would indicate the opposite - that all contaminant transport is due to advection. The sign signifies the direction, with positive and negative values indicating a flux in the rightward and leftward direction respectively. E.g. a value of -0.8 indicates that the 80\% of the magnitude of the contaminant transport is due to horizontal (along the y-axis) diffusion, and occurs leftward.}
  \label{fig:model_horizontal_diff}
\end{figure}

Another useful feature of post-processing is that it can be used for bug searching and to evaluate where the mesh can be potentially improved.
When the transport equation is used to numerically model contaminant transport, there is a tendency for the solution to oscillate around the "true" solution, and thereby violate mass conservation, if the mesh size in a particular element is too large.
This can be quantified by the cell Péclet number, which characterizes the relative magnitude of advection/diffusion in a cell.
\begin{equation}
  \mathrm{Pe_{cell}} = \frac{\mathrm{adv_{cell}}}{\mathrm{diff_{cell}}} = \frac{u_g h}{2 D_\mathrm{eff}}
\end{equation}
here $u_g$ [\si{\metre\per\second}] is the soil-gas airflow velocity;
$h$ [\si{\metre}] is the mesh size in the element or cell;
and $D_\mathrm{eff}$ [\si{\metre\squared\per\second}] is the effective diffusivity in the cell.
If $\mathrm{Pe_{cell}} > 1$ there is a risk that this oscillating behavior will manifest.
Small exceedances, $~\mathrm{Pe_{cell}} < 25$, are usually mitigated by various stabilization schemes, which are inherently integrated into COMSOL as well as many other FEM packages, but for larger values further mesh refinement may be required.\par

Figure \ref{fig:model_cell_peclet} shows $\mathrm{Pe_{cell}}$ as a volume plot, and excludes all values that fall below one.
As we can see, only the region close to the groundwater exceeds $\mathrm{Pe_{cell}}$, which is due to the very small $D_\mathrm{eff}$ there.
The exceedance is small, so the stabilization scheme is able to compensate which is confirmed by Figure \ref{fig:model_concentration} (no oscillations visible).
This is also a region where even if such oscillations occurred, would probably not affect the indoor contaminant concentration.
Regardless, Figure \ref{fig:model_cell_peclet} shows where the mesh may potentially be refined, which comes in handy to know if one runs a model where airflow velocities are significantly higher than in this example.\par

\begin{figure}[htb!]
  \centering
  \includegraphics[width=0.75\textwidth]{model_cell_peclet.png}
  \caption{Volume plot showing where the cell Péclet number exceeds 1 and its actual value. I.e. it suggests where the mesh may be improved.}
  \label{fig:model_cell_peclet}
\end{figure}


\appendix
\clearpage
\begin{appendices}

  \section{Geometry Generation}

  To create our quarter geometry, only a few simple geometric objects and Boolean operations are required: two cuboids, two rectangles, one Boolean difference operation, and one Boolean join operation.
  Figure \ref{fig:geometry} shows the resulting geometry.
  Note that $z = \SI{0}{\metre}$ is the groundwater/soil interface and the plane of symmetry is around the $(x, y) = (\SI{0}{\metre},\SI{0}{\metre})$ axis\par

  To create the soil surrounding the building using the COMSOL geometry generator:
  \begin{enumerate}
    \item Create a \SI{15}{\metre} by \SI{15}{\metre} by \SI{4}{\metre} block with its base at $(x, y, z) = (\SI{0}{\metre},\SI{0}{\metre},\SI{0}{\metre})$. This is the entire soil domain.
    \item Create a \SI{5}{\metre} by \SI{5}{\metre} by \SI{1}{\metre} block with its base at $(x, y, z) = (\SI{0}{\metre},\SI{0}{\metre},\SI{3}{\metre})$. This will represent the volume that the house take up in the soil, i.e. the underground portion of the basement.
    \item Perform a difference operation, removing the "basement" block from the "soil" block.
  \end{enumerate}
  At this point you will see that a quarter soil domain has been created, with an empty space that represents a house with a foundation slab located \SI{1}{\metre} bgs.\par

  The foundation crack will be modeled by joining two  \SI{1}{\centi\metre} wide strip that spans the perimeter of the surface that represents the house foundation.
  This strip is created by joining two rectangles on foundation surface:
  \begin{enumerate}
    \item Define a work plane \SI{3}{\metre} above zero. This allows us to place two-dimensional objects on the surface of or inside a three-dimensional object.
    \item On the work plane create a \SI{5}{\metre} by \SI{1}{\centi\metre} rectangle with its base at $(x, y) = (\SI{0}{\metre},\SI{5}{\metre} - \SI{1}{\centi\metre})$. This represents one side of the perimeter crack.
    \item Copy the rectangle and rotate it \SI{90}{\degree} around the corner of the foundation, i.e. $(x, y) = (\SI{5}{\metre} - \SI{0.5}{\centi\metre},\SI{5}{\metre} - \SI{0.5}{\centi\metre})$.
    \item Join the two rectangles to create a unified perimeter foundation crack.
  \end{enumerate}
  Now the geometry of this VI scenario is complete.\par

  \section{Properties}
  % TODO: Make sure gravel density data is correct
  % TODO: Flip the table?
  \begin{table}
    \centering
    \caption{Properties and van Genuchten parameters of select soil types\cite{abreu_conceptual_2012}.}
    \label{tbl:soils}
  \begin{tabular}{c c c c c c c}
    \toprule
    \multirow{2}{*}{Soil type} & Permeability & Density & Porosity & Residual moisture & \multicolumn{2}{c}{van Genuchten parameters} \\
    & $\kappa \; \mathrm{(m^2)}$ & $\rho \; \mathrm{(kg/m^3)}$ & $\theta_t$ & $\theta_r$ & $\alpha$ & $m$ \\
    \hline
    Sand & \num{9.9e-12} & 1430 & 0.38 & \num{5.3e-2} & 3.5 & 3.2 \\
    Loamy sand  & \num{1.6e-12} & 1430 & 0.39 & \num{4.9e-2} & 3.5 & 1.7 \\
    Sandy loam  & \num{5.9e-13}  & 1460 & 0.39 & \num{3.9e-2} & 2.7 & 1.4 \\
    Sandy clay loam  & \num{2.0e-13} & 1430 & 0.38 & \num{6.3e-2} & 2.1 & 1.3 \\
    Loam  & \num{1.9e-13}& 1380 & 0.40 & \num{6.1e-2} & 1.5 & 1.5 \\
    Silt loam  & \num{2.8e-13} & 1380 & 0.44 & \num{6.5e-2} & 0.51 & 1.7 \\
    Clay loam  & \num{1.3e-13}  & 1500 & 0.44 & \num{7.9e-2} & 1.6 & 1.4 \\
    Silty clay loam & \num{1.7e-13} & 1390 & 0.48 & \num{9.0e-2} & 0.84 & 1.5 \\
    Silty clay  & \num{1.5e-13} & 1300 & 0.48 & \num{1.1e-1} & 1.6 & 1.3 \\
    Silt  & \num{6.7e-13} & 1260 & 0.49 & \num{5.0e-2} & 0.66 & 1.7 \\
    Sandy clay  & \num{1.7e-13} & 1470 & 0.39 & \num{1.2e-1} & 3.3 & 1.2 \\
    Clay  & \num{2.3e-13} & 1330 & 0.46 & \num{9.8e-2} & 1.3 & 1.3 \\
    Gravel\cite{dan_capillary_2012} & \num{1.3e-9} & 1430 & 0.42 & \num{5.0e-3} & 100 & 2.19 \\
    \bottomrule
  \end{tabular}
  \end{table}
\end{appendices}


No models are true representations of reality, but some of them may be useful.
Ever since Newton first wrote his laws of motion, mankind has tried to describe reality with an ever increasing number of mathematical statements.
With the advent of computation and advancements in numerical methods our capabilities to mathematically describe physical systems has dramatically increased.
Even so, real-world systems are too complex to be fully modeled, but mathematical representations may be used to approximate and reveal useful insights of how they function.\par

This is especially true for vapor intrusion (VI) models.
Often it is impossible or difficult to conduct controlled studies of VI sites making models an important tool for understanding these sites and the VI phenomena.
The previous chapter is proof of this as it is readily apparent that a multitude of VI models of varying complexity have been developed over the years, and has become an integral part of the scientific VI community.
From the simple Johnson \& Ettinger one-dimensional model to full three-dimensional finite element models we see that the increased complexity of the model allowed for a greater number of VI topics and phenomena to be explored.\par

The development of the three-dimensional finite element models begin with a conceptual site model (CSM) of a VI site.
In general when one develops models, it is best in the beginning to keep the model as simple as possible, and not to add overly complex features or excessive physics.


% TODO: Integrate or remove paragraphs below
To model vapor intrusion (VI), a number of partial differential equations (PDEs) which describe the relevant physics must be solved.
Many of these PDEs are also coupled in implicit and explicit ways, e.g. the PDE describing vapor flow in the soil affect the contaminant concentration, which in turn affect the indoor air concentration, which in turn also affects the contaminant concentration in the soil.

The indoor air space is perhaps the most important part of modeling VI, as the goal of these models ultimately is to predict indoor exposure given external factors.
One could therefore assume that most of the effort in modeling VI should be spent to accurately represent the interior.
This would be very impractical however, as building interiors are so diverse.
Even if one would spend the time to model an interior, this would dramatically increase the number of mesh elements required to solve the model.
Additionally, the air flow inside the building must be calculated, and even using a simplified version of Navier-Stokes, like Reynolds Averaged Navier-Stokes, the computational cost would be significant.
For these reasons, the indoor air space is simply modeled as a continuously stirred tank (CST), and paradoxically becomes the simplest component of the VI model.\par

Most of the effort will be spent to accurately model the physics in the soil underlying the building of interest.

\chapter{Geometry Generation}

To create our quarter geometry, only a few simple geometric objects and Boolean operations are required: two cuboids, two rectangles, one Boolean difference operation, and one Boolean join operation.
Figure \ref{fig:geometry} shows the resulting geometry.
Note that $z = \SI{0}{\metre}$ is the groundwater/soil interface and the plane of symmetry is around the $(x, y) = (\SI{0}{\metre},\SI{0}{\metre})$ axis\par

To create the soil surrounding the building using the COMSOL geometry generator:
\begin{enumerate}
  \item Create a \SI{15}{\metre} by \SI{15}{\metre} by \SI{4}{\metre} block with its base at $(x, y, z) = (\SI{0}{\metre},\SI{0}{\metre},\SI{0}{\metre})$. This is the entire soil domain.
  \item Create a \SI{5}{\metre} by \SI{5}{\metre} by \SI{1}{\metre} block with its base at $(x, y, z) = (\SI{0}{\metre},\SI{0}{\metre},\SI{3}{\metre})$. This will represent the volume that the house take up in the soil, i.e. the underground portion of the basement.
  \item Perform a difference operation, removing the "basement" block from the "soil" block.
\end{enumerate}
At this point you will see that a quarter soil domain has been created, with an empty space that represents a house with a foundation slab located \SI{1}{\metre} bgs.\par

The foundation crack will be modeled by joining two  \SI{1}{\centi\metre} wide strip that spans the perimeter of the surface that represents the house foundation.
This strip is created by joining two rectangles on foundation surface:
\begin{enumerate}
  \item Define a work plane \SI{3}{\metre} above zero. This allows us to place two-dimensional objects on the surface of or inside a three-dimensional object.
  \item On the work plane create a \SI{5}{\metre} by \SI{1}{\centi\metre} rectangle with its base at $(x, y) = (\SI{0}{\metre},\SI{5}{\metre} - \SI{1}{\centi\metre})$. This represents one side of the perimeter crack.
  \item Copy the rectangle and rotate it \SI{90}{\degree} around the corner of the foundation, i.e. $(x, y) = (\SI{5}{\metre} - \SI{0.5}{\centi\metre},\SI{5}{\metre} - \SI{0.5}{\centi\metre})$.
  \item Join the two rectangles to create a unified perimeter foundation crack.
\end{enumerate}
Now the geometry of this VI scenario is complete.\par

% Physics section and rest subsections?
\section{Vapor Intrusion Physics and Governing Equations}

Throughout this section the books \textit{Contaminant hydrogeology}\cite{fetter_contaminant_1993} by Fetter, and \textit{Dynamics of fluids in porous media}\cite{bear_dynamics_1972} by Bear are used as sources, especially on topics regarding soil physics and contaminant transport in soils.
The COMSOL reference manuals\cite{comsol_comsol_nodate,comsol_subsurface_nodate} are also used as sources.

% TODO: Maybe change c_g -> c_{g,ck} ?

\subsection{The Indoor Environment}\label{sec:indoor}

The impacts on the indoor air space is perhaps the most important part of modeling VI, as the goal of these models ultimately is to predict indoor exposure given some external factors.
The indoor environment is, however, only modeled implicitly as a continuously stirred tank reactor (CSTR).\par
We assume that all contaminant entry into the house occurs via the foundation crack.
Once the contaminant enters the interior, it is instantly perfectly mixed, which is a key assumption of a CSTR.
Contaminant expulsion occurs via air exchange with the outdoor environment, and is regulated by \textit{air exchange rate} $A_e$, which dictates what fraction of the indoor air is exchanged with outdoor air over a given period of time.
For instance, a common air exchange rate for a house is \SI{0.5}{\per\hour}, i.e. half of the indoor air is exchanged every hour.
It should be noted that in this simple VI model implementation, we assume that there are no indoor sources nor that any sorption of contaminant to/from any indoor materials occurs.
Thus, the reaction term that would ordinarily be part of a CSTR is dropped (but is reintroduced in Chapter (TBD)) and the change in indoor contaminant concentration is thus given by \eqref{eq:cstr}. % TODO: Add chapter reference
\begin{equation}\label{eq:cstr}
  V_\mathrm{bldg}\frac{\partial c_\mathrm{in}}{\partial t} = n_\mathrm{ck} - V_\mathrm{bldg} A_e c_\mathrm{in}
\end{equation}
Here $c_\mathrm{in}$ [\si{\mol\per\metre\cubed}] is the indoor air contaminant concentration;
$n_\mathrm{ck}$ [\si{\mol\per\second}] is the contaminant entry (or exit) rate into the building via the foundation crack;
$A_e = \SI{0.5}{\per\hour}$ is the air exchange rate;
Finally, $V_\mathrm{bldg} = \SI{300}{\metre\cubed}$ is the volume of the house interior (or basement in this case).\par

A limitation of this approach is that we only consider one control volume or compartment, while in reality indoor contaminant concentrations can vary significantly between compartments, in particular between different floors.
There are VI models that use multiple compartments, which in essence are just coupled CSTRs\cite{murphy_multi-compartment_2011}.
Basements typically have higher indoor contaminant concentrations than other floors, so in this implementation we assume that our sole compartment is the house basement, which $V_\mathrm{bldg} = \SI{300}{\metre\cubed}$ reflects.\par

Solving \eqref{eq:cstr} requires us to determine the contaminant entry and air exchange rates.
Air exchange rates can vary quite significantly, and are a significant source of temporal variability in VI, a topic that will be further explored in Chapter (TBD). % TODO: Reference chapter
However, they typically vary around relatively well-known values as air exchange rates are regulated in building codes.
For residential buildings, it is typical that air exchange rate is around $A_e = \SI{0.5}{\per\hour}$ and thus for simplicity we will choose this value.\par

\paragraph{Contaminant entry into the building}

Contaminant entry rates are significantly more difficult to determine, as they depend on air velocity through the foundation breach and the concentration gradient across it.
The determination of these is the main point, and challenge in VI modeling.\par

% TODO Add figure showing schematic of the entry into the building

The contaminant entry $n_\mathrm{ck}$ is given by integrating the contaminant entry flux $j_\mathrm{ck}$ across the foundation crack boundary $A_\mathrm{ck}$.
\begin{equation}
  n_\mathrm{ck} = \int_{A_\mathrm{ck}} j_\mathrm{ck} dA
\end{equation}
The contaminant flux through the foundation crack is modeled as transport between two parallel plates and has an advective and a diffusive component.
\begin{equation}
  j_\mathrm{ck} = j_\mathrm{advection} + j_\mathrm{diffusion}
\end{equation}
Since contaminant concentration indoors is lower than it is in the soil or near foundation crack region a concentration gradient from the soil-gas to the indoor will exist.
The interior of the crack is not explicitly modeled, but assumed to only contain air and thus we assume the diffusion coefficient is the same as in air.
\begin{equation}
  j_\mathrm{diffusion} = - \frac{D_\mathrm{air}}{L_\mathrm{slab}} (c_{in} - c_g)
\end{equation}
here $D_\mathrm{air} = \SI{7.2e-6}{\metre\squared\per\second}$ is the diffusion coefficient of TCE in air as a sample contaminant of interest; other contaminant of common concern have comparable diffusivities. % TODO: Make sure this is right
$L_\mathrm{slab} = \SI{15}{\centi\metre}$ is the thickness of the foundation slab;
$c_{in}$ [\si{\mol\per\metre\cubed}] is the indoor contaminant concentration;
$c_g$ [\si{\mol\per\metre\cubed}] is the contaminant gas-phase concentration at the foundation crack boundary.\par

Advective transport through the slab can occur in both directions, i.e. contaminants can be carried from the soil into the house and from the house into the soil\cite{holton_creation_2018}.
The direction of this transport depend on the direction of the flow, with a positive sign indicating that airflow goes into the house.
\begin{equation}
  j_{advection} = \begin{cases}
    u_{ck} c_g & u_{ck} \geq 0 \\
    u_{ck} c_{in} & u_{ck} < 0
\end{cases}
\end{equation}
here $u_{ck}$ [\si{\metre\per\second}] is the airflow velocity through the foundation crack.

Thus the total contaminant transport through the foundation crack is given by \eqref{eq:contaminant_entry}.
\begin{equation}\label{eq:contaminant_entry}
  j_{ck} = \begin{cases}
    u_{ck} c_g - \frac{D_\mathrm{air}}{L_\mathrm{slab}} (c_{in} - c_g) & u_{ck} \geq 0 \\
    u_{ck} c_{in} - \frac{D_\mathrm{air}}{L_\mathrm{slab}} (c_{in} - c_g) & u_{ck} < 0
\end{cases}
\end{equation}
Not only will \eqref{eq:contaminant_entry} be used to calculate the contaminant entry rate into house, but it is a necessary boundary condition for calculating the contaminant concentration in the soil.
However, as we see, \eqref{eq:contaminant_entry} is a function of both the soil-gas concentration at the foundation crack boundary $c_g$ and the indoor contaminant concentration $c{in}$, thus these two are coupled and need to be solved simultaneously.\par

\subsection{The Importance Of Soil Moisture Content}\label{sec:van_genuchten}

The portion of soil between groundwater and ground surface is variably saturated with water and is called the vadose zone.
Under environmental conditions, TCE and many other contaminants are miscible in water, and will be partitioned between water and vapor phases, which has profound effects on the total contaminant transport in the vadose zone.
The rates of diffusion in liquids and gases usually differ by orders of magnitudes.
Likewise, advective transport in the two phases occurs at vastly different rates, and as such it is important to account for the effects of soil moisture content on transport of contaminant.\par

Water filling of soil pores from the groundwater surface is driven by a negative pressure gradient induced by surface tension, called capillary potential and is here represented by $\psi$.
This capillary potential is a function of the soil moisture content, and becomes increasingly negative as the water content decreases, and is zero when the soil matrix is saturated with water.
The capillary potential varies with the hydraulic properties of specific soil types.\par

In addition to the capillary potential, soil moisture content is driven by a gravitational potential, e.g. induced groundwater flow due to some elevation gradient.
The total soil moisture potential $\phi$, is the sum of the capillary and gravitational potentials, here expressed as a pressure head.
\begin{equation}
  \phi = \frac{\psi(\theta_w)}{\rho g} + h_\mathrm{g} = h + h_\mathrm{g}
\end{equation}
where $\phi$ [\si{\metre}] is the total soil moisture potential;
$\psi$ [\si{\pascal}] is the capillary potential;
$h$ [\si{\metre}] is the capillary potential expressed as a pressure head above the groundwater/soil interface;
$\theta_w$ is the volumetric moisture content by volume soil, i.e. dimensionless;
$\rho$ [\si{\kilo\gram\per\metre\cubed}] is the density of water;
$g$ [\si{\metre\per\second\squared}] is the acceleration due to gravity;
and $h_\mathrm{g}$ [\si{\metre}] is the gravitational potential above a reference plane.\par

In the VI scenario considered here, we assume that the groundwater is at steady-state and its soil interface is flat\footnote{
If these assumptions cannot be made and one wishes to model the groundwater and vadose zones as a continuum, the soil moisture content is determined by Richard's equation.
}.
Thus, the soil moisture content is entirely determined by the capillary potential $h$.\par

There are two common methods for modeling the capillary potential, one developed by \citeauthor{brooks_properties_1966}\cite{brooks_properties_1966} in 1966, and another by \citeauthor{van_genuchten_closed-form_1980}\cite{van_genuchten_closed-form_1980} in 1980.
Both of these are semi-empirical approaches and relies on experimentally determined parameters for a specific soil type to be used.
In this work, we only use van Genuchten's method, simply because parameters for a wide variety of soils have been made available by the EPA (Table \ref{tbl:soils}).\par

\subsubsection{van Genuchten's Soil-Water Retention Model}

The relationship between capillary pressure and moisture content is called \textit{soil moisture retention}, and is what van Genuchten's method describes.
Specifically, his method models the water saturation of the soil and is given by \eqref{eq:van_genuchten_saturation}.
\begin{equation}\label{eq:van_genuchten_saturation}
  \mathrm{Se} =
    \begin{cases}
      \frac{1}{(1 + (\alpha |h|)^n)^m} & h < 0 \\
    1 & h \geq 0
    \end{cases}
\end{equation}
here $\mathrm{Se}$ is the saturation, which ranges from 0 to 1, where 1 is fully saturated with water;
$\alpha$, $m$, and $n=\frac{1}{1-m}$ are the empirically determined van Genuchten parameters given in Table \ref{tbl:soils};
and $h$ [\si{\metre}] is the capillary pressure head as elevation above the groundwater/soil interface.\par

It is important to note that $\mathrm{Se} = 0$ does not mean that there is no moisture in the soil; soils retain a small amount of water in the matrix - residual moisture content (which is soil specific).
Thus, the soil moisture content is given by \eqref{eq:van_genuchten_soil_moisture}
\begin{equation}\label{eq:van_genuchten_soil_moisture}
  \theta_w =
    \begin{cases}
      \theta_r + \mathrm{Se}(\theta_t - \theta_r) & h < 0 \\
      \theta_t & h \geq 0
    \end{cases}
\end{equation}
where $\theta_w$ is the volumetric soil moisture content;
$\theta_t$ is the soil porosity;
and $\theta_r$ is the residual moisture content.\par

By extension, the soil gas or air content is given by
\begin{equation}\label{eq:gas_porosity}
  \theta_g = \theta_t - \theta_w
\end{equation}
An example of the soil saturation and moisture content as a function of height above groundwater is shown in Figure \ref{fig:retention_curve}.
Note the steep decline in moisture content near the groundwater interface - that is the capillary zone and we will see that this zone presents a significant barrier to contaminant transport from the groundwater in section \ref{sec:transport}.\par

\begin{figure}[htb!]
  \centering
  \begin{subfigure}{\textwidth}
    \centering
    \includegraphics[width=0.75\textwidth]{van_genuchten.pdf}
    \caption{Example of a soil moisture retention curve as a function of pressure head above the groundwater/soil interface.}
    \label{fig:retention_curve}
  \end{subfigure}
  \begin{subfigure}{\textwidth}
    \centering
    \includegraphics[width=0.75\textwidth]{relative_permeability.pdf}
    \caption{Relative air and water permeability in the vadose zone or above the groundwater interface.}
    \label{fig:relative_permeability}
  \end{subfigure}
  \caption[van Genuchten soil moisture retention curve and relative permeability]{The van Genuchten soil moisture retention curve and relative permeability of soil to air and water transport.}
\end{figure}

The presence of water in the soil matrix has profound implications for transport as the pore space available for transport may be restricted.
For instance, in the capillary zone, contaminant transport is mostly limited by the water phase, while gas phase transport is extremely limited because air-filled porosity is limited and largely isolated.
The opposite is true near the ground surface, where most of the pore space is filled with air.\par

Soil already limits transport, i.e. it is harder to pump water through a soil column than a pipe of the same diameter.
This extra phase-specific transport inhibition is modeled by a \textit{relative permability} and is given by \eqref{eq:van_genuchten_relative_permeability}.
\begin{equation}\label{eq:van_genuchten_relative_permeability}
  k_r =
    \begin{cases}
      (1-\mathrm{Se})^l \big(1 - \mathrm{Se}^\frac{1}{m} \big)^{2m} & h < 0 \\
      0 & h \geq 0
    \end{cases}
\end{equation}
here $k_r$ is the relative permeability for air;
and $l = 0.5$ is another van Genuchten parameter.\par % TODO: How do I motivate this?

The relative permeability varies from 0 to 1, where 0 indicates that the soil matrix is completely impermeable to the fluid, while a value of 1 means that there is no additional permeability cost.
Figure \ref{fig:relative_permeability} shows how the relative gas permeability varies in the vadose zone.\par

\subsection{Airflow In The Vadose Zone}\label{sec:darcys_law}

In our VI scenario, the depressurized house induces an advective airflow from the ground surface, through the vadose zone, and into the house via the foundation crack, carrying contaminant vapors with it.
This airflow is modeled using a modified version of Darcy's Law.
The modification is made to account for the variable moisture content in the vadose zone, which is established in section \ref{sec:van_genuchten}.\par

Darcy's Law describes the flow of a fluid through a porous medium.
This flow is driven by a pressure gradient, and its magnitude depends on the permeability of the porous medium and the fluids viscosity.
\begin{equation}\label{eq:darcys_law_saturated}
  \vec{u} = -\frac{\kappa}{\mu} \nabla p
\end{equation}
here $\vec{u}$ [\si{\m\per\second}] is the airflow velocity vector;
$\kappa$ [\si{\metre\squared}] is the permeability of the porous media;
$\mu$ [\si{\pascal\second}] is the dynamic viscosity of the fluid;
and $\nabla$ [\si{\pascal\per\metre}] is the pressure gradient.\par

This \eqref{eq:darcys_law_saturated} formulation of Darcy's Law assumes that the porous media is saturated with the fluid,\footnote{Darcy's Law also assumes that the flow is in the laminar regime, i.e. the Reynolds number $\mathrm{Re} < 1$.
Due to the small pressure gradients in most VI scenarios, this assumption is rarely unfulfilled, but if it is, then Brinkman's equation should be used instead.} % TODO: Source for Brinkmans equation
hence the need to modify this expression.
While porosity is not directly part of \eqref{eq:darcys_law_saturated}, it is an intrinsic property of the permeability $\kappa$ of the porous media; the variably saturated pores variably changes the permeability.
This variation in permeability is modeled using the relative permeability expression from van Genuchten's equation \eqref{eq:van_genuchten_relative_permeability}.
The effective soil permeability is the product of the soil permeability and its relative permeability giving our modified Darcy's Law \eqref{eq:darcys_law_unsaturated}.
\begin{equation}\label{eq:darcys_law_unsaturated}
  \vec{u} = -\frac{(1-k_r)\kappa}{\mu} \nabla p
\end{equation}
Recall that by definition $k_r$ is the relative permeability of the soil to \textit{water} and thus $1-k_r$ is that to \textit{gas}.\par

To calculate the soil-gas velocity field in the vadose zone, we need a continuity equation, which for fluid flow is \eqref{eq:fluid_continuity}.
\begin{equation}\label{eq:fluid_continuity}
  \frac{\partial \rho}{\partial t} + \nabla \cdot (\rho \vec{u}) = 0
\end{equation}
Inserting our modified Darcy's Law for the velocity gives \eqref{eq:vapor_transport}.
\begin{equation}\label{eq:vapor_transport}
  \frac{\partial}{\partial t} (\rho \theta_g) + \nabla \cdot \rho \Big( -\frac{(1-k_r) \kappa}{\mu} \nabla p \Big) = 0
\end{equation}
where $\theta_g$ is the gas-filled porosity of the soil from \eqref{eq:gas_porosity};
$\rho = \SI{1.225}{\kilogram\per\metre\cubed}$ is the density of air;
and $\mu = \SI{18.5e-6}{\pascal\second}$ is the dynamic viscosity of air.
Contaminant vapor concentrations are typically highly dilute in VI scenarios, and therefore we assume that the contaminant does not affect the transport properties of air.\par

In order to solve \eqref{eq:vapor_transport} we need to define some boundary conditions.
In our CSM, air is pulled from the atmosphere through the ground surface and into the building via the foundation crack.
To model this only three boundary conditions are required.\par

\paragraph{Boundary Conditions}

The first is to define a pressure gauge, i.e. a reference point for where the pressure is zero, which is where air will be pulled from.
This is the applied to the ground surface boundary.
The second is that we apply the indoor/outdoor pressure difference (-5 Pa) to the foundation crack boundary.
The third type is applied to all remaining boundaries and is a no flow boundary condition, indicating that no flow passes through these boundaries.
We also make sure that we specify the symmetry planes present.
\begin{align}
  &\text{Ground surface} &p = 0 \; \mathrm{(Pa)} \\
  &\text{Foundation crack} &p = p_\mathrm{in/out} = -5 \; \mathrm{(Pa)} \\
  &\text{Remaining} &-\vec{n}\cdot\rho\vec{u} = 0
\end{align}
where $\vec{n}$ is the boundary normal vector.\par

\paragraph{Initial Conditions}

For steady-state problems, the initial conditions do not influence the solution.
Transient simulations however, require initial conditions and these are assumed to be given by the steady-state solution.\par

\paragraph{Basis function}

A hat function is used as the basis function for solving Darcy's Law in FEM.\par

\subsection{Mass Transport In The Vadose Zone}\label{sec:transport}

To begin deriving a governing equation for contaminant transport in our VI scenario, we consider the continuity equation which states that the change of concentration in some volume of space depends on the advective and diffusive fluxes in and out of the system, as well as any generation or consumption inside the system.
\begin{equation}
  \frac{\partial c}{\partial t} + \nabla \cdot(j_\mathrm{adv} + j_\mathrm{diff}) - G = 0
\end{equation}
here $c$ [\si{\mol\per\metre\cubed}] is the concentration of the chemical species;
$t$ [\si{\second}] is time;
$j_\mathrm{adv}$ and $j_\mathrm{diff}$ [\si{\mol\per\second\per\metre\squared}] are the advective and diffusive fluxes respectively;
and $G$ [\si{\mol\per\second}] is the generation or consumption of the chemical species.\par

In our model we will assume that $G = 0$ as the groundwater is the sole contaminant source and TCE does not readily degrade in soils. % TODO: Source for TCE degradation
However, this term should remain and an appropriate expression developed if one wants to model:
\begin{itemize}
  \item Biodegradation of some compound in the soil.
  \item Radon intrusion (remember radon gas is generated in soils and rocks).
  \item A soil or subsurface source, e.g. a leaky tank or evaporation from a (pure) contaminant spill.
\end{itemize}\par

The advective flux is given by
\begin{equation}
  j_\mathrm{adv} = \vec{u} c
\end{equation}
where $\vec{u}$ [\si{\metre\per\second}] is a velocity vector.
The diffusive flux is given by Fick's Law
\begin{equation}
  j_\mathrm{diff} = -D \nabla c
\end{equation}
where $D$ [\si{\metre\squared\per\second}] is the diffusion coefficient of the contaminant in the solute;
and $\nabla c$ [\si{\mol\per\metre\cubed\per\metre}] is a concentration gradient.
Thus we get the advection-diffusion equation which generally governs transport of a chemical species
\begin{equation}
  \frac{\partial c}{\partial t} + \nabla \cdot(\vec{u} c + -D \nabla c) = 0
\end{equation}
However, this will not accurately represent contaminant transport in the vadose zone due to
\begin{itemize}
  \item Contaminant transport occurs inside a variably saturated porous matrix, significantly affecting transport properties.
  \item The contaminant concentration in the vadose zone will be distributed between three phases - gas, water, and solid (via sorption).
\end{itemize}\par

The total contaminant concentration in the soil will be used in lieu of just concentration, i.e. $c \rightarrow c_T$ and thus the total contaminant concentration is the sum of the gas, water, and solid phase concentrations.
\begin{equation}
  c_T = \theta_w c_w + \theta_g c_g + c_s \rho_b
\end{equation}
Here $\theta_g$ and $\theta_w$ are the gas-filled and water-filled porosities respectively;
$c_w$ and $c_g$ [\si{\mol\per\metre\cubed}] are the contaminant concentrations in water and gas respectively;
$c_s$ [\si{\mol\per\kilogram}] is the solid phase or sorbed concentration per mass of soil;
and $\rho_b = (1-\theta_t) \rho$ [\si{\kilogram\per\metre\cubed}] is the bulk density of the soil, which can be calculated from the soil porosity $\theta_t$ and solid phase density of the soil $\rho$ [\si{\kilogram\per\metre\cubed}].\par

The attentive reader will now notice that our governing equation depend on three variables instead of one.
However, remember that we're concerned with low contaminant concentrations, we can relate the gas and liquid phase concentrations via Henry's Law \eqref{eq:henrys_law}
\begin{equation}\label{eq:henrys_law}
  c_g = K_H c_w
\end{equation}
where $K_H = 0.402$ is the dimensionless Henry's Law constant for TCE at \SI{20}{\degreeCelsius}. % TODO: TCE K_H source
We also assume that there are no temperature gradients throughout the vadose zone.\par

The solid phase concentration can be related to the others via a linear sorption isotherm.
Here either the gas-solid or water-solid sorption interaction can be chosen; the former is used in Chapter (TBD) to we will explore effect of gas-solid sorption. % TODO: Chapter reference
\begin{equation}
  c_s = \begin{cases}
    K_p c_w & \text{Water-solid sorption} \\
    K_p c_g = K_p K_H c_w & \text{Gas-solid sorption}
\end{cases}
\end{equation}
here $K_p$ [\si{\metre\cubed\per\kilogram}] is a sorption partitioning coefficient.\par

Another approach is to simply ignore the role of sorption completely, i.e. $K_p = 0$, which has historically been done in VI modeling and is done in this example too.
The reason for this is two-fold.
\begin{enumerate}
  \item Relevant sorption data has not been available.
  \item With an infinite source assumption, and at steady-state, sorption doesn't affect the solution; these have been common assumptions in most VI models so far.
\end{enumerate}
Regardless, we will continue with the sorption $K_p$ term, because this will become relevant in Chapter (TBD) where experimentally derived relevant sorption data is available.\par % TODO: Chapter reference

With Henry's Law and the linear sorption assumption we can relate the total contaminant concentration in the soil matrix to the water-phase contaminant concentration.
\begin{equation}
  c_T = (\theta_w + \theta_g K_H + K_H K_p \rho_b) c_w
\end{equation}
The terms in front of $c_w$ are collected as $R = (\theta_w + \theta_g K_H + K_H K_p \rho_b)$.
This term is called the \textit{retardation factor} and reflects the increased contaminant residence time in the matrix due to shifting between the phases, and becomes an important factor in transient transport simulations.\par

In the vadose zone, advective transport can occur in both the water and gas phases inside the soil pores.
\begin{equation}
  j_\mathrm{adv} = \vec{u}_{w,pore} c_w \theta_w + \vec{u}_{g,pore} c_g \theta_g
\end{equation}
here $\vec{u}_{w,pore}$ and $\vec{u}_{g,pore}$ [\si{\metre\per\second}] are the water and gas phase \textit{pore} velocities vectors respectively.
However, from mass conservation, we know that the product of the pore velocity and porosity gives the superficial velocity of a fluid in porous media, i.e. the Darcy's Law velocities.
This together with Henry's Law gives
\begin{equation}
  j_\mathrm{adv} = (\vec{u}_w + \vec{u}_g K_H ) c_w
\end{equation}
and here $\vec{u}_w$ + $\vec{u}_g$ [\si{\metre\per\second}] are the Darcy or superficial velocity vectors.\par

In section \ref{sec:van_genuchten} we assumed that there is no gravitational water potential in the soil matrix, and it follows that the water in the pores is stationary, i.e. $\vec{u}_w = 0$ giving
\begin{equation}
  j_\mathrm{adv} = \vec{u}_g K_H c_w
\end{equation}
where $\vec{u}_g$ is the solution from solving Darcy's Law in section \ref{sec:darcys_law}.\par

To model a scenario where there is a gravitational water potential, one would have to solve two-phase Darcy's Law to get both $\vec{u}_g$ and $\vec{u}_w$.
This significantly complicates the mass transport aspect as well, and as such is beyond the scope of this work.\par

The diffusive transport expression likewise needs to be adjusted, and the total diffusive flux through the pore matrix is given by
\begin{equation}
  j_\mathrm{diff} = -(D_w \theta_w \tau_w \nabla c_w + D_g \theta_g \tau_g \nabla c_g)
\end{equation}
here $D_w = \SI{1.02e-9}{\metre\squared\per\second}$ and $D_g = \SI{6.87e-6}{\metre\squared\per\second}$ are the contaminant diffusion coefficients of TCE in pure water and gas respectively; % TODO: Add diffusion values here?
and $\tau_w$ and $\tau_g$ are the water and gas tortuosity terms.\par

Due to the irregular shapes of pores, diffusion of a chemical species will inevitably often occur along a tortuous path, which the tortuosity terms attempt to capture.
As tortuosity depend on the structure of the porous matrix, it is difficult to accurately portray, but a popular approach is to use Millington \& Quirks model\cite{millington_permeability_1961}.
\begin{equation}\label{eq:millington-quirk}
  \tau_w = \frac{\theta_w^{\frac{7}{3}}}{\theta_t^2}, \; \tau_g = \frac{\theta_g^{\frac{7}{3}}}{\theta_t^2}
\end{equation}
here $\theta_t$ is the total or saturated porosity of the soil matrix.
Another popular approach is Bruggeman's model, or if possible, a custom version can be used.\par % TODO: Add Bruggeman's reference here

Combining \eqref{eq:millington-quirk} and \eqref{eq:henrys_law} in our diffusion flux expression gives
\begin{equation}
  j_\mathrm{diff} = -\Big(D_w \frac{\theta_w^{\frac{10}{3}}}{\theta_t^2} + D_g \frac{\theta_g^{\frac{10}{3}}}{\theta_t^2} K_H\Big) \nabla c_w
\end{equation}
the terms in front of $\nabla c_w$ can be collected as an effective diffusion coefficient $D_\mathrm{eff}$ [\si{\metre\squared\per\second}], which with our isothermal vadose zone assumption only depends on the soil moisture content.
Thus we get the final diffusive flux expression
\begin{equation}
  j_\mathrm{diff} = - D_\mathrm{eff}\nabla c_w
\end{equation}
Figure \ref{fig:D_eff} shows how the effective diffusivity varies from being close to that of the pure water diffusivity near the capillary zone, and increases to something closer to gas-phase diffusivity as the soil moisture decreases.\par

\begin{figure}
  \includegraphics[width=\textwidth]{effective_diffusivity.pdf}
  \caption{Effective diffusivity of TCE in the vadose zone using Millington-Quirks model. Soil water and gas filled porosites are calculated using van Genuchten's equations.}
  \label{fig:D_eff}
\end{figure}

Putting all this together finally gives us the governing equation for contaminant transport in the vadose zone for our modeled VI scenario.
\begin{equation}\label{eq:mass_transport}
  R \frac{\partial c_w}{\partial t} = \nabla \cdot (D_\mathrm{eff} \nabla c_w) - K_H \vec{u}_g \cdot \nabla c_w
\end{equation}
To solve this we need to define some boundary and initial conditions.\par

% TODO: Rewrite these in the form c(t, vec{x}) = ..., i.e. c(t, atm) = 0, c(t, source) = c_gw etc

\paragraph{Boundary Conditions}

In this VI scenario, the sole contaminant source is assumed to be the homogenously contaminated groundwater, which we assume to have a fixed concentration.
The atmosphere acts as a contaminant sink and thus this is simply a zero concentration boundary condition.
Contaminants leave the soil domain and enter the building through a combination of advective and diffusive gas phase transport.
The boundary condition applied to all other boundaries is a no-flow boundary.
\begin{align}
  &\text{Atmosphere} & c_w = \SI{0}{\mol\per\metre\cubed} \\
  &\text{Groundwater} & c_w = c_{gw} = \SI{0.1}{\mol\per\metre\cubed} \\
  &\text{Foundation crack} & -\vec{n} \cdot \vec{N} = \frac{-j_{ck}}{K_H} \; \si{\mol\per\metre\squared\per\second}\\
  &\text{All other} & -\vec{n} \cdot \vec{N} = \SI{0}{\mol\per\metre\squared\per\second}
\end{align}
$\vec{n} \cdot \vec{N}$ is the dot product between the boundary normal vector and the contaminant flux;
$j_ck$ is the contaminant vapor flux into the building.
We assume that only contaminants in the gas phase enter the building, and dividing $j_{ck}$ by $K_H$ we get proper accounting in terms of the water phase concentration accounting in the main transport equation \ref{eq:mass_transport}.\par

\paragraph{Initial Conditions}

For steady-state problems, the initial conditions do not influence the solution.
Transient simulations however, require initial conditions and these are assumed to be given by the steady-state solution.\par

\paragraph{Basis function}

A second order polynomial (quadratic) function is used as the basis function for solving the transport equation.\par

\section{Meshing}\label{sec:meshing}

A mesh is a collection of small discrete elements that in combination form a larger geometry or domain.
Meshing is the process of generating a mesh.
Meshing is perhaps one of the most important and challenging aspects of solving a FEM model and a well-constructed mesh is necessary for accurate and reliable results.\par

In theory, an infinitely fine mesh will give the analytical solution to a PDE but obviously the computational costs would be infinite then as well; one must always balance the accuracy of the solution and computational resources.
This balancing act is somewhat of an art and there are no easily defined rights or wrongs.
However, there are general guidelines that are useful to keep in mind while meshing.
But before we get into those it is worth to spend some more time on what a mesh is.\par

The most fundamental unit of the mesh is the element(s) that comprise the mesh.
There are many different types of elements that can be used for meshing and choosing which ones to use depend primarily on the spatial dimensionality of the model, the particularities of the geometry, and the physics that we wish to model.
Obviously different element types are by necessity needed to model a 2D vs. 3D geometry; you cannot mesh a 3D geometry with 2D squares.
This distinction is not very interesting and any lesson learnt about meshing in one of these dimensions is easily generalizable to the other.
Thus, we will exclusively discuss the meshing of 3D elements.\par

There are primarily four types of 3D mesh elements available - the tetrahedral, cuboid, prism, and pyramid - see Figure \ref{fig:3d_elements}.
These can be combined in various ways to represent any 3D geometry.
The most general out of these is the tetrahedral and will approximate any geometry well.
It is not always the most effective choice for meshing a geometry and another element type may be better suited.
This is easiest illustrated with an example.\par

Imagine that you are trying to simulate the laminar flow of some fluid through a pipe and been clever enough to realize that by virtue of symmetry only a wedge of the pipe is necessary to be explicitly modeled.
We also realize that the flow through the pipe is going to primarily have a gradient in the direction of the flow.
In this scenario, it might be beneficial to use prism elements rather than tetrahedrals.
Furthermore we could also primarily make the mesh fine in the flow direction while keeping it relatively coarse in other directions.
This would allow us to achieve a solution of high accuracy while still keeping the number of elements relatively small.\par


% TODO: Add figure
\begin{figure}
  %\includegraphics{}
  \caption{Four common mesh elements used to mesh three-dimensional geometries.}
  \label{fig:3d_elements}
\end{figure}




\subsection{Mesh Study}

\section{Solver Configuration}

A solver(s) is required to solve the VI model, and a few considerations need to be taken when choosing one.
For simplicity we will now first consider a stationary or steady-state problem.
Since our model is a multiphysics problem, i.e. many of the physics depend on each other, we first need consider how to couple our physics.
The physics can be coupled by either using a \textit{segregated} or \textit{fully coupled} approach.\par

\paragraph{Segregated vs. fully coupled physics}

In a segregated solver, each governing equation or physics is solved separately in a specific order.
For instance, in our VI example we could solve Darcy's Law first, get some solution, then use that in the transport equation, solve that, and then solve the indoor concentration equation, i.e. we solve one system of equation per physics.
These steps are simply iterated until convergence occurs in all of the separated steps.
The fully coupled approach assembles a single large system of equation from all of the physics.
Both of these approaches will reach the same solution, but the fully coupled approach will do so faster, but at the expense of using more memory.\par

\paragraph{Direct vs. iterative solver}

Within each of these coupling approaches, we need to specify a solver to solve the system of equations.
Here we are again faced with a choice, and we could either use a \textit{direct} or \textit{iterative} solver.
Direct solvers, as the name implies, arrive at a solution directly and are based on LU-decomposition.
Iterative solvers on the other hand, iteratively approach the solution, and are based on conjugate gradient method.
The advantage of direct solvers is that they are faster, but use more memory, while iterative solvers are slower but use less memory.
In terms of choosing a solver algorithm, there are many options, but MUMPS and GMRES will be used as the respective algorithm for direct and iterative solvers.\par

\paragraph{Time-dependent solvers}

To solve a transient or time-dependent problem (which will be done in subsequent chapters) a solver to step forward in time is required.
A too large time step will cause stability issues and ultimately convergence will be impossible, but obviously some discrete time step is required for a solution to be achievable.
A time-dependent solver picks an appropriate time-step and there are some popular approaches, such as using some high-order Runge-Kutta (RK) or backwards differentiation formula (BDF).
Regardless of the type of solver, for each time step the system of equations will be solved using one of the aforementioned solvers.
The difference between RK and BDF is that RK explicitly discretizes time while BDF does so implicitly.
In this work we will only use BDF as it is more stable than RK.\par

\paragraph{Choosing solvers}

The choice of solver will not affect (or should not at least) the solution to the problem.
However, it can have a large impact on computational time and resources, and these considerations dictate solver choice (this is also partially dependent on the mesh used, as this will affect memory usage too).
In this example, and throughout the models used in this work, we will favor speed over memory and therefore fully couple all our equations and use direct solvers.\par

\subsection{Adaptive Mesh Refinement}

The accuracy of the solution obtained by FEM is dependent on the quality of the mesh, something that was discussed in section \ref{sec:meshing}.
While the mesh designer can do much to create a mesh that performs well for the particular problem posed, refinement of the mesh is often needed and should be performed for every new model.\par

There are two types of mesh refinements in FEM.
The first type reduces the size of the elements and thereby the accuracy of the solution, this is called \textit{h-type} refinement ($h$ is often used to denote the mesh size).
The second increases the order of the polynomial of the basis function, called \textit{p-type} refinement which will likewise increases the solution accuracy.\par

h-type refinement is generally more attractive because it is simpler and the computational cost of p-type refinement increase faster than h-type.
However, p-types are useful if the user imports an already existing mesh, and is unable to change it, rendering h-type refinement impossible.
These two method can be combined to perform a \textit{hp-type} refinement.\par

Refinement is usually done by an algorithm, which is possible because FEM has the built-in ability to estimate the local error of the solution anywhere in the domain.
The downside with using an algorithm is that the user has little control over how the mesh is refined.
The user can also manually refine the mesh by solving the model and plot how some relevant metric converges as the mesh is refined.
This can be a very time consuming, and therefore algorithms are usually preferable; a hybrid solution is to manually alter the mesh after the algorithmic mesh refinement.\par

Refinement can either be done locally or globally.
Global refinement involves defining some singular metric that will be used to evaluate the quality of the mesh, e.g. one might use the total stress in a metal bar as a metric here.
In local refinement, one still has to define some metric for evaluating the quality of the refinement, but evaluation only occurs on a subset of the domain, e.g. the stress on just one boundary of the same metal bar.
In both approaches the elements that have the largest estimated local error are refined; this error estimation is an inherent feature of FEM.
The optimal type of refinement varies by problem, but a global refinement will generally be more computationally expensive.\par

In this work we will use a global h-type refinement and use the indoor contaminant concentration $c_{in}$ as our refinement metric.
COMSOLs refinement algorithm has the nice ability to reinitialize the mesh, and can thereby coarsen elements, i.e. increase $h$ where the local error is very small.
This is handy as a fine mesh is not needed far away from the foundation crack - saving computational resources.
In this example we will tell the algorithm to refine the mesh three times, and stop if the total number of elements exceed 1 million, with a maximum coarsening factor of 3, and element growth rate of 1.7, i.e. the number of elements increase by roughly 70\% each iteration.\par

The result of this refinement can be seen in Figure \ref{fig:mesh_refinement} where the original and refined mesh are juxtaposed.
Notice how the mesh is now denser near the foundation, the boundary layers tighter (in particular near the groundwater boundary), and how the elements are larger in the periphery.
The original and refined meshes has 362,657 and 1,065,743 elements respectively.\par

\begin{figure}[htb!]
  \centering
  \begin{subfigure}[b]{\textwidth}
    \includegraphics[width=\textwidth]{meshed_model.png}
    \caption{Original mesh. 362,657 elements.}
    \label{fig:mesh_before_refinement}
  \end{subfigure}
  \begin{subfigure}[b]{\textwidth}
    \includegraphics[width=\textwidth]{mesh_refined.png}
    \caption{Refined mesh after two steps of global refinement w.r.t. the indoor contaminant concentration. 1,065,743 elements.}
    \label{fig:mesh_after_refinement}
  \end{subfigure}
    \caption{Original and refined mesh.}
    \label{fig:mesh_refinement}
\end{figure}

\section{Post-processing \& Results}

One of the benefits of using a FEM software like COMSOL is its advanced post-processing capabilities.
This allows the user to examine the physics driving VI in great detail.
Figure \ref{fig:model_pressure} shows the resulting pressure field from solving Darcy's Law, as well as the associated airflow streamlines in the soil.
Here we see the pressure in the near foundation crack region is roughly the same as the house pressurization of \SI{-5}{\pascal}, which quickly decreases towards the ground surface.
It is also apparent how this pressure field induces a airflow from the ground surface, with air near the house heading relatively straight to the foundation crack, whereas the air further away from the house penetrates deeper into the soil and almost "whirlwinds" underneath the house.\par

\begin{figure}[htb!]
  \centering
  \includegraphics[width=0.75\textwidth]{model_pressure.png}
  \caption[Modeled Darcy's pressure field in soil]{Pressure field from Darcy's Law with associated airflow streamlines.}
  \label{fig:model_pressure}
\end{figure}

\begin{figure}[htb!]
  \centering
  \includegraphics[width=0.75\textwidth]{model_concentration.png}
  \caption[Modeled contaminant concentration in soil]{Contaminant concentration in the soil, normalized to groundwater concentration and log-transformed, with transport streamlines.}
  \label{fig:model_concentration}
\end{figure}

The contaminant concentration in the soil, normalized to the groundwater source concentration and log-transformed, with the contaminant flux streamlines, is examined in Figure \ref{fig:model_concentration}.
Here see that far away from the house, the contaminant vapor simply diffuse straight from the groundwater source to the atmosphere, while beneath the house foundation, contaminant vapors accumulate because the foundation acts as a diffusion blocker.
Based on those streamlines we can conclude that the advective component of the flux is very here small.
Perhaps surprisingly, we do not see a significant advective transport downwards along the wall of the house.
However, considering that the soil type is sandy loam, airflow velocities are expected to be small.\par

One might think that advective transport is large in the horizontal direction along the foundation slab, as the transport and airflow streamlines are so similar.
However, by inspecting Figure \ref{fig:model_velocity_crack} we see that airflow velocities are not greater here than elsewhere, and therefore the advective transport is not either.
To make sense of this, we can inspect the horizontal diffusive flux, divided by the magnitude of the total flux
\begin{equation}
  \frac{j_\mathrm{diff,y-direction}}{|j_\mathrm{total}|}
\end{equation}
to see what portion of the total contaminant flux transport the diffusive horizontal represents here.
Figure \ref{fig:model_horizontal_diff} shows that the horizontal contaminant transport underneath the foundation is in fact driven by the large contaminant concentration gradient between the region underneath and outside the house foundation.
This shows the power of modeling and how it can reveal things that at first seem intuitively correct, but in fact are not.\par

\begin{figure}[htb!]
  \centering
  \includegraphics[width=0.75\textwidth]{model_velocity_crack.png}
  \caption{Airflow velocity $\vec{u}_g$ [\si{\milli\metre\per\hour}] near the foundation crack with associated its streamlines.}
  \label{fig:model_velocity_crack}
\end{figure}

\begin{figure}[htb!]
  \centering
  \includegraphics[width=0.75\textwidth]{model_transport_flux_y.png}
  \caption[Analysis of horizontal diffusion flux.]{Horizontal (y-axis) diffusive flux component normalized to the magnitude of the total flux. A value of 1 here indicates that the total contaminant transport flux is due to diffusion, while 0 would indicate the opposite - that all contaminant transport is due to advection. The sign signifies the direction, with positive and negative values indicating a flux in the rightward and leftward direction respectively. E.g. a value of -0.8 indicates that the 80\% of the magnitude of the contaminant transport is due to horizontal (along the y-axis) diffusion, and occurs leftward.}
  \label{fig:model_horizontal_diff}
\end{figure}

Another useful feature of post-processing is that it can be used for bug searching and to evaluate where the mesh can be potentially improved.
When the transport equation is used to numerically model contaminant transport, there is a tendency for the solution to oscillate around the "true" solution, and thereby violate mass conservation, if the mesh size in a particular element is too large.
This can be quantified by the cell Péclet number, which characterizes the relative magnitude of advection/diffusion in a cell.
\begin{equation}
  \mathrm{Pe_{cell}} = \frac{\mathrm{adv_{cell}}}{\mathrm{diff_{cell}}} = \frac{u_g h}{2 D_\mathrm{eff}}
\end{equation}
here $u_g$ [\si{\metre\per\second}] is the soil-gas airflow velocity;
$h$ [\si{\metre}] is the mesh size in the element or cell;
and $D_\mathrm{eff}$ [\si{\metre\squared\per\second}] is the effective diffusivity in the cell.
If $\mathrm{Pe_{cell}} > 1$ there is a risk that this oscillating behavior will manifest.
Small exceedances, $~\mathrm{Pe_{cell}} < 25$, are usually mitigated by various stabilization schemes, which are inherently integrated into COMSOL as well as many other FEM packages, but for larger values further mesh refinement may be required.\par

Figure \ref{fig:model_cell_peclet} shows $\mathrm{Pe_{cell}}$ as a volume plot, and excludes all values that fall below one.
As we can see, only the region close to the groundwater exceeds $\mathrm{Pe_{cell}}$, which is due to the very small $D_\mathrm{eff}$ there.
The exceedance is small, so the stabilization scheme is able to compensate which is confirmed by Figure \ref{fig:model_concentration} (no oscillations visible).
This is also a region where even if such oscillations occurred, would probably not affect the indoor contaminant concentration.
Regardless, Figure \ref{fig:model_cell_peclet} shows where the mesh may potentially be refined, which comes in handy to know if one runs a model where airflow velocities are significantly higher than in this example.\par

\begin{figure}[htb!]
  \centering
  \includegraphics[width=0.75\textwidth]{model_cell_peclet.png}
  \caption{Volume plot showing where the cell Péclet number exceeds 1 and its actual value. I.e. it suggests where the mesh may be improved.}
  \label{fig:model_cell_peclet}
\end{figure}

%\section{Review of Vapor Intrusion Models}\label{sec:model_review}

Mathematical models of VI were from an early stage adopted by investigators and regulators alike.
The primary purpose of these was to providence screening-level risk assessment, i.e. determine if a particular site likely to impacted by VI based upon specific site characteristics such as groundwater contaminant concentration measurements.
This was necessary as VI sites are potentially numerous and a means to prioritize was needed.
Since such models had successfully been used in radon intrusion, similar ones were, and still are, developed for VI\cite{u.s._environmental_protection_agency_oswer_2015}.\par

The EPA has recommended the use of VI models as a screening risk assessment tool as well as a line-of-evidence in VI investigations in conjunction with field measurements\cite{u.s._environmental_protection_agency_oswer_2015}.
Likewise, various VI models have been used in many European countries in similar applications\cite{provoost_accuracy_2009}.
However, the main obstacle of using models in VI investigations has been, and still is, the difficulties of validating them.
These difficulties stem from the lack of available comprehensive datasets of VI sites and the inability to change the conceptual site model (CSM) that underpins the development of many of the most widely used VI models, e.g. if a particular model assumes the only contaminant source is the groundwater, it will never perform well for a site that is characterized by a preferential pathway.
Regardless, VI models offer a means to examine the underlying physics that drive VI and is therefore a valuable research tool.\par

\subsection{Analytical Models}

One of the first, and arguably one of the most well-used VI model was developed by \citeauthor{johnson_heuristic_1991}\cite{johnson_heuristic_1991}, the J\&E model, and was based on much of the modeling work by \citeauthor{nazaroff_predicting_1988}\cite{nazaroff_predicting_1988}.
Here a VI scenario similar to the one presented here was used as a basis for their mode, i.e. a house overlying an infinitely contaminated groundwater source, where contaminant vapors enter through a foundation crack along the perimeter of the foundation.
However, due they sought to develop an analytical model, and therefore certain physics was discarded to enable them to solve the associated PDE.\par

One such is that contaminant transport from the groundwater source to the building foundation was assumed to occur solely through diffusion.
This is a reasonable assumption, as we have seen airflow is very slow in the soil, and especially in the deeper parts of the soil, so in most scenarios, contaminant transport here will be dominated by diffusion.
The contaminant diffusivity was likewise modified using Millington-Quirks model.
However, their implementation lacked a way to model the soil moisture content, which instead had to be supplied by the user.
Multiple soil layers were supported and with sufficient knowledge or assumptions, and using these, effective diffusivities could be reasonably approximated.\par

While diffusion was assumed to be the only transport mechanism in the soil, both advection and diffusion was assumed to contribute to contaminant entry into the building.
The advective contaminant flow through the foundation crack was here determined using a modified version of Darcy's Law that had been developed previously by Nazaroff\cite{nazaroff_radon_1985} where the driving force was the pressure differential between the indoor and outdoor environments.
However, this approach lacks the relative permeability term from van Genuchten, and requires user input to determine the effective soil permeability.
Contaminant entry through the crack was modeled as transport between two parallel plates, and involved solving the one-dimensional advection-diffusion equation at steady-state.
Indoor contaminant concentration was determined in a similar fashion as presented here, i.e. as a steady-state mass balance between the contaminant entry and expulsion, the latter which is controlled by air exchange rate.
A major drawback of this model is its one-dimensional nature, which forces all of the contaminant released from groundwater beneath a building to enter that building.
That is, no lateral transport of contaminant could be included.\par

In 1998 the EPA implemented the J\&E model as a spreadsheet tool for screening risk, where the user could give a wide variety of input such as air exchange rate, building pressurization, groundwater contaminant concentration, define multiple soil layers with their associated permeabilities and porosities, etc.
Thus, the model was adopted by investigators and regulators as a risk assessment screening tool\cite{u.s._environmental_protection_agency_oswer_2015}.
However, recently many state regulatory agencies have begun to question the use of these sorts of models in VI investigations and currently are given relatively low weight when considering if a site is impacted by VI.\par

Following the J\&E model, a wide variety of analytical models were developed.
These were often similar to the J\&E model in many regards, and often used the same governing equations, but with modifications to accommodate different VI scenarios.
For instance some would have contaminant entry via a crawl space instead of a foundation crack\cite{r._human_1994}, or include soil biodegradation\cite{anderssen_modelling_1997,hers_evaluation_2000}.
\citeauthor{yao_review_2013}\cite{yao_review_2013} wrote a comprehensive review of VI models, discussing their advantages and disadvantages, and which type of scenarios they modeled.
However, due to the analytical nature of these models, some physics had to be omitted in order to develop an analytical solution to that particular problem; this is the inherent disadvantage of analytical VI models.\par

\subsection{Numerical Modeling}

Numerical models do not require the sacrifice of any physical phenomena to be solvable and can be solved in up to three-dimensions, while most analytical models are one-dimensional.
Thus numerical models can offer a more detailed and generalized description of a wider range of VI scenarios.
However, solving numerical models to a satisfactory accuracy can be challenging and often require some expertise on behalf of the user, and such can be less accessible compared to  the analytical spreadsheet models.
But from a research perspective they are far more interesting for examining the physics driving VI.\par

\subsubsection{Abreu and Johnson Model.}

\citeauthor{abreu_effect_2005}\cite{abreu_effect_2005} developed one of the first numerical model of VI - the "ASU model".
This model considered the same VI scenario as in the J\&E model and what has been presented here; an infinitely contaminated groundwater source with contaminant entry into the overlying building occurring through a \SI{1}{\milli\metre} perimeter foundation crack.
The Abreu model was a three-dimensional model developed using a finite difference approach.\par

Abreu used a similar mathematical description to the one presented here, i.e. used Darcy's Law to characterize flow of the soil-gas, the advection-diffusion equation for contaminant soil transport, the indoor environment was modeled as a CSTR, and the expression for contaminant entry into the building was the one developed in the original J\&E model.
Biodegradation was also supported in their model.\par

A key difference between the Abreu model and this one was that they do not simulate the effect of soil moisture as a function of elevation above groundwater.
Instead, different soil moisture content would either be defined by the user for the entire soil domain, or for specific layers - very much like in the J\&E model.
Thus, contaminant transport in the soil would be less well described.\par

The Abreu model was used in a collaborative project with the EPA to investigate a wide range of VI topics.
For instance, they investigated the effect of other buildings adjacent to a VI impacted building, how a laterally located groundwater source (i.e. the source is not directly below the building) affect contaminant entry, and combinations of these.
They also considered finite sources in transient simulations, effect of an impermeable ground cover (like a sidewalk) around a building, and many more.
However, a limitation of their study here was that they only considered sandy soils.\par

\subsubsection{The Brown Model}

A new addition to the family of VI models was the predecessor to the finite element model presented here.
This 3D model was originally developed by \citeauthor{pennell_development_2009}\cite{pennell_development_2009} - the Brown model.
In terms of governing equations and mathematical description of VI, it was quite similar to the Abreu model.
Specifically, soil-gas airflow was again modeled using Darcy's Law, soil contaminant transport with the advection-diffusion equation, the indoor as a CSTR, and expression for contaminant entry into the building was the one from the J\&E model.
The FEM nature of the model meant that heterogenous soil conditions could easily be modeled.
Another benefit of this model was that it was able to run transient, or time-dependent simulations.
However, this model did not calculate soil moisture content and its effect on contaminant transport.\par

Investigating various heterogenous soil conditions was the topic of one of the first works using this model by \citeauthor{bozkurt_simulation_2009}\cite{bozkurt_simulation_2009}.
They investigated how different soil layers of different properties alter the soil-gas contaminant concentration profiles.
Their findings reinforced the importance of accurately characterizing the geology underneath a VI impacted in the development of its conceptual site model.
In particular, clay layers in the soil were found to have a particularly profound effect on soil-gas contaminant concentrations.\par

\citeauthor{yao_vapor_2011}\cite{yao_vapor_2011} used the same model to investigate how "capping" around a building affects soil-gas contaminant concentrations, e.g. how does a sidewalk affect contaminant concentration profiles.
Using this they showed that caps in close proximity to a building can have a significant effect on soil-gas contaminant concentrations.
For instance, buildings with very shallow foundations but were surrounded by an impermeable pavement, had relatively higher indoor contaminant concentrations than buildings without paving but whos foundations were a few meters bgs.
Some other VI investigated by Yao et. al. was to explain the order of magnitude variability in attenuation factors in the EPA database\cite{yao_examination_2013-1}, oxygen limited biodegradation of VOCs in soils\cite{yao_estimation_2014}, and the effect of modeling other contaminant entry pathways other than a perimeter crack\cite{yao_simulating_2013}, where it was concluded that for the "classic" modeled VI scenario, the location or shape of the crack has little impact on overall VI.\par

\citeauthor{shen_impacts_2016}\cite{shen_impacts_2016} used the Brown model to study the effect of time-varying soil-gas entry (i.e. volumetric air flow) and air exchange rate.
Here they showed that these variations can significantly contribute to variations in indoor contaminant concentrations.
Air exchange rate could under some circumstances contribute to roughly an order of magnitude, while soil-gas entry more than that.\par

\subsubsection{CVI2D and PVI2D}

\citeauthor{verginelli_excel-based_2016}\cite{verginelli_excel-based_2016} developed a steady-state two-dimensional analytic VI model.
Here they considered the "classic" VI scenario with a free-standing building, surrounded by open ground, with a perimeter crack, and a groundwater source.
A solution to the presented governing equations was established using the Schwarz–Christoffel mapping method.
This model comes in two versions, one for primarily chlorinated solvents - chlorinated vapor intrusion tool 2D (CVI2D).
The other mainly deals with petroleum contaminants and VOCS, and supports (oxygen limited) biodegradation - petroleum vapor intrusion tool 2D (PVI2D).
These tools have been well received, in particular by the Chinese regulatory community, where they are often use to assess potential VI risk at brownfield sites prior to new construction.\par


\appendix
\clearpage
\begin{appendices}

  \section{Geometry Generation}

  To create our quarter geometry, only a few simple geometric objects and Boolean operations are required: two cuboids, two rectangles, one Boolean difference operation, and one Boolean join operation.
  Figure \ref{fig:geometry} shows the resulting geometry.
  Note that $z = \SI{0}{\metre}$ is the groundwater/soil interface and the plane of symmetry is around the $(x, y) = (\SI{0}{\metre},\SI{0}{\metre})$ axis\par

  To create the soil surrounding the building using the COMSOL geometry generator:
  \begin{enumerate}
    \item Create a \SI{15}{\metre} by \SI{15}{\metre} by \SI{4}{\metre} block with its base at $(x, y, z) = (\SI{0}{\metre},\SI{0}{\metre},\SI{0}{\metre})$. This is the entire soil domain.
    \item Create a \SI{5}{\metre} by \SI{5}{\metre} by \SI{1}{\metre} block with its base at $(x, y, z) = (\SI{0}{\metre},\SI{0}{\metre},\SI{3}{\metre})$. This will represent the volume that the house take up in the soil, i.e. the underground portion of the basement.
    \item Perform a difference operation, removing the "basement" block from the "soil" block.
  \end{enumerate}
  At this point you will see that a quarter soil domain has been created, with an empty space that represents a house with a foundation slab located \SI{1}{\metre} bgs.\par

  The foundation crack will be modeled by joining two  \SI{1}{\centi\metre} wide strip that spans the perimeter of the surface that represents the house foundation.
  This strip is created by joining two rectangles on foundation surface:
  \begin{enumerate}
    \item Define a work plane \SI{3}{\metre} above zero. This allows us to place two-dimensional objects on the surface of or inside a three-dimensional object.
    \item On the work plane create a \SI{5}{\metre} by \SI{1}{\centi\metre} rectangle with its base at $(x, y) = (\SI{0}{\metre},\SI{5}{\metre} - \SI{1}{\centi\metre})$. This represents one side of the perimeter crack.
    \item Copy the rectangle and rotate it \SI{90}{\degree} around the corner of the foundation, i.e. $(x, y) = (\SI{5}{\metre} - \SI{0.5}{\centi\metre},\SI{5}{\metre} - \SI{0.5}{\centi\metre})$.
    \item Join the two rectangles to create a unified perimeter foundation crack.
  \end{enumerate}
  Now the geometry of this VI scenario is complete.\par

  \section{Properties}
  % TODO: Make sure gravel density data is correct
  % TODO: Flip the table?
  \begin{table}
    \centering
    \caption{Properties and van Genuchten parameters of select soil types\cite{abreu_conceptual_2012}.}
    \label{tbl:soils}
  \begin{tabular}{c c c c c c c}
    \toprule
    \multirow{2}{*}{Soil type} & Permeability & Density & Porosity & Residual moisture & \multicolumn{2}{c}{van Genuchten parameters} \\
    & $\kappa \; \mathrm{(m^2)}$ & $\rho \; \mathrm{(kg/m^3)}$ & $\theta_t$ & $\theta_r$ & $\alpha$ & $m$ \\
    \hline
    Sand & \num{9.9e-12} & 1430 & 0.38 & \num{5.3e-2} & 3.5 & 3.2 \\
    Loamy sand  & \num{1.6e-12} & 1430 & 0.39 & \num{4.9e-2} & 3.5 & 1.7 \\
    Sandy loam  & \num{5.9e-13}  & 1460 & 0.39 & \num{3.9e-2} & 2.7 & 1.4 \\
    Sandy clay loam  & \num{2.0e-13} & 1430 & 0.38 & \num{6.3e-2} & 2.1 & 1.3 \\
    Loam  & \num{1.9e-13}& 1380 & 0.40 & \num{6.1e-2} & 1.5 & 1.5 \\
    Silt loam  & \num{2.8e-13} & 1380 & 0.44 & \num{6.5e-2} & 0.51 & 1.7 \\
    Clay loam  & \num{1.3e-13}  & 1500 & 0.44 & \num{7.9e-2} & 1.6 & 1.4 \\
    Silty clay loam & \num{1.7e-13} & 1390 & 0.48 & \num{9.0e-2} & 0.84 & 1.5 \\
    Silty clay  & \num{1.5e-13} & 1300 & 0.48 & \num{1.1e-1} & 1.6 & 1.3 \\
    Silt  & \num{6.7e-13} & 1260 & 0.49 & \num{5.0e-2} & 0.66 & 1.7 \\
    Sandy clay  & \num{1.7e-13} & 1470 & 0.39 & \num{1.2e-1} & 3.3 & 1.2 \\
    Clay  & \num{2.3e-13} & 1330 & 0.46 & \num{9.8e-2} & 1.3 & 1.3 \\
    Gravel\cite{dan_capillary_2012} & \num{1.3e-9} & 1430 & 0.42 & \num{5.0e-3} & 100 & 2.19 \\
    \bottomrule
  \end{tabular}
  \end{table}
\end{appendices}

\end{comment}



\begin{comment}
Things to bring up:

- Why is it beneficial to model VI?


- Background on modeling in VI research
-- History & modeling applications
-- Limitations of other people's models
* Should this come before or after my mathematical description of VI?

- Mathematical description of VI modeling
* Start with a non-technical description of a CSM and the processes/physics underlying it. A nice graph showing all of it (including labeled physics and fluxes) would be nice.

- Short description of FEM
-- COMSOL
-- Basic mathematical concept
-- Advantages/disadvantages of this approach compared to others, i.e. why pick FEM over other numerical schemes?
-- Meshing
\end{comment}


\end{document}
