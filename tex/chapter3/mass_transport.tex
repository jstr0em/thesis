\subsection{Mass Transport in Unsaturated Porous Media}\label{sec:mass_transport}

Mass transport of a chemical species occurs primarily through diffusive and advective transport and is typically governed by the advection-diffusion equation (sometimes -reaction is added)
\begin{equation}
  \frac{\partial c_i}{\partial t} + \nabla \cdot (-D \nabla c_i) + \vec{u} \cdot \nabla c_i = R_i
\end{equation}
where $c_i$ is the concentration of the chemical species; $t$ is time; $D$ is the diffusion coefficient; $\vec{u}$ is the bulk fluid velocity vector; and $R_i$ is a reaction term.
As such the first term is the change of concentration in some control volume, the second and third terms are the diffusive and advective fluxes leaving or entering the control volume, and the fourth is whatever change in concentration due to chemical reactions.\par

However, this governing equation is too simplistic to accurately model vapor contaminant transport in the vadose zone.
As has been discussed before, the transport properties vary significantly in unsaturated porous depending on the soil matrix water saturation.
The vadose zone is also a three-phase system, where at any given time some contaminant will be partitioned between the soil, vapor, and liquid phases.
The contaminant will also move between these three phases through volatilization/solvation and sorption.
To accurately model the mass transport of contaminants through the vadose zone, all of these phenomena must be accounted for.\par

% TODO: Rewrite so c_i is the gas-phase instead
\begin{equation}
  \frac{\partial}{\partial t} (\theta c_i) +
  \frac{\partial}{\partial t} (\rho_b c_{P,i}) +
  \frac{\partial}{\partial t} (a_v c_{G,i}) +
  \vec{u} \cdot \nabla c_i =
  \nabla \cdot [(D_{D,i} + D_{eff,i}) \nabla c_i] +
  R_i + S_i
\end{equation}


\subsubsection{Boundary conditions}

As always, some BC are necessary to be specified to solve the PDE governing the mass transport.
To model the basic CSM, four different kinds of BCs are required, one for the contaminant source, a second for the sink, third for the entry rate (coupling) with the indoor environment, and fourth are the no flow BCs that have been applied in previous scenarios.

\paragraph{Groundwater source}

In our CSM contaminant vapors emanate from a homogenously contaminated groundwater source; contaminated with some $c_gw$ concentration.
These contaminant vapors can be assumed to emanate from the groundwater according to Henry's Law and the product.
The product of the liquid contaminant concentration and the dimensionless Henry's Law constant thus give the vapor contaminant at the interface between the vadose zone and water table.

\paragraph{Contaminant sink}

Any contaminant vapor that reaches the atmosphere is assumed to immediately become infinitely diluted, i.e. that the vapor contaminant here is zero.
This is simplest modeled by setting the vapor contaminant concentration to zero at the ground surface boundary.

\paragraph{Foundation crack entry}

% TODO: TBD...

\paragraph{Summary}

The last BC is, as in the other cases, a no flow boundary and applied to the remaining boundaries.
In summary the following BCs are used.
\begin{align}\label{eq:mass_transport_bc}
  &\text{Water table} &c = c_{gw} K_H \; \mathrm{(mol/m^3)} \\
  &\text{Ground surface} &c = 0 \; \mathrm{(mol/m^3)} \\
  &\text{Foundation crack} &x \\
  &\text{Remaining} &-\vec{n}\cdot\vec{N} = 0 \; \mathrm{(mol/(m^2\cdot s))}
\end{align}
Since the CSM is a steady-state simulation the initial values is not important to specify, but can reduce computation time.
If a transient simulation is run the initial values are very important and typically a steady-state solution is used as the initial values.\par
