\subsection{Mass Transport in Unsaturated Porous Media}\label{sec:mass_transport}

Mass transport of a chemical species occurs primarily through diffusive and advective transport and is typically governed by the advection-diffusion equation (sometimes -reaction is added)
\begin{equation}\label{eq:advection_diffusion_reaction}
  \frac{\partial c_i}{\partial t} + \nabla \cdot (-D \nabla c_i) + \vec{u} \cdot \nabla c_i = R_i
\end{equation}
where $c_i$ is the concentration of the chemical species; $t$ is time; $D$ is the diffusion coefficient; $\vec{u}$ is the bulk fluid velocity vector; and $R_i$ is a reaction term.
As such the first term is the change of concentration in some control volume, the second and third terms are the diffusive and advective fluxes leaving or entering the control volume, and the fourth is whatever change in concentration due to chemical reactions.\par

However, this governing equation is too simplistic to accurately model vapor contaminant transport in the vadose zone.
As has been discussed before, the transport properties vary significantly in unsaturated porous depending on the soil matrix water saturation.
The vadose zone is also a three-phase system, where at any given time some contaminant will be partitioned between the soil, vapor, and liquid phases.
The contaminant will also move between these three phases through volatilization/solvation and sorption.
To accurately model the mass transport of contaminants through the vadose zone, all of these phenomena must be accounted for.\par

To rectify this the first term in \eqref{eq:advection_diffusion_reaction} is split to reflect the three phases present.
\begin{equation}\label{eq:mass_distribution}
  \frac{\partial c_i}{\partial t} =
  \frac{\partial}{\partial t} \Big( \theta c_i + \rho_b c_{p,i} + \theta_g c_{g,i} \Big)
\end{equation}
here $\theta$ is the water filled porosity and its product with $c_i$ represent the portion of $c_i$ in the liquid phase; $\rho_b$ is the bulk density of the soil and $c_{p,i}$ is $c_i$ sorbed onto/into the solid phase; $\theta_g$ is the vapor filled porosity and $c_{g,i}$ is the volatilized $c_i$.\par

From Richards' equation we get both $\theta$ and $\theta_g = \epsilon - \theta$ ($\epsilon$ is the total porosity of the soil).
The bulk density is given by $\rho_b = (1-\epsilon) \rho$ where $\rho$ is the soil density.
The sorbed concentration is $c_{p,i} = K_p c_i$ where $K_p$ is the sorption isotherm in $\mathrm{m^3/kg}$ thus $c_{p,i}$ is given in $\mathrm{mol/kg}$.
The vapor or gas phase concentration is assumed to be governed by Henry's Law and as such $c_{g,i} = K_{H,i} c_i$ - note that we use the dimensionless Henry's Law constant.\par

By taking the derivatives in \eqref{eq:mass_distribution} and by introducing an effective diffusion coefficient $D_\mathrm{eff}$ in lieu of $D$ we arrive at
\begin{equation}\label{eq:mass_transport}
  \begin{split}
    \frac{\partial c_i}{\partial t} (\theta + \rho_b K_{p,i} + \theta_g K_{H,i}) +
    \frac{\partial \theta}{\partial t} c_i (1-K_{H,i}) +
    \frac{\partial \epsilon}{\partial t} c_i (K_{H,i} - \rho K_{p,i}) \\ +
    \vec{u} \cdot \nabla c_i =
    \nabla \cdot (D_{eff,i} \nabla c_i) +
    R_i
  \end{split}
\end{equation}
here one could add reactions to $R_i$ to for instance model biodegradation of contaminants.
However, since we are interested in modeling VI of TCE, we can omit this since it does not readily degrade.\par

\paragraph{Effective diffusivity}

The last unexplained part of \eqref{eq:mass_transport} is what the effective diffusion coefficient $D_\mathrm{eff,i}$ is.
Liquid and gas diffusion coefficients usually differ by orders of magnitude and it is necessary to introduce an effective diffusion coefficient to describe diffusion in unsaturated porous media.
This is commonly done via the Millinton-Quirk equation\cite{millington_permeability_1961}
\begin{equation}\label{eq:millington_quirk}
  D_\mathrm{eff} = D_\mathrm{water} \frac{\theta^{\frac{10}{3}}}{\epsilon^2} + \frac{D_\mathrm{air}}{K_H} \frac{\theta_g^{\frac{10}{3}}}{\epsilon^2}
\end{equation}
where $D_\mathrm{water}$ is the liquid diffusion coefficient; $D_\mathrm{air}$ is the gas or vapor diffusion coefficient; and $K_H$ is the dimensionless Henry's Law constant.\par

\subsubsection{Boundary conditions}

As always, some BC are necessary to be specified to solve the PDE governing the mass transport.
To model the basic CSM, four different kinds of BCs are required, one for the contaminant source, a second for the sink, third for the entry rate (coupling) with the indoor environment, and fourth are the no flow BCs that have been applied in previous scenarios.

\paragraph{Groundwater source}

In our CSM contaminant vapors emanate from a homogenously contaminated groundwater source; contaminated with some $c_gw$ concentration.
These contaminant vapors can be assumed to emanate from the groundwater according to Henry's Law and the product.
The product of the liquid contaminant concentration and the dimensionless Henry's Law constant thus give the vapor contaminant at the interface between the vadose zone and water table.

\paragraph{Contaminant sink}

Any contaminant vapor that reaches the atmosphere is assumed to immediately become infinitely diluted, i.e. that the vapor contaminant here is zero.
This is simplest modeled by setting the vapor contaminant concentration to zero at the ground surface boundary.

\paragraph{Foundation crack entry}

% TODO: TBD...

\paragraph{Summary}

The last BC is, as in the other cases, a no flow boundary and applied to the remaining boundaries.
In summary the following BCs are used.
\begin{align}\label{eq:mass_transport_bc}
  &\text{Water table} &c = c_{gw} K_H \; \mathrm{(mol/m^3)} \\
  &\text{Ground surface} &c = 0 \; \mathrm{(mol/m^3)} \\
  &\text{Foundation crack} &x \\
  &\text{Remaining} &-\vec{n}\cdot\vec{N} = 0 \; \mathrm{(mol/(m^2\cdot s))}
\end{align}
Since the CSM is a steady-state simulation the initial values is not important to specify, but can reduce computation time.
If a transient simulation is run the initial values are very important and typically a steady-state solution is used as the initial values.\par
