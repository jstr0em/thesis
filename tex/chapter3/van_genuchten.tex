The distribution of soil moisture in the soil matrix has profound implications for the advective and diffusive transport of contaminants.
Soil has a limited amount of pore volume available for contaminant transport, and the presence of water restricts this further; decreasing permeability of the soil and subsequently reduces air flow.
Diffusivity of the contaminant will also be retarded by the water.
The contaminant will dissolve into and evaporate from water and the transport will partially occur through water.
Liquid diffusion coefficients are usually around four orders of magnitude smaller than in air.

The soil moisture content of soils can be estimated in many ways, but two common approaches is to use the analytical formulas of \textit{van Genuchten} or \textit{Brooks and Corey}.
Both of these formulas give the soil moisture content as a function of the fluid pressure head, $H_p$.
By definition, when the pressure head is equal to or greater than zero, $H_p \geq 0$, the soil is assumed to be 100\% saturated with the fluid.
In this work, \textit{van Genuchten's} formula is used.

The soil moisture content, $\theta$ is given by.
\begin{equation}
  \theta = \begin{cases}
    \theta_r + \mathrm{Se}(\theta_s - \theta_r) & H_p < 0 \\
    \theta_s & H_p \geq 0
\end{cases}
\end{equation}

The saturation is given by.
\begin{equation}
  \mathrm{Se} = \begin{cases}
    \frac{1}{(1 + |\alpha H_p|^m)^m} & H_P < 0 \\
    1 & H_p \geq 0
  \end{cases}
\end{equation}

\begin{equation}
  C_m = \begin{cases}
    \frac{\alpha m}{1-m}(\theta_s - \theta_r)\mathrm{Se}^{\frac{1}{m}}\big( 1 - \mathrm{Se}^{\frac{1}{m}} \big)^m & H_p < 0 \\
    0 & H_p \geq 0
  \end{cases}
\end{equation}

\begin{equation}
  k_r = \begin{cases}
    \mathrm{Se}^l \big[ 1 - \big( 1 - \mathrm{Se}^\frac{1}{m} \big) \big]^2 & H_p < 0 \\
    0 & H_p \geq 0
  \end{cases}
\end{equation}
