No models are true representations of reality, but some of them may be useful.
Ever since Newton first wrote his laws of motion, mankind has tried to describe reality with an ever increasing number of mathematical statements.
With the advent of computation and advancements in numerical methods our capabilities to mathematically describe physical systems has dramatically increased.
Even so, real-world systems are too complex to be fully modeled, but mathematical representations may be used to approximate and reveal useful insights of how they function.\par

This is especially true for vapor intrusion (VI) models.
Often it is impossible or difficult to conduct controlled studies of VI sites making models an important tool for understanding these sites and the VI phenomena.
The previous chapter is proof of this as it is readily apparent that a multitude of VI models of varying complexity have been developed over the years, and has become an integral part of the scientific VI community.
From the simple Johnson \& Ettinger one-dimensional model to full three-dimensional finite element models we see that the increased complexity of the model allowed for a greater number of VI topics and phenomena to be explored.\par

% TODO: Write about that we use COMSOL, a little about FEM, and why we use COMSOL

The development of the three-dimensional finite element models begin with a conceptual site model (CSM) of a VI site.
In general when one develops models, it is best in the beginning to keep the model as simple as possible, and not to add overly complex features or excessive physics.
As such, we begin with a very simple CSM which may be seen in Figure \ref{fig:csm}.\par

This CSM features a residential building with a 10 by 10 m footprint, with a concrete foundation one meter below ground-surface (bgs).
Along the perimeter of foundation there is a one cm wide breach, through the subsurface contaminants enter the house.
Three meters below the foundation, there is a contaminated groundwater source, from which contaminants vapor continuously evaporate.
The house is assumed to be depressurized relative to the atmosphere which creates a pressure gradient, allowing air to be pulled through the ground-surface, soil, and into the house - carrying some contaminants with it.
The indoor air is also exchanged at a constant rate with the outside environment, which is the only way the contaminant leave the house.
For simplicity we also assume that the soil is completely homogenous.\par

% TODO: This paragraph needs some more elaboration
The next step is to choose the appropriate physics to accurately model the VI processes at this site.
This is divided up into two steps.
In the first we consider the indoor environment.
In the second the soil surrounding the structure.\par

\begin{figure} % TODO: Create figure
  %\includegraphics{}
  \caption{Example of a simple conceptual site model of a vapor intrusion site.}
  \label{fig:csm}
\end{figure}
