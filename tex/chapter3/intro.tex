\section{Introduction}

No models are true representations of reality, but some of them may be useful.
Ever since Newton first wrote his laws of motion, mankind has tried to describe reality with an ever increasing number of mathematical statements.
With the advent of computation and advancements in numerical methods our capabilities to mathematically describe physical systems has dramatically increased.
Even so, real-world systems are too complex to be fully modeled, but mathematical representations may be used to approximate and reveal useful insights of how they function.\par

This is especially true for vapor intrusion (VI) models.
Often it is impossible or difficult to conduct controlled studies of VI sites making models an important tool for understanding these sites and the VI phenomena.
The previous chapter is proof of this as it is readily apparent that a multitude of VI models of varying complexity have been developed over the years, and has become an important part of the scientific VI community.
From the simple Johnson \& Ettinger one-dimensional model to full three-dimensional finite element models (FEM) we see that the increased complexity of the model allowed for a greater number of VI topics and phenomena to be explored.\par

% TODO: Write about that we use COMSOL, a little about FEM, and why we use COMSOL
The processes of VI may be described by partial differential equations (PDEs).
Unfortunately, there rarely are any analytical solutions to these (except in the most simple cases) and numerical methods are required to find approximate solutions.
One of the most powerful numerical methods for solving PDEs is the finite element method, which not only allows us to find solutions to PDEs but does so for complex three-dimensional geometries.\par

The purposes of this thesis is not to explain the FEM in any great detail, but there are many great resources available for those who are interested to learn more. % TODO: Add reference to some FEM books
There are however, two things that are important to know what makes the FEM unique.\par

The first is the FEM divides up a complicated geometry into smaller \textit{finite elements}, hence its name.
Which elements exactly depend on the dimensionality of the model and the specific problem that one wish to solve.
Three-dimensional geometries are usually represented by tetrahedral and two-dimensional ones by triangles.\par
% TODO: Maybe write a bit more about how the elements are stitched together to give a full solution.

The second is that the solution to a PDE may be represented by a linear combination of a series of \textit{basis functions} with an associated function \textit{coefficient}.
\begin{equation}
  u \approx \sum_i u_i \psi_i
\end{equation}
where $u$ is the solution to the PDE, $u_i$ is the coefficient associated with the basis function $\psi_i$.
This approximation allows the PDE to be discretized into a matrix and the $u_i$ coefficients are solved for.
Any function may serve as a \textit{basis function}, but typically a simple one is chosen (for simpler computation) like a linear hat function or low-degree polynomial.
In certain applications, some basis functions perform better than other, but in most cases linear hat functions or second-degree polynomials are preferable.\par

% TODO: Maybe add some more information about the weak-formulation and how it is associated with this.

% MODELING STARTS WITH A CSM
The development of the three-dimensional finite element vapor intrusion models begin with a conceptual site model (CSM) of a VI site.
In general when one develops models, it is best in the beginning to keep the model as simple as possible, and not to add overly complex features or excessive physics.
As such, we begin with a very simple CSM which may be seen in Figure \ref{fig:csm}.\par

% DESCRIBING THE CSM
This CSM features a residential building with a 10 by 10 m footprint, with a concrete foundation one meter below ground-surface (bgs).
Along the perimeter of foundation there is a one cm wide breach, through the subsurface contaminants enter the house.
Three meters below the foundation, there is a contaminated groundwater source, from which contaminants vapor continuously evaporate.
The house is assumed to be depressurized relative to the atmosphere which creates a pressure gradient, allowing air to be pulled through the ground-surface, soil, and into the house - carrying some contaminants with it.
The indoor air is also exchanged at a constant rate with the outside environment, which is the only way the contaminant leave the house.
For simplicity we also assume that the soil is completely homogenous.\par

\begin{figure} % TODO: Create figure
  %\includegraphics{}
  \caption{Example of a simple conceptual site model of a vapor intrusion site.}
  \label{fig:csm}
\end{figure}

To implement this CSM as a finite-element model several steps must be followed.
\begin{enumerate}
  \item Construct a model geometry (domain).
  \item Assign relevant partial differential equations (PDEs) and boundary conditions (BCs) that describe the physics.
  \item Mesh the geometry.
  \item Configure and choose solvers.
  \item Post-processing.
\end{enumerate}
Each step will be carefully explained, beginning with the construction of the domain.
