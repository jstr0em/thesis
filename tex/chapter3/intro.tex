No models are true representations of reality, but some of them may be useful.
Ever since Newton first wrote his laws of motion, mankind has tried to describe reality with an ever increasing number of mathematical statements.
With the advent of computation and advancements in numerical methods our capabilities to mathematically describe physical systems has dramatically increased.
Even so, real-world systems are too complex to be fully modeled, but mathematical representations may be used to approximate and reveal useful insights of how they function.\par

This is especially true for vapor intrusion (VI) models.
Often it is impossible or difficult to conduct controlled studies of VI sites making models an important tool for understanding these sites and the VI phenomena.
The previous chapter is proof of this as it is readily apparent that a multitude of VI models of varying complexity have been developed over the years, and has become an integral part of the scientific VI community.
From the simple Johnson \& Ettinger one-dimensional model to full three-dimensional finite element models we see that the increased complexity of the model allowed for a greater number of VI topics and phenomena to be explored.\par

The development of the three-dimensional finite element models begin with a conceptual site model (CSM) of a VI site.
In general when one develops models, it is best in the beginning to keep the model as simple as possible, and not to add overly complex features or excessive physics.


% TODO: Integrate or remove paragraphs below
To model vapor intrusion (VI), a number of partial differential equations (PDEs) which describe the relevant physics must be solved.
Many of these PDEs are also coupled in implicit and explicit ways, e.g. the PDE describing vapor flow in the soil affect the contaminant concentration, which in turn affect the indoor air concentration, which in turn also affects the contaminant concentration in the soil.

The indoor air space is perhaps the most important part of modeling VI, as the goal of these models ultimately is to predict indoor exposure given external factors.
One could therefore assume that most of the effort in modeling VI should be spent to accurately represent the interior.
This would be very impractical however, as building interiors are so diverse.
Even if one would spend the time to model an interior, this would dramatically increase the number of mesh elements required to solve the model.
Additionally, the air flow inside the building must be calculated, and even using a simplified version of Navier-Stokes, like Reynolds Averaged Navier-Stokes, the computational cost would be significant.
For these reasons, the indoor air space is simply modeled as a continuously stirred tank (CST), and paradoxically becomes the simplest component of the VI model.\par

Most of the effort will be spent to accurately model the physics in the soil underlying the building of interest.
