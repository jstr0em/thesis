\documentclass[../thesis.tex]{subfiles}

\begin{document}
\chapter{Modeling Methodology}

\section{Introduction}

\section{Water Flow in Unsaturated Porous Media}

% Theory soil-water potential and its modeling with Richard's equation
\subsection{Soil-Water Potential}

% van Genuchten's equation
\subsection{Soil-Water Retention Curve}

% Darcy's Law
\section{Vapor Transport in Unsaturated Porous Media}

% Mass Transport
\section{Mass Transport in Unsaturated Porous Media}

% TODO: Add multiphase theory for NAPL and DNAPL transport in vadose zone?

\section{References}

% TODO: Use a \input for each section instead of writing things directly

Vapor transport in porous media is described by \textit{Darcy's Law}.
The vapor velocity depends on the pressure gradient in the soil, is proportional to the permeability of the soil matrix, and is inversely proportional to the viscosity of the fluid.

\begin{equation}\label{eq:darcys_law_saturated}
  \vec{u} = -\frac{\kappa}{\mu}\nabla p
\end{equation}

For Darcy's Law to be valid, two assumptions must be fulfilled:
\begin{enumerate}
  \item The fluid must be in the laminar regime, typically $\mathrm{Re} < 1$.
  \item The soil matrix must be saturated with the fluid.
\end{enumerate}
In VI-modeling, the first assumption is fulfilled, but the second is not.
Most of the contaminant vapor transport takes place in the partially saturated vadose zone and thus, \eqref{eq:darcys_law_saturated} needs modification.

In partially saturated soils, a varying portion of the soil pores are available for vapor transport, with the rest being occupied by water, affecting the effective permeability of the soil.
To model this, a relative permeability property, $k_r$, is introduced:
\begin{equation}
  \kappa_\mathrm{eff} = k_r \kappa_s
\end{equation}
$k_r$ is a dimensionless parameter that varies between 0 and 1, and $\kappa_s$ is the saturated, or simply the soil matrix permeability.

This gives the modified Darcy's Law used in VI-modeling:
\begin{equation}\label{eq:darcys_law}
  \vec{u} = -\frac{k_r \kappa_s}{\mu}\nabla p
\end{equation}


\subsection{Soil Moisture Retention Curves}

The distribution of soil moisture in the soil matrix has profound implications for the advective and diffusive transport of contaminants.
Soil has a limited amount of pore volume available for contaminant transport, and the presence of water restricts this further; decreasing permeability of the soil and subsequently reduces air flow.
Diffusivity of the contaminant will also be retarded by the water.
The contaminant will dissolve into and evaporate from water and the transport will partially occur through water.
Liquid diffusion coefficients are usually around four orders of magnitude smaller than in air.

The soil moisture content of soils can be estimated in many ways, but two common approaches is to use the analytical formulas of \textit{van Genuchten} or \textit{Brooks and Corey}.
Both of these formulas give the soil moisture content as a function of the fluid pressure head, $H_p$.
By definition, when the pressure head is equal to or greater than zero, $H_p \geq 0$, the soil is assumed to be 100\% saturated with the fluid.
In this work, \textit{van Genuchten's} formula is used.

The soil moisture content, $\theta$ is given by.
\begin{equation}
  \theta = \begin{cases}
    \theta_r + \mathrm{Se}(\theta_s - \theta_r) & H_p < 0 \\
    \theta_s & H_p \geq 0
\end{cases}
\end{equation}

The saturation is given by.
\begin{equation}
  \mathrm{Se} = \begin{cases}
    \frac{1}{(1 + |\alpha H_p|^m)^m} & H_P < 0 \\
    1 & H_p \geq 0
  \end{cases}
\end{equation}

\begin{equation}
  C_m = \begin{cases}
    \frac{\alpha m}{1-m}(\theta_s - \theta_r)\mathrm{Se}^{\frac{1}{m}}\big( 1 - \mathrm{Se}^{\frac{1}{m}} \big)^m & H_p < 0 \\
    0 & H_p \geq 0
  \end{cases}
\end{equation}

\begin{equation}
  k_r = \begin{cases}
    \mathrm{Se}^l \big[ 1 - \big( 1 - \mathrm{Se}^\frac{1}{m} \big) \big]^2 & H_p < 0 \\
    0 & H_p \geq 0
  \end{cases}
\end{equation}

\section{Temperature \& Pressure Dependent Properties}



\end{document}
