% TODO: You really need to sit down and wrap your head around all of this and how it's derived.

\subsection{Water Flow in Unsaturated Porous Media}\label{sec:richards}

The vadose zone or unsaturated zone is a region of soil between the top of the ground surface and the water table.
In the vadose zone there are two fluid phases, one gas and the other liquid (usually air and water) inside the porous soil matrix giving a three phase system; only one fluid phase (gas or liquid) exist in the saturated zone.
As a result, the transport properties in the vadose zone differ from that in a zone saturated where there are only two phases present - water and soil.\par



\subsubsection{Soil-Water Potential}

The driving force, or soil-water potential, for the filling and draining of pore water in soils are due to a pressure and a gravitational potential and given by $\phi$.


The driving force for the pore water in the vadose zone is a negative pressure caused by the surface tension of the water.
This phenomena is called \textit{capillary potential} or \textit{matrix potential}, $\psi$, which depends on the volumetric water content $\theta$ in the soil.

\subsubsection{Soil-Water Retention Curve}

The distribution of soil moisture in the soil matrix has profound implications for the advective and diffusive transport of contaminants.
Soil has a limited amount of pore volume available for contaminant transport, and the presence of water restricts this further; decreasing permeability of the soil and subsequently reduces air flow.
Diffusivity of the contaminant will also be retarded by the water.
The contaminant will dissolve into and evaporate from water and the transport will partially occur through water.
Liquid diffusion coefficients are usually around four orders of magnitude smaller than in air.

The soil moisture content of soils can be estimated in many ways, but two common approaches is to use the analytical formulas of \textit{van Genuchten} or \textit{Brooks and Corey}.
Both of these formulas give the soil moisture content as a function of the fluid pressure head, $H_p$.
By definition, when the pressure head is equal to or greater than zero, $H_p \geq 0$, the soil is assumed to be 100\% saturated with the fluid.
In this work, \textit{van Genuchten's} formula is used.

The soil moisture content, $\theta$ is given by.
\begin{equation}
  \theta = \begin{cases}
    \theta_r + \mathrm{Se}(\theta_s - \theta_r) & H_p < 0 \\
    \theta_s & H_p \geq 0
\end{cases}
\end{equation}

The saturation is given by.
\begin{equation}
  \mathrm{Se} = \begin{cases}
    \frac{1}{(1 + |\alpha H_p|^m)^m} & H_P < 0 \\
    1 & H_p \geq 0
  \end{cases}
\end{equation}

\begin{equation}
  C_m = \begin{cases}
    \frac{\alpha m}{1-m}(\theta_s - \theta_r)\mathrm{Se}^{\frac{1}{m}}\big( 1 - \mathrm{Se}^{\frac{1}{m}} \big)^m & H_p < 0 \\
    0 & H_p \geq 0
  \end{cases}
\end{equation}

\begin{equation}
  k_r = \begin{cases}
    \mathrm{Se}^l \big[ 1 - \big( 1 - \mathrm{Se}^\frac{1}{m} \big) \big]^2 & H_p < 0 \\
    0 & H_p \geq 0
  \end{cases}
\end{equation}



\subsubsection{Richard's Equation}

\begin{equation}\label{eq:richards}
  \rho \Big( \frac{C_m}{\rho g} + \mathrm{Se}S \Big) \frac{\partial p}{\partial t} +
  \nabla \cdot \rho \Big( -\frac{\kappa_s}{\mu} k_r (\nabla p + \rho g \nabla D)\Big) =
  Q_m
\end{equation}
Where $p$ is the capillary potential; $C_m$ is the specific moisture capacity; $\mathrm{Se}$ is the effective saturation; $S$ is the storage coefficient; $\kappa_s$ is the saturated permeability of the porous media; $\mu$ is the fluid viscosity; $k_r$ is the effective permeability; $\rho$ is the fluid density; $g$ is the acceleration of gravity; $D$ is the elevation or head; and $Q_m$ is a source term, a positive or negative value represent a source or sink respectively.\par
