\section{Soil Physics Governing Vapor Intrusion}\label{sec:soil_domain}

The soil surrounding the structure in our VI model is, unlike the indoor environment, modeled explicitly.
In this section we will walk through each physics, the associated governing equation, and boundary conditions required to model the contaminant transport in soil.
The following physics and governing equations will be covered:
\begin{enumerate}
  \item Water flow in unsaturated porous media
  \item Vapor transport in unsaturated porous media
  \item Mass transport in partially saturated medium
\end{enumerate}
In addition to these, the modeling of the temperature distribution in the soil is covered in Appendix .\par % TODO: Reference the appendix once it exists.

The vapor contaminant transport through the soil in VI occurs through the vadose zone - soil that is partially filled with water, giving a three-phase transport system.
This partial water content has profound effects on both advective and diffusive transport and modeling the water content is achieved via \textit{Richard's equation}.
Since the vapor and mass transport in the soil are so dependent on the soil water content, it is covered first in the following section \ref{sec:richards}.\par

The mass transport in the soil has both an advective and diffusive component.
The advective transport in the soil is dictated by the vapor flow in the soil which is described by \textit{Darcy's Law} making it the next logical step to cover in section \ref{sec:darcys}.
The diffusive transport depends on the contaminant vapor concentration itself and accurately modeling this requires coupling with all of the physics discussed so far (including the indoor environment).
Therefore the mass transport physics, governed by the \textit{advection-diffusion equation} will be covered last in section \ref{sec:mass_transport}.\par 

\subsection{Water Flow in Unsaturated Porous Media}\label{sec:richards}


\subsubsection{Soil-Water Potential}

\subsubsection{Soil-Water Retention Curve}

The distribution of soil moisture in the soil matrix has profound implications for the advective and diffusive transport of contaminants.
Soil has a limited amount of pore volume available for contaminant transport, and the presence of water restricts this further; decreasing permeability of the soil and subsequently reduces air flow.
Diffusivity of the contaminant will also be retarded by the water.
The contaminant will dissolve into and evaporate from water and the transport will partially occur through water.
Liquid diffusion coefficients are usually around four orders of magnitude smaller than in air.

The soil moisture content of soils can be estimated in many ways, but two common approaches is to use the analytical formulas of \textit{van Genuchten} or \textit{Brooks and Corey}.
Both of these formulas give the soil moisture content as a function of the fluid pressure head, $H_p$.
By definition, when the pressure head is equal to or greater than zero, $H_p \geq 0$, the soil is assumed to be 100\% saturated with the fluid.
In this work, \textit{van Genuchten's} formula is used.

The soil moisture content, $\theta$ is given by.
\begin{equation}
  \theta = \begin{cases}
    \theta_r + \mathrm{Se}(\theta_s - \theta_r) & H_p < 0 \\
    \theta_s & H_p \geq 0
\end{cases}
\end{equation}

The saturation is given by.
\begin{equation}
  \mathrm{Se} = \begin{cases}
    \frac{1}{(1 + |\alpha H_p|^m)^m} & H_P < 0 \\
    1 & H_p \geq 0
  \end{cases}
\end{equation}

\begin{equation}
  C_m = \begin{cases}
    \frac{\alpha m}{1-m}(\theta_s - \theta_r)\mathrm{Se}^{\frac{1}{m}}\big( 1 - \mathrm{Se}^{\frac{1}{m}} \big)^m & H_p < 0 \\
    0 & H_p \geq 0
  \end{cases}
\end{equation}

\begin{equation}
  k_r = \begin{cases}
    \mathrm{Se}^l \big[ 1 - \big( 1 - \mathrm{Se}^\frac{1}{m} \big) \big]^2 & H_p < 0 \\
    0 & H_p \geq 0
  \end{cases}
\end{equation}


\subsection{Vapor Transport in Unsaturated Porous Media}\label{sec:darcys}

Fluid transport in porous media is governed by \textit{Darcy's Law} and was originally formulated by Henry Darcy based on his work on describing water flow through soil under the influence of gravity.
Since then it been found to derivable in several ways from the Navier-Stokes equations\cite{bear_dynamics_1972} and may be stated as a pressure gradient driven velocity.
\begin{equation}\label{eq:darcys_law_saturated}
  \vec{u} = -\frac{\kappa}{\mu}\nabla p
\end{equation}
Here $\vec{u}$ is the fluid velocity; $\kappa$ the soil permeability; $\mu$ is the fluid viscosity; and $\nabla p$ is the pressure gradient.
In VI modeling we're interested in the flow of contaminant vapors but since the contaminant concentrations are typically very low, the transport properties may be taken from those of pure air.\par

% TODO: Elaborate on the assumptions, i.e. that in DL the viscosity shear effects may be neglected. But this breaks down when Re is too high.

For Darcy's Law to be valid, two assumptions must be fulfilled:
\begin{enumerate}
  \item The fluid must be in the laminar regime, typically $\mathrm{Re} < 1$.
  \item The soil matrix must be saturated with the fluid.
\end{enumerate}
Typically the vapor flows in most VI scenarios are sufficiently slow for the first condition to be fulfilled.
And if they are not, there are modifications to Darcy's Law that
Most of the contaminant vapor transport takes place in the partially saturated vadose zone and thus, \eqref{eq:darcys_law_saturated} needs modification.\par

In partially saturated soils, a varying portion of the soil pores are available for vapor transport, with the rest being occupied by water, affecting the effective permeability of the soil.
To model this, we use the relative permeability property, $k_r$, from section \ref{sec:richards} is used.
\begin{equation}
  \kappa_\mathrm{eff} = (1-k_r) \kappa_s
\end{equation}
Note that in this Darcy's formulation $(1 - k_r)$ is used to refer to the relative permeability of vapor, e.g. that 0 indicates the soil is completely impermeable for vapor flow (and vice versa).\par
This gives the modified Darcy's Law used in VI-modeling:
\begin{equation}\label{eq:darcys_law}
  \vec{u} = -\frac{(1-k_r) \kappa_s}{\mu}\nabla p
\end{equation}

However, \eqref{eq:darcys_law} only gives the vapor velocity as a function of the pressure gradient, to properly model the vapor flow in the soil matrix we need to incorporate Darcy's Law into a continuity equation giving
\begin{equation}\label{eq:vapor_transport}
  \frac{\partial}{\partial t} (\rho \epsilon) + \nabla \cdot \rho \Big( -\frac{(1-k_r) \kappa_s}{\mu} \nabla p \Big) = Q_m
\end{equation}
which is governing equation for vapor flow in porous media.\par

\paragraph{Boundary conditions}

In order to solve \eqref{eq:vapor_transport} we need to define some boundary conditions.
In our CSM, air is pulled from the atmosphere through the ground surface and into the building via the foundation crack.
To model this only three boundary conditions are required.\par

The first is to define a pressure gauge, i.e. a reference point for where the pressure is zero, which is where air will be pulled from.
This is the applied to the ground surface boundary.
The second is that we apply the indoor/outdoor pressure difference (~5 Pa) to the foundation crack boundary.
The third type is applied to all remaining boundaries and is a no flow boundary condition, indicating that no flow passes through these boundaries.
We also make sure that we specify the symmetry planes present.
\begin{align}
  &\text{Ground surface} &p = 0 \; \mathrm{(Pa)} \\
  &\text{Foundation crack} &p = p_\mathrm{in/out} = -5 \; \mathrm{(Pa)} \\
  &\text{Remaining} &-\vec{n}\cdot\rho\vec{u} = 0
\end{align}
where $\vec{n}$ is the boundary normal vector.\par

\documentclass{article}

\usepackage{amsmath}
\usepackage[utf8]{inputenc}


\begin{document}

\section{Mass Transport in Partially Saturated Porous Media}

The vadose zone is a three-phase system and thus any chemical specie is distributed between these three phases.
However, the mass transport still only occur through the gas and liquid phases of the system, therefore the transport of the \textit{total} concentration $c_T$ is due to diffusive and advective transport in these phases.
\begin{equation}\label{eq:mass1}
  \frac{\partial c_T}{\partial t} = \nabla [D_w \theta_w \tau_w \nabla c_w + D_g \theta_g \tau_g \nabla c_g]
    - \nabla (v_w \theta_w c_w + v_g \theta_g c_g)
\end{equation}
Here $D_w$ and $D_g$ are the water and gas diffusion constants respectively; $\theta_w$ and $\theta_g$ are the water and gas filled porosities; $\tau_w$ and $\tau_g$ are the tortuosity, correcting for diffusivity in porous media; $v_w$ and $v_g$ are the water and gas velocity; finally $c_w$ and $c_g$ are the water and gas phase concentrations.\par

As stated above, the total concentration is distributed across the three phases.
\begin{equation}
  c_T = \theta_w c_w + c_g \theta_g + c_s \rho_b
\end{equation}
Where the first and second terms correspond to the water and gas concentrations; the third correspond to the sorbed concentration, where $c_s$ is the sorbed concentration by mass and $\rho_b$ is the soil bulk density.
In order to solve \eqref{eq:mass1} we need to state everything in terms of one dependent variable, which we will see is the water concentration $c_w$.

From Henry's Law we know that a gas concentration is proportional to the water concentration via the eponymous constant $H$.
\begin{equation}
  c_g = H c_w
\end{equation}
By assuming linear sorption we can describe the sorbed concentration as
\begin{equation}
  c_s = \begin{cases}
    K_p c_w &\text{Water phase sorption} \\
    K_p c_g = K_p H c_w &\text{Gas phase sorption}
  \end{cases}
\end{equation}
where $K_p$ is the sorption isotherm.
For simplicity we will here assume water phase sorption.

Using this we can restate $c_T$ is terms of the water phase concentration.
\begin{equation}
  c_T = (\theta_w + \theta_g H + K_p \rho_b) c_w = R c_w
\end{equation}
The \textit{retardation factor} $R$ is introduced to simplify writing.

Now we substitute all of this in \eqref{eq:mass1}.
\begin{equation}\label{eq:mass2}
  R \frac{\partial c_w}{\partial t} = \nabla [(D_w \theta_w \tau_w + D_g \theta_g \tau_g H)\nabla c_w]
    - \nabla [(v_w \theta_w + v_g \theta_g H) c_w]
\end{equation}
Here we recognize that $(D_w \theta_w \tau_w + D_g \theta_g \tau_g H)$ is the effective diffusivity $D_\mathrm{eff}$, which gives the final expression
\begin{equation}\label{eq:mass3}
  R \frac{\partial c_w}{\partial t} = \nabla [D_\mathrm{eff} \nabla c_w]
    - \nabla [(v_w \theta_w + v_g \theta_g H) c_w]
\end{equation}

Most soil-physics books are concerned with water moving in porous media, with the gas assumed to be immobile and occupy small pockets in the porous media.
In this case $v_g = 0$, dropping that term which gives
\begin{equation}\label{eq:mass4}
  R \frac{\partial c_w}{\partial t} = \nabla [D_\mathrm{eff} \nabla c_w]
    - \nabla [v_w \theta_w c_w]
\end{equation}
This is the most common form found of the governing equation for mass transport in partially saturated porous media, and the equation that COMSOL solves.

Obviously this does not quite describe the vapor intrusion scenario, where we are concerned with a mobile gas phase and a stationary water phase.
Although, this does not necessarily have to be the case, and we could in theory keep both velocity fields if we were interested in such an problem.
Regardless, for most of our applications we assume that the soil water is stationary $v_w = 0$ leading to
\begin{equation}\label{eq:mass5}
  R \frac{\partial c_w}{\partial t} = \nabla [D_\mathrm{eff} \nabla c_w]
    - \nabla [v_g \theta_g H c_w]
\end{equation}
The implications of this is that we must multiply the gas velocity field with the Henry's Law constant to correctly reflect the transport problem.

Another implication is that \textbf{we must set all our boundary conditions in terms of the water phase concentration} $c_w$.
So for one, the concentration boundary condition at the groundwater source must be the groundwater \textit{water} concentration, and not the typical one where we multiply it by $H$; as we've seen, this previous correction is built into the governing equation.

The crack entry flux $j_{ck}$ must also be adjusted.
This one is a bit trickier though, since we're only concerned the gas phase concentration entering through the crack.
Thus it must be stated as a function of the gas phase concentration, i.e. $j_{ck} = f(c_g)$ and this is the contaminant flux that enters the overlying building.
But since we must state every boundary condition in terms of $c_w$, we must scale the boundary condition in the model using Henry's Law as well, and thus it should be $j_{ck}/H$.

\end{document}

