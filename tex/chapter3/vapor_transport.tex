\subsection{Vapor Transport in Unsaturated Porous Media}\label{sec:darcys}

Fluid transport in porous media is governed by \textit{Darcy's Law} and was originally formulated by Henry Darcy based on his work on describing water flow through soil under the influence of gravity.
Since then it been found to derivable in several ways from the Navier-Stokes equations\cite{bear_dynamics_1972} and may be stated as a pressure gradient driven velocity.
\begin{equation}\label{eq:darcys_law_saturated}
  \vec{u} = -\frac{\kappa}{\mu}\nabla p
\end{equation}
Here $\vec{u}$ is the fluid velocity; $\kappa$ the soil permeability; $\mu$ is the fluid viscosity; and $\nabla p$ is the pressure gradient.
In VI modeling we're interested in the flow of contaminant vapors but since the contaminant concentrations are typically very low, the transport properties may be taken from those of pure air.\par

% TODO: Elaborate on the assumptions, i.e. that in DL the viscosity shear effects may be neglected. But this breaks down when Re is too high.

For Darcy's Law to be valid, two assumptions must be fulfilled:
\begin{enumerate}
  \item The fluid must be in the laminar regime, typically $\mathrm{Re} < 1$.
  \item The soil matrix must be saturated with the fluid.
\end{enumerate}
Typically the vapor flows in most VI scenarios are sufficiently slow for the first condition to be fulfilled.
And if they are not, there are modifications to Darcy's Law that
Most of the contaminant vapor transport takes place in the partially saturated vadose zone and thus, \eqref{eq:darcys_law_saturated} needs modification.\par

In partially saturated soils, a varying portion of the soil pores are available for vapor transport, with the rest being occupied by water, affecting the effective permeability of the soil.
To model this, we use the relative permeability property, $k_r$, from section \ref{sec:richards} is used.
\begin{equation}
  \kappa_\mathrm{eff} = (1-k_r) \kappa_s
\end{equation}
Note that in this Darcy's formulation $(1 - k_r)$ is used to refer to the relative permeability of vapor, e.g. that 0 indicates the soil is completely impermeable for vapor flow (and vice versa).\par
This gives the modified Darcy's Law used in VI-modeling:
\begin{equation}\label{eq:darcys_law}
  \vec{u} = -\frac{(1-k_r) \kappa_s}{\mu}\nabla p
\end{equation}

However, \eqref{eq:darcys_law} only gives the vapor velocity as a function of the pressure gradient, to properly model the vapor flow in the soil matrix we need to incorporate Darcy's Law into a continuity equation giving
\begin{equation}\label{eq:vapor_transport}
  \frac{\partial}{\partial t} (\rho \epsilon) + \nabla \cdot \rho \Big( -\frac{(1-k_r) \kappa_s}{\mu} \nabla p \Big) = Q_m
\end{equation}
which is governing equation for vapor flow in porous media.\par

\paragraph{Boundary conditions}

In order to solve \eqref{eq:vapor_transport} we need to define some boundary conditions.
In our CSM, air is pulled from the atmosphere through the ground surface and into the building via the foundation crack.
To model this only three boundary conditions are required.\par

The first is to define a pressure gauge, i.e. a reference point for where the pressure is zero, which is where air will be pulled from.
This is the applied to the ground surface boundary.
The second is that we apply the indoor/outdoor pressure difference (~5 Pa) to the foundation crack boundary.
The third type is applied to all remaining boundaries and is a no flow boundary condition, indicating that no flow passes through these boundaries.
We also make sure that we specify the symmetry planes present.
\begin{align}
  &\text{Ground surface} &p = 0 \; \mathrm{(Pa)} \\
  &\text{Foundation crack} &p = p_\mathrm{in/out} = -5 \; \mathrm{(Pa)} \\
  &\text{Remaining} &-\vec{n}\cdot\rho\vec{u} = 0
\end{align}
where $\vec{n}$ is the boundary normal vector.\par
