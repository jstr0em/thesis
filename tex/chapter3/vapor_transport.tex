Vapor transport in porous media is described by \textit{Darcy's Law}.
The vapor velocity depends on the pressure gradient in the soil, is proportional to the permeability of the soil matrix, and is inversely proportional to the viscosity of the fluid.

\begin{equation}\label{eq:darcys_law_saturated}
  \vec{u} = -\frac{\kappa}{\mu}\nabla p
\end{equation}

For Darcy's Law to be valid, two assumptions must be fulfilled:
\begin{enumerate}
  \item The fluid must be in the laminar regime, typically $\mathrm{Re} < 1$.
  \item The soil matrix must be saturated with the fluid.
\end{enumerate}
In VI-modeling, the first assumption is fulfilled, but the second is not.
Most of the contaminant vapor transport takes place in the partially saturated vadose zone and thus, \eqref{eq:darcys_law_saturated} needs modification.

In partially saturated soils, a varying portion of the soil pores are available for vapor transport, with the rest being occupied by water, affecting the effective permeability of the soil.
To model this, a relative permeability property, $k_r$, is introduced:
\begin{equation}
  \kappa_\mathrm{eff} = k_r \kappa_s
\end{equation}
$k_r$ is a dimensionless parameter that varies between 0 and 1, and $\kappa_s$ is the saturated, or simply the soil matrix permeability.

This gives the modified Darcy's Law used in VI-modeling:
\begin{equation}\label{eq:darcys_law}
  \vec{u} = -\frac{k_r \kappa_s}{\mu}\nabla p
\end{equation}
