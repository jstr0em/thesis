\section{Geometry}

Designing the model geometry is the first step to creating a 3D FEM model.
It is one of the most important steps, as the geometry will dictate the model accuracy and astute geometry design will help save computational resources.
When designing a geometry the FEM user should have the following goals in mind:
\begin{enumerate}
  \item Represent the model geometry as accurately as possible.
  \item Avoid unnecessarily fine details.
  \item Try to leverage symmetry to reduce geometry size.
\end{enumerate}\par

The first point is somewhat self-explanatory, as we obviously want to create a model geometry that is as similar to what we want to model as possible.
The second points can at times run counter to the first and may be more self-evident once meshing is more thoroughly discussed.
Tiny details often require a significant number of mesh elements to be fully resolved, disproportionally adding to the total number of mesh element, and may significantly increase computational costs.
This is when the skill and judgement of the modeler comes in - choosing which details to omit and which to keep.
As a rule-of-thumb one should for the most part try to only model parts of the geometry that is of significant value to the question that one wants answered.
In VI modeling, one such obvious area is the crack or breach in the foundation through which contaminant vapors enter the structure, resolving this tiny part of the geometry is of great importance.\par

The third point is something that the modeler should always be on the lookout for when designing a model geometry - if there are any planes of symmetry in the geometry.
Finding a plane of symmetry allows us to reduce the size of the model and save significantly computational costs.
A simple example of this is one wants to model a pipe with static mixers inside, then only a sector of the cylinder's face may be necessary to be modeled.
Using the simple CSM described by Figure \ref{fig:csm} only a quarter of the house and surrounding property is necessary to be explicitly modeled, cutting the number of required mesh elements down to just a quarter of what would otherwise be necessary - a huge computational saving!\par

\subsection{Geometric Components}

Model geometries are typically designed in some sort of computer assisted design (CAD) software.
The exact tools and techniques available to the modeler will vary from software to software, with some featuring import options for real-world scanned 3D geometries to combining simple geometric objects through various Boolean operations.
The software we use, COMSOL, uses primarily the latter method of combining simple objects to form more complicated ones but more capabilities may be purchased.\par

To create a model geometry of the CSM in Figure \ref{fig:csm}, only a few simple geometric objects and Boolean operations are required - two cuboids, two rectangles, one Boolean difference operation, and one Boolean join operation.
The following steps are needed:
\begin{enumerate}
  \item Create a block or cuboid that is 15 meter wide and long, and with a height of 4 meter.
  \item Create another block that is 5 meter wide and long, with a height of 1 meter.
  \item Place the second block 3 meter above zero, so that the top surfaces of the two blocks intersect.
  \item Perform a difference operation, removing the smaller block from the first one.
\end{enumerate}
At this point you will see that a quarter soil domain has been created, with an empty space that will represent a house with a foundation slab located 1 meter below ground-surface.\par

The foundation crack will be modeled as a 1 centimeter wide strip that spans the perimeter of the surface that represents the house foundation.
To create the crack do the following:
\begin{enumerate}
  \item Create a work plane 3 meter above zero.
  \item Create two rectangles that are as long as the foundation, with a width of 1 centimeter, rotating one 90 degrees, and making sure that they are place along the foundation perimeter.
  \item Join the two rectangles using a Boolean union (do not keep the interior boundaries).
\end{enumerate}
Now that the foundation crack is generated, we have designed a model geometry of the simple CSM and the complete geometry may be seen in Figure \ref{fig:geometry}.
The next step is to choose and setup the appropriate physics required to model VI, beginning with modeling the indoor environment.\par

% TODO: Create figure
\begin{figure}[htb!]
  %\includegraphics{}
  \caption{The complete geometry of the CSM described in Figure \ref{fig:csm}.}
  \label{fig:geometry}
\end{figure}

% TODO: Add references to specific appendices
In the appendix, there will be further explanations for additions to the model geometry that will be necessary for modeling various VI scenarios.\par
% TODO: Make list of the types of geometries that can be seen in the appendix (or examples at least).
