Preferential pathways have recently been recognized for the significant role that they may play in enhancing exposure after their discovery at some well-studied VI sites.
The nature and specific effect of a preferential pathway can vary greatly and is largely site specific; generalizing their impact is therefore difficult.
One of these well-studied sites in Layton, Utah however, has a preferential pathway that had a very significant impact, which offers us an opportunity to explore this case in detail.
Through our numerical model and data analysis we are able to reveal how the preferential pathway at this site both enhanced the contaminant availability and advective potential which caused its significant impact.
We also examine the impact of the preferential pathway at another well-studied site in Indianapolis, Indiana, and compare that case to the one in Layton, and show that for seemingly relatively similar conditions, a preferential pathway may have very different results.
