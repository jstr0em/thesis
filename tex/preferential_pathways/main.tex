\documentclass[../thesis.tex]{subfiles}

% graphics path
\graphicspath{
  {../figures/preferential_pathway/},
  {../../figures/preferential_pathway/}
}

% TODO: Hers et al. 2014 is an example of diffusion limited transport. Use it!

\begin{document}
\chapter{Preferential Pathways: Drivers Of Temporal And Spatial Variability In Vapor Intrusion}

\import{./}{abstract.tex}
\import{./}{intro.tex}
\import{./}{modeling.tex}
\import{./}{kriging.tex}
\import{./}{results.tex}
\import{./}{results2.tex}


\begin{comment}

Introduction:

Temporal variability issue in VI.
ASU and Indie houses purchases to deal investigate aspects of this.
Both sites revealed PPs.

Examples of PPs:
- Leaky pipe & long-distance sewer transport - Indie
- Land drain - ASU house
- Sewer type
  - Pennell Boston version
  - Danish study

ASU house in particularly interesting because of the closing of PP/LD.
- What happened?
  - Reduction in variability (figure)
  - Reduction is sub-surface spatial variability (figure)
  - Change in relationship between p_in and c_in (figure)
- (3) allows us to mechanistically investigate what happened with models. Answer questions like:
  - Why the change in p_in and c_in relationship?
  - Which conditions give rise to the significant effect of a PP of this kind?

p_in and c_in analysis at Indie shows a quite different (weaker) relationship (figure)
- Why?
  - Harder to model as we do not know as much of the PP
  - Kriging of soil-gas data to give an idea of what went on.

Method:

ASU house:
- Site description and subsequent model development (figure)
  - Gravel sub-base
  - Motivation for PP BCs
    - Groundwater source concentration in pipe
      - Infiltration (depth of sewer in relation to groundwater + happened at Indie)
      - Guo manhole measurements
    - Pressure gauge/air source (just simplest approach and many approaches could have been taken)

Kriging:
- I need to review this...

Results:

ASU house:
- PP increased advective potential. Conditions:
  - Contaminated PP
  - Gravel sub-base
  - Air source

Indie:
- A bit unclear, but Kriging shows:
  - Sewer likely leaks somewhere under the front lawn.
  - May explain the different behavior, i.e. the distance separation is a middle-ground between ASU pre and post closing of LD.

Conclusion:

Screening for PPs should be routine part of VI investigations.

Examples:
- Danish suggestions (sample in sewers/manholes and behind water traps)
- CPM could be useful
  - Combined with covering manhole to see if the building communicates with the sewer

\end{comment}

\end{document}
