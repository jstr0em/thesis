\section{Summary}

Preferential pathways have recently been recognized for the significant role that they may play in enhancing vapor intrusion (VI).
The nature and specific effect of a preferential pathway can vary greatly and is largely site specific; generalizing their impact can therefore be difficult.
Two well-studied VI sites were revealed to be impacted by preferential pathways, providing an excellent opportunity to explore their influence.
One of these sites was in Layton, Utah, which featured a preferential pathway that had a very significant impact on, among other things, but in particular, the temporal variability of indoor contaminant concentration.
This preferential pathway was at a later date closed, and the comparison between the periods before and after the closing offers some unprecedented opportunities for understanding preferential pathways.
A preferential pathway similar to that one will be implemented in our numerical model, and through analysis of data from the site, will offer insights how preferential pathways can enhance VI.
The implications of these insights have wider consequences for VI investigations in general, which is further explored in Chapter \ref{chp:transport_implications}.
Another well-studied site in Indianapolis, Indiana, was also affected by a preferential pathway, but its role and influence was very different from the Layton site, and in particular played a significant role in the spatial variability of subsurface contaminant concentrations, a topic which will explored using data from that site, as well as from the Layton site.
