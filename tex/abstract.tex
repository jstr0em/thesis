\begin{abstract}
% What is the problem?
Vapor intrusion (VI) investigations, the effort to determine the exposure and associated human health-risk at a VI impacted building, are often complicated by significant spatial and temporal variability in concentrations of contaminants of concern.
Over the years there have been efforts to develop new techniques and methodologies that aim to reduce the uncertainties associated with these variabilities.
The goal is to simplify and improve the robustness of VI site investigations.
The development of the controlled pressure method (CPM), where the pressurization of a building is controlled in an effort to increase or decrease contaminant entry into the building, is one such example.
Another approach is to use indicators, tracers, and surrogates (ITS) to help guide when to conduct site investigations, ideally increasing the likelihood of determining the maximum indoor contaminant concentrations.\par

% Why is the problem hard?
Both of these approaches rely on a quasi-deterministic relationship between some external variable, such as building pressurization, and indoor contaminant concentration.
However, site-specific conditions can give rise to very different responses to such an external variable.
To effectively use CPM or ITS, a better mechanistic understanding of contaminant transport and exposure is needed.\par

% What is my approach to solving this problem?
In this thesis, we develop three-dimensional finite element models of VI impacted buildings from a first principles perspective.
These models combined with analysis of field data from VI sites, allows us to explore the physical mechanisms that drive VI.
By considering the dominant contaminant transport mechanism at a site, e.g. if advective or diffusive transport dominates, we can explain why a change in building pressurization can lead to differences in contaminant concentration variability at different sites.
We can also better understand how the various factors governing VI contribute to the overall variability.\par

% What are the consequences of my approach
By classifying the dominant contaminant transport mechanism at a site, we can more effectively anticipate how a particular site will respond to some external stimuli.
This will in turn reduce the effort required to, and increase the robustness of the techniques used determine the relevant human exposure at a VI site.\par
\end{abstract}
