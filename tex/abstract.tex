\documentclass[../thesis.tex]{subfiles}

\begin{abstract}


% What is the problem?
Vapor intrusion (VI) investigations, the effort to determine the exposure and associated human health-risk at a VI impacted building, are often complicated by significant spatial and temporal variability in VI.
Over the years there have been efforts to develop new techniques and methodologies that aim to reduce the issues associated with these variabilities; simplifying and improving the robustness of VI site investigations.
The development of the controlled pressure method (CPM), where the pressurization of a building is controlled to increase or decrease contaminant entry into the building, is one such example.
Another approach is to use indicators, tracers, and surrogates (ITS) to help guide when to conduct site investigations, ideally increasing the likelihood of determining the higher indoor contaminant concentrations.\par

% Why is the problem hard?
Both of these approaches rely on a quasi-deterministic relationship between some external variable, such as building pressurization, and indoor contaminant concentration.
However, site-specific conditions can give rise to very different responses to such an external variable, and to effectively use CPM or ITS, a better mechanistic understanding of contaminant transport and exposure are needed.\par

% What is my approach to solving this problem?
In this thesis, we develop three-dimensional finite element models of VI impacted buildings from a first principles perspective.
These models combined with analysis of field data from VI sites, allows us to explore the physical mechanisms that drive VI.
By considering the dominant contaminant transport mechanism at a site, e.g. if advective or diffusive transport dominates, we can explain why a change in building pressurization can lead to such differences in contaminant concentration at different sites.
We can also better understand how various factors in VI contribute to the overall variability.\par

% What are the consequences of my approach
By classifying the dominant contaminant transport mechanism at a site, we can more effectively anticipate how a particular site will respond to some external stimuli.
This will in turn reduce the effort required to, and increase the robustness of the techniques used determine the relevant human exposure at a VI site.\par

\begin{comment}

% OLD
The significant temporal variability of vapor intrusion (VI) complicate site investigations and increase the chances of over- or underestimating human exposure to indoor air contaminants.
Conducting longer and more temporally high-resolution studies of suspected VI sites is one way to address this issue.
However, this approach is undesirable as these types of studies can be more technically difficult and exceedingly costly.
Improving our understanding of the factors that influence this temporal variability is therefore essential for accurate assessment of human exposure to these contaminants.\par

This is challenging due to the complex nature of VI and the heterogeneity of VI sites, leading to highly non-linear and often unique systems.
Adding to this is the fact that only a few long-term and high-resolution VI site studies exist, further rendering a robust understanding of the drivers the temporal variability in VI elusive.
In response to this, techniques such as a the controlled pressure method or the use of seasonal correction factors based on statistical analysis of VI sites, have been developed to help heuristically determine the relevant human exposure.
But none of these circumvent the need for a careful understanding of the drivers of VI, without which new mischaracterization may be introduced.\par

By implementing the physics that govern VI from first-principles in three-dimensional finite element models, we gain complete control of the VI phenomena.
This allows us to investigate the role of individual factors and site characteristics that influence VI.
Using these models in combination with statistical analysis of the aforementioned VI high-resolution datasets, we gain an opportunity to investigate the highly dynamic effects of building pressurization and sorption on transient variability in VI.
As well as how these dynamic processes interplay with external weather factors in the short-term and in the long-term.\par

% TODO: Add paragraph 4
% -- What are the consequences of your choice/thesis?
% - What is the result/impact?
% - What does this enable?
\end{comment}

\end{abstract}
