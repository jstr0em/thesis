\section{Conclusions}\label{sec:conclusions}

The work presented here offers new insights into how sorption effect may influence VI; exploring how sorption affects VI fundamentally and in some applications.
Based on what has been presented here, we can draw some conclusions.
\begin{itemize}
  \item The sorption and desorption kinetics, while they vary quite a bit, do not seem to make a significant difference in most VI applications or cases, at least not the ones considered here. But, what has been apparent is that it is the sorption capacities of the particular material that matters, although these seem to have to be on the same scale as cinderblock to truly be significant.
  \item The sorptive capacities of materials vary significantly and it may not be obvious which ones have large sorptive capacities and which do not. More materials, in combination with other contaminants, need to be tested to gain a more comprehensive understanding of which materials are of particular concern in VI.
  \item Sorption fundamentally retards contaminant transport in soils, effectively increasing the residence time of the contaminant in the soil. However, sorption will only start to retard transport if the sorption partition coefficient is large enough to exceed the "naturally" induced residence by the soil moisture content. Sorption on materials in the indoor environment functions in a similar way, and like a capacitance, inhibits changes in indoor concentration. These particular phenomena may be relevant to consider if one wants to try to influence the contaminant transport and entry rate into the building via forceful pressurization of the building, e.g. the controlled pressure method, as these attempts may be less effective than expected due to sorption.
  \item Contaminant desorption from indoor materials may also have a significant effect of mitigation systems, potentially delaying effective mitigation from taking place from a matter of hours to several days or weeks, depending on the amount and kind of indoor materials present.
\end{itemize}
