\section{Summary}

Many vapor intrusion (VI) contaminants have the capacity to sorb onto a variety of materials commonly found in buildings as well as soils, yet the role and effect of sorption in VI is largely unstudied.
To bridge this gap we measure the sorptive capacities of trichloroethylene (TCE) on some materials at VI relevant concentrations; finding that material sorptive capacities vary orders of magnitudes, with cinderblock having a capacity to hold up to almost 41,000 times more contaminant than a comparable TCE contaminated air volume.
Using these experimentally derived data together with a three-dimensional numerical model of VI, we then explore the retarding effect that sorption has on contaminant transport in soils and indoor environments.
We also apply the model to investigate how the contaminant desorption from these materials, following the implementation of a successful VI mitigation scheme, affect contaminant expulsion.
We find that desorption may cause significant delay: in some cases taking months longer than if there were no sorbed contaminants.
