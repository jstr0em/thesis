\section{Introduction}\label{chp4:intro}

Early on in the history of VI investigations, it was discovered that many VI sites had significant temporal variation of indoor air contaminant concentration.
This variability presented a problem for all concerned, as it made it more difficult to assess the relevant human exposure at a site.
Of particular concern was the discovery that some sites were characterized by active and inactive periods.
This meant that it was possible for a VI investigation to give a false-positive result, further complicating things.\par

To address the growing concerns of temporal variability in VI, the EPA, together with Arizona State University (ASU), purchased a VI impacted building near Hill Air Force Base (AFB) in Utah.
The purpose of this was to conduct a long-term high-resolution scientific inquiry, with which the rich dataset would reveal the nature of these temporal variabilities, and offer unprecedented insights into the VI phenomenon.
Another key objective was to assess the viability of the controlled pressure method (CPM) as an investigative tool (but more on that later).\par

The house in question, henceforth referred to as the ASU house, was outfitted a wide array of instrumentation to monitored a slew of factors deemed pertinent for VI, e.g. building pressurization, weather factors, air exchange rate, and much more.
Of most significance was the measuring of contaminant concentrations inside the building, but also in the soil-gas, and contaminated groundwater below.
Many soil-gas and groundwater probes were placed at different locations and depths - detailing how the contaminant was distributed throughout the soil and groundwater around the house.
Indoor air sampling also took place in different parts of the building.\par

The measurements, in particular indoor air contaminant concentration measurements, revealed what was perhaps one of the most significant temporal variabilities recorded at a VI site.
Figure \ref{fig:asu_indoor_concentrations} shows this variability across the entire x year study period. % TODO: Add how many years it actually was (roughly)?
As can be seen, orders of magnitude of variability on both short- and long-term timescales were recorded at the site, causing even more concern about the nature of the variability.\par

\begin{figure}[htb!] % TODO: Make figure and show research phase - CPM, no CPM etc
  %\includegraphics[width=\textwidth]{asu_indoor_concentrations.pdf}
  \caption{The temporal variability of indoor air contaminant concentrations recorded at the ASU house. Measurements were taken in the basement. }
  \label{fig:asu_indoor_concentrations}
\end{figure}

The CPM was employed during the middle stages of the study, to see if and how the temporal variability would be reduced, this period is shown in Figure \ref{fig:asu_indoor_concentrations}.
This had the desired effect and significantly reduced the variability, keeping the indoor air contaminant concentration relatively steady at the higher concentration observed during the "natural" or no-CPM phases.
While this was a success, it didn't explain why there was so much variability during the non-CPM periods.\par

It was not until the latter part of the CPM study phase that a land drain underneath the house was uncovered.
The purpose of this land drain was to drain any water from the gravel filled sub-slab into the sewer (which it was connected to).
The researchers dug out the land drain and fitted a valve on it, allowing them to turn it's influence on and off.\par

The land drain was first turned off during the end of CPM phase of the study, dramatically decreasing the variability


% TODO: 2 pane subplot.
% Left pane: Maybe zoom in to the CPM period here with a time series figure.
% Right pane: Boxplot of CPM and no CPM periods concentations


\begin{figure}
  %\includegraphics[width=\textwidth]{}
  \caption{}
  \label{}
\end{figure}
