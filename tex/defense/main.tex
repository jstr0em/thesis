\documentclass{beamer}
\usepackage[utf8]{inputenc}
\usepackage{import}
\usepackage{caption}
\usepackage{graphicx}
\usepackage{comment}
\usepackage{graphicx}
\usepackage{nth}
\usepackage{siunitx}
\usepackage[style=numeric,sorting=none]{biblatex}
\addbibresource{../references.bib}

\graphicspath{
  {../../figures/introduction/},
  {../../figures/preferential_pathways/},
  {../../figures/defense/},
  {../../figures/methods/},
  {../../figures/transport_implications/},
}

% beamer settings
\beamertemplatenavigationsymbolsempty

% title slide
\title{Understanding the Dynamics of Vapor Intrusion Processes}
\subtitle{Through Numerical Modeling}
\author{Jonathan Ström}
\institute{Brown University \\ School of Engineering}
\date{May 2020}

\begin{document}

\frame{\titlepage}

\import{./}{intro.tex}
\import{./}{modeling.tex}
\import{./}{preferential_pathways.tex}

\begin{frame}
  \centering
  \huge Thank you!
\end{frame}

\begin{frame}[allowframebreaks]
\printbibliography
\end{frame}

\import{./}{transport.tex}

\begin{comment}
\AtBeginSection[]
{
  \begin{frame}
    \frametitle{Table of Contents}
    \tableofcontents[currentsection]
  \end{frame}
}



Goal: ~20 slides with real content. 35-40 minute presentation.
~5 for introduction
~2 for modeling
~5 for ASU house & CPM analysis
~3 for advection/diffusion
~2 for ITS and pressure

-: Perspective on indoor air quality & radon
-: VI introduction
-: VI investigations
-: Attenuation factors (their uses) and VI investigation empiricism problems
-: Numerical modeling in VI
  -- VI investigations highly empirical
  -- Difficult to understand due to complexity of sites
  -- First-principles modeling allows us to unravel this complexity (sensitivity analysis) & investigate physical phenomena
-: Background of VI modeling
  -- Analytical models, i.e. J&E
    -- Cannot be modified - of limited use
  -- Numerical models
    -- More generalizable, but can be more difficult to use
    -- A few have been developed but not well-used in VI investigations
    -- Not used in a transient setting
-: Use models to explore in particular temporal variability


-: Examining ASU house and CPM (and issues with the assumptions)
-: ASU house background
-: CPM
-: ASU house model
-: Replicating ASU results
-: Introducing

-: Which site conditions give rise to advective dominated entry?
-: Soil and foundation types

-: ITS (issues with ITS approach, similar to )

-: Predicting indoor contaminant concentration using weather as ITS (proxy for pressure)




What do I want to say?

How does my work contribute to the scientific understanding of VI?
- Numerical modeling offers the ability to gain key physical insights of the VI process at a site that typical empirical approach to VI site investigations are unlikely to discover.
- Through this effort we explore how by determining if contaminant entry into a building is characterized by advection or diffusion can be crucial for assessing VI risk.

What is the empirical approach in VI and what are the issues with it?
- Simple data collection and analysis with no real mechanistic understand of VI process at a site
- Illustrated by how the EPA uses attenuation factors


Why should we be concerned with VI?
- Prevalence of the problem (or potential anyway)
- Health risks (carcinogenic)
- TCE and PCE common examples

What is VI?
- Explain using graphics
- Use Radon story to tell how much can accumulate in indoor environment

How do modeling help us understand VI?
- Developed from first-principles and can control the physics (keep variables fixed etc)
- Used early on, developed from Radon models
- Early models analytical and limited
  - Steady-state
  -

What do I mean by CSM and how do they relate to model development?
\end{comment}

\end{document}
