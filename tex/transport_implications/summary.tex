\section{Summary}

This chapter explores the importance of dominant transport mechanism for contaminant vapors entry into a building.
There are two commonly accepted mechanisms by which contaminants enter a building from the soil beneath it – advection and diffusion.
In the case of advection, contaminant is carried into a building as a constituent of soil gas, which is swept into the building by a pressure gradient from soil to building interior.
Such gradients need not be large and a few \si{\pascal} can drive such a flow.
On the other hand, if the interior of the structure is pressurized relative to the sub-foundation soil, then the advective flow will be out of the building.
Regardless of the existence or direction of a pressure gradient, there generally exists a contaminant concentration gradient from soil into the structure, which means that there will always be a driving force for diffusive entry of the contaminant into the building.
Only if the outward advective flow is high enough can this diffusive entry be overcome, and contaminant entry prevented.
Advective transport is likely to only be dominant at VI site with some feature or pathway that allows facile delivery of air (soil-gas) the entry point in the foundation.
This is because there is generally a large resistance to transport of soil gas through the surrounding soil, bearing in mind that the pressure gradients available to drive any flows are normally quite small.
It has been shown in previous modeling studies that for quite comparable levels of contaminant vapor beneath the foundation slab, very different indoor air contaminant concentrations may be encountered, depending upon the ability of the surrounding soil to allow significant advective flow\cite{bozkurt_simulation_2009,pennell_development_2009}.
This fact emphasizes the importance in vapor intrusion investigations of identifying features that will allow significant advective flow from soil to the building interior.
When advective entry is the dominant mode of contaminant entry into a building, use of building pressurization (in a controlled pressure method, (CPM) study) as an investigative tool is likely to offer more definitive results than when diffusive entry is dominant.
The existence of an advective entry dominated scenario also may enable using weather (e.g., barometric pressure or wind) and temperature as predictors of building pressurization and from that, contaminant entry rates.
Historically, the most widely held view of vapor intrusion has been that it is advective entry dominated, and indeed at many sites that is the case. But where this assumption does not hold can lead to confusion when interpreting results.
