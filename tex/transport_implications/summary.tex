\section{Summary}

In this chapter, the importance of considering the dominant transport mechanism of contaminant vapors into a building is explored.
Advective transport is likely to only be dominant at VI site with some feature or pathway that allows air to by-pass the large resistance to transport of the surrounding soil.
This in turn increases the importance of finding these sort of features, as these are likely what will make using building pressurization as an ITS or using a technique such as CPM effective tool in VI site investigations.
We also explore using weather and temperature as a predictor of building pressurization, which can be used to help explain how VI impacted buildings respond in different climates. 
