\documentclass[../thesis.tex]{subfiles}

% graphics path
\graphicspath{
  {../figures/transport_implications/},
  {../../figures/transport_implications/},
  {../figures/preferential_pathways/},
  {../../figures/preferential_pathways/},
}

\begin{document}
\chapter{Implications Of Considering Transport Type In Vapor Intrusion Investigations}

\import{./}{summary.tex}
\import{./}{soil_peclet.tex}
\import{./}{investigation_proposal.tex}
\import{./}{pressure_prediction.tex}



\begin{comment}

Previous chapter review:
- Pe number analysis shows why different relationship between p_in and c_in
- May be indicative of a broader problem in VI investigations. Implications:
  - CPM
  - Use of ITS
  - Etc

- Shows importance to consider nature of transport at site for effective applications of this
  - Introduce flowsheet for investigation paradigm

- Seasonal aspect of this
  - Why higher c_in during winter
  - Explain with p_in for ASU and Indie (seasonal distribution of these)

- Predicting building pressurization (When are they the largest)
  - Show that we would predict that c_in and p_in is higher during

- Predicting air exchange rate (when is it the smallest?)

- How common
- Will explore some cases of this. Occur through modeling
  - Consider:
    - Soil type
    - Pressure
    - Depth

- Many soil cannot support flowrates fast enough
  - Advective transport likely to occur through
    - Most permeable soil types
    - Various site characteristics can facilitate this
      - Preferential pathways obvious one
        - Give examples on how to find these
      - Examples of other site possible cases (speculative)

- Determining advective/diffusive transport becomes the new issue


\end{comment}

\end{document}
